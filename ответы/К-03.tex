\documentclass[__main__.tex]{subfiles}

\begin{document}

\qtitle{К}{03}
Вычислите постоянную Стефана-Больцмана, воспользовавшись формулой Планка.\\


Попробуем вычислить постоянную С-Б при помощи формулы Планка:
\begin{gather*}
    U_\omega(T)=\frac{\hbar\omega^3}{\pi^2c^3}\cdot\frac{1}{e^{\frac{\hbar\omega}{K_BT}}-1}
\end{gather*}
Запишем энергетическую светимость:
\begin{gather*}
    R^\ast(T)
    =
    \int_{0}^{\infty}d\omega r^\ast_\omega(T)
    =
    \int_{0}^{\infty}d\omega\frac{c}{4}U_\omega(T)
    =\\
    =
    \left.
    \int_{0}^{\infty}\frac{\hbar\omega^3}{4\pi^2c^2}\cdot\frac{d\omega}{e^{\frac{\hbar\omega}{K_BT}}-1}
    \right|_{x=\frac{\hbar\omega}{K_BT}}
    =
    \frac{\hbar}{4\pi^2c^2}\left(\frac{K_BT}{\hbar}\right)^4\int_{0}^{\infty}\frac{x^3\cdot dx}{e^x-1}
\end{gather*}
Посчитаем интеграл отдельно:
\begin{gather*}
    \int_{0}^{\infty}\frac{x^3dx}{e^x-1}
    =
    \int_{0}^{\infty}dx\cdot x^3\frac{e^{-x}}{1-e^{-x}}
    =
    \int_{0}^{\infty}dx\cdot x^3\sum_{p=1}^{\infty}e^{-px}
    =\\
    \left.
    \sum_{p=1}^{\infty}\int_{0}^{\infty}dx\cdot x^3e^{-px}
    \right|_{y=-px}
    =
    \sum_{p=1}^{\infty}\left(\frac{1}{p^4}\right)\int_{0}^{\infty}dy\cdot y^3e^{-y}
    =
    \zeta(4)\cdot\Gamma(4)
    =
    \frac{\pi^4}{15},
\end{gather*}
где $\zeta(y)$ -- дзета-функция Римана.
Получаем $\sigma=\frac{\pi^2K_B^4}{60c^2\hbar^3}$, т.к. з-н Стефана-Больцмана: $R^\ast(T)=\sigma T^4$


\end{document}