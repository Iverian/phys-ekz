\documentclass[__main__.tex]{subfiles}

\begin{document}

\qtitle{Ч}{19}
Постулат симметризации для частиц с полуцелым спином. Определитель Фока-Слэтера. Принцип запрета Паули. Функция распределения Ферми-Дирака. Газ невзаимодействующих электронов в основном состоянии, энергия Ферми.\\ 

\textbf{Постулат симметризации для частиц с полуцелым спином: } состояния системы, содержащей N тождественных частиц, будут все антисимметричными относительно перестановок этих N частиц.\\

\textbf{Определитель Фока-Слэтера}
\begin{definition}
Определитель Фока-Слэтера - антисимметричная относительно перестановки частиц волновая функция многочастичной квантовомеханической системы, построенная из одночастичных функций.
\end{definition}

Эта волновая функция для системы из N идентичных частиц будет иметь вид:

$$\psi(\vec{r_1}, \vec{r_2}, \cdots, \vec{r_n}) = \frac{1}{\sqrt{N!}}\begin{vmatrix} 
\psi_1(\vec{r_1}) & \psi_2(\vec{r_1}) & \cdots & \psi_i(\vec{r_1}) & \cdots & \psi_N(\vec{r_1}) \\ 
\psi_1(\vec{r_2}) & \psi_2(\vec{r_2}) & \cdots & \psi_i(\vec{r_2}) & \cdots & \psi_N(\vec{r_2}) \\ 
 &  &  & \vdots &  & \\ 
\psi_1(\vec{r_i}) & \psi_2(\vec{r_i}) & \cdots & \psi_i(\vec{r_i}) & \cdots & \psi_N(\vec{r_i}) \\
&  &  & \vdots &  & \\
\psi_1(\vec{r_N}) & \psi_2(\vec{r_N}) & \cdots & \psi_i(\vec{r_N}) & \cdots & \psi_N(\vec{r_N}) \end{vmatrix}$$

где $\psi_i(\vec{r})$ - линейно независимые одночастичные волновые функции.

\begin{definition}
Фермион - частица (или квазичастица) с полуцелым значением спина. Примеры: кварки (образуют протоны и нейтроны, которые также являются фермионами), лептоны (электроны, мюоны, нейтрино).
\end{definition}

\textbf{Принцип запрета Паули:} В замкнутой квантовой системе из одинаковых фермионов в одном полностью определенном состоянии не может находиться более одной частицы.\\

\textbf{Функция распределения Ферми-Дирака}

Значение химического потенциала $\mu$ при нулевой температуре в статистической физики принято называть энергией Ферми: 
$$\varepsilon_f \equiv \mu \; (T = 0)$$

Функция распределения Ферми-Дирака определяет вероятность заселения (из-за двух возможнх ориентаций спина на каждом уровне энергии могут находиться 2 электрона) уровня с энергией $\varepsilon$ и имеет вид:

$$f(\varepsilon) = 1/(e^{\frac{\varepsilon - \varepsilon_f}{kT}} + 1)$$

\textbf{Газ невзаимодействующих электронов в основном состоянии}

Из-за того, что принцип Паули запрещает находиться двум фермионам (а электроны к ним относятся) в совершенно одинаковых состояниях, а система находится в состоянии с минимальной энергией (основное состояние), поэтому все состояния с энергиями $$\varepsilon_n \le \varepsilon_f$$ будут заполнены электронами, т.е:

$$n_f = \begin{cases}1, & \forall \varepsilon_n \le \varepsilon_f \\ 0, & \forall \varepsilon_n > \varepsilon_f \end{cases}$$

Рассмотрим энергию Ферми:

$$\varepsilon_f = \frac{p_f^2}{2m} = \frac{\hbar^2k_f^2}{2m} = \frac{\hbar^2 \pi^2}{2ml^2}n_f^2$$

Так как число частиц совпадает с числом состояний, то получаем:

$$N = \frac{\pi}{3}n_f^3$$

$$n_f^2 = \left(\frac{3}{\pi}N\right)^{2/3}$$

$$\varepsilon_f = \frac{\hbar^2}{2m}\left(3\pi^2\frac{N}{V}\right)^{2/3}$$

\end{document}