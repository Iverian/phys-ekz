\documentclass[__main__.tex]{subfiles}

\begin{document}

\qtitle{Ч}{05}
Гильбертово пространство состояний системы из двух частиц (а) разных сортов, (б) одного сорта. Принцип тождественности частиц. Постулат симметризации, принцип запрета Паули.\\ 

\textbf{Принцип тождественности частиц}\\
В системе одинаковых частиц реализуются только такие состояния, которые не меняются при перестановке местами двух любых частиц.\\
\textbf{Постулат симметризации}\\
\begin{definition}
Состояния системы, содержащей N тождественных частиц, будут все либо симметричными, либо антисимметричными относительно перестановок этих N частиц.\\
\end{definition}
\textbf{Принцип запрета Паули}\\
Рассмотрим систему из двух фермионов, то получаются антисимметричные функции в качестве решений:
\begin{gather}
\Psi_{ij}(1,2) = -\Psi_{ij}(2,1)
\end{gather}
Тогда, чтобы получить решение, соответствующее реальности нам надо альтернировать:\\
\begin{gather}
\Psi_{ij}(1,2)=\frac{1}{\sqrt{2}}(\phi_i(1)\phi_j(2)+\phi_i(2)\phi_j(1)) = \frac{1}{\sqrt{2}} \begin{vmatrix}
\phi_i(1) & \phi_i(2)\\
\phi_j(1) & \phi_j(2)
\end{vmatrix}
\end{gather}
Значит, если положить в последнем выражении $i=j$, то определитель занулится $\Longrightarrow$ система из двух одинаковых фермионов не может находится в состоянии $\Psi$ содержащих одинаковые одночастичные состояния $\phi$ - это мы получили " Принцип запрета Паули".
\end{document}