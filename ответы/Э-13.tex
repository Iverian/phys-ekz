\documentclass[__main__.tex]{subfiles}

\begin{document}

\qtitle{Э}{13}
Воспользуйтесь законом Био-Савара-Лапласа, чтобы вычислить магнитное поле прямого постоянного тока $I$ на расстоянии $R$ от него. Проверьте результат с помощью теоремы о циркуляции $\vec{B}$.\\ 

\textbf{Закон Био - Савара}

	Рассмотрим вопрос о нахождении магнитного поля, создаваемого постоянными электрическими токами. Этот вопрос будем решать, исходя из закона 
	\begin{gather}
	\llabel{1}
	\vec{B} = \frac{\mu_0}{4\pi} \frac{q[\vec{v},\vec{r}]}{r^3},
	\end{gather}
	который определяет индукцию поля $\vec{B}$ равномерно  движущегося точеченого заряда. Подставим в \lref{1} вместо $q$ заряд $\rho dV$, где $dV$ - элементарный объём, $\rho$ - объёмная плотность заряда, являющегося носителем тока и учтём, что $\rho \vec{v} = \vec{j}$ согласно определению вектора плотности тока. Тогда формула \lref{1} приобретёт следующий вид:
	
	\begin{gather}
	\llabel{2}
	d\vec{B}= \frac{\mu_0}{4\pi} \frac{[\vec{j},\vec{r}] dV}{r^3}.
	\end{gather}
	
	Если же ток $I$ течёт по тонкому проводу с площадью поперечного сечения $\Delta S$, то 
	
	$$j dV = d \Delta S dl = l dl,$$
	
	где $dl$ - элемент длины провода. Введя векктор $d\vec{l}$ в направлении тока $I$, перепишем предыдущее равенство так:
	
	$$j dV = I d\vec{l}.$$
	
	Векторы $\vec{j} dV$ и $I d\vec{l}$ называются соответственно \textit{объёмным и линейным элементами тока}. Произведя в формуле \lref{2} замену объёмного элемента тока на линейный получим
	
	\begin{gather}
	\llabel{3}
		d\vec{B} = \frac{\mu_0}{4\pi} \frac{I[d\vec{l},\vec{r}]}{r^3}
	\end{gather}
	
	Формулы \lref{2} и \lref{3} выражают \textit{закон Био-Савара}.
	
	Полное поле $\vec{B}$ в соответствии с принципом суперпозиции определяется в результате интегрирования выражений \lref{2} или \lref{3} по всем элементам тока:
	
	$$\vec{B} = \frac{\mu_0}{4\pi} \int \frac{[\vec{j},\vec{r}]}{r^3}, \ \ \ \vec{B} = \frac{\mu_0}{4\pi}\oint \frac{I[d\vec{l},\vec{r}]}{r^3} $$
	
	\textbf{Теорема о циркуляции вектора $\vec{B}$}
	Циркуляция вектора $\vec{B}$ по произвольному контуру $\Gamma$ равна произведению $\mu_0$ на алгебраическую сумму токов, охватываемых контуром $\Gamma$:
	
	\begin{gather}
	\oint \vec{B} d\vec{l} = \mu_0 I,
	\end{gather}
	
	где $I = \sum I_k$.
%%

\end{document}