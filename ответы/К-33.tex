\documentclass[__main__.tex]{subfiles}

\begin{document}

\qtitle{К}{33}
Получите с помощью формулы Планка приближённые выражения для объёмной спектральной плотности излучения в области (а) $\hbar\omega\ll k_{B}T$, (б) $\hbar\omega\ll k_{B}T$. Преобразуйте формулу Планка для объемной спектральной плотности излучения $u_{\omega}(T)$ от переменной $\omega$ к переменной $\lambda$.\\ 

\begin{definition}
	Формула Планка — выражение для спектральной плотности мощности излучения (спектральной плотности энергетической светимости) абсолютно чёрного тела, которое было получено Максом Планком для плотности энергии излучения $u(\omega,T)$ 
\end{definition}
\begin{flalign}
u(\omega,T)=\frac{\hbar \omega^3}{4 \pi^2 c^2} \cdot \frac{1}{e^\frac{\hbar\omega}{k_{B} T}-1}
\end{flalign}

где $k_B$ - это постоянная Больцмана.\\

\textbf{Примечание: пункты (а) и (б) в задании абсолютно идентичны, поэтому пишу только для (а)}

Так как $\hbar\omega \ll k_{B}T$, то из разложения Тейлора следует, что $e^{\frac{\hbar\omega}{kT}} \approx 1 + \frac{\hbar \omega}{k_{B}T}$. Таким образом, путем замены из формулы Планка легко выводится формула Рэлея-Джинса:

$$u(\omega, T) = k_{B}T\frac{\omega^2}{4\pi^2 c^2}$$

Чтобы преобразовать формулу Планка от переменной $\omega$ к $\lambda$, воспользуемся тем, что:

$$\int_0^{\infty} u(\omega, T) d\omega = \int_0^{\infty} u(\lambda, T) d\lambda$$

Так как $\omega = \frac{2\pi c}{\lambda}$, $u(\omega, T)d\omega = -u(\lambda,T)d\lambda$, $\frac{d\omega}{d\lambda} = -\frac{2\pi c}{\lambda^2}$ и учитывая $\hbar = \frac{h}{2\pi}$, то
получаем, что:

$$u(\lambda, T) = \frac{2\pi c}{\lambda^2} u(\omega, T) = \frac{2\pi h c^2}{\lambda^5} \cdot \frac{1}{e^\frac{hc}{k_B T \lambda} - 1}$$

\end{document}