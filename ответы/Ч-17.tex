\documentclass[__main__.tex]{subfiles}

\begin{document}

\qtitle{Ч}{17}
Найти при $T=0$ (а) максимальную кинетическую энергию свободных электронов в металле, если известна их концентрация $\nu$, (б) определить отношение средней кинетической энергии к максимальной.\\

\begin{enumerate}
    \item
          При нулевой температуре максимальная кинетическая энергия свободных электронов в атоме совпадает с энергией Ферми $\varepsilon_F(0)$. Тогда получаем, что:
          $$
              \varepsilon_{max} = \varepsilon_F(0) = \frac{\hbar^2}{2m}(3\pi^2\nu)^{\frac{2}{3}}
          $$
    \item
          Найдем среднюю кинетическую энергию свободных электронов:
          $$
              \left<\varepsilon\right>=\frac{\int_0^{\varepsilon_F(0)}\varepsilon g(\varepsilon) d\varepsilon}{\int_0^{\varepsilon_F(0)}g(\varepsilon) d\varepsilon} = \frac{3}{5}\varepsilon_F(0)
          $$
          где $g(\varepsilon) = \frac{1}{2\pi^2}\left(\frac{2m}{\hbar^2}\right)^{\frac{3}{2}}\sqrt{\varepsilon}$ -- функция плотности числа состояний. Таким образом получаем, что отношение средней кинетической энергии к максимальной равно $\frac{3}{5}$.
\end{enumerate}




\end{document}