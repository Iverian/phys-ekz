\documentclass[__main__.tex]{subfiles}

\begin{document}

\qtitle{О}{02}
Плоская монохроматическая световая волна с интенсивностью $I_0$ падает нормально на непрозрачный экран с круглым отверстием. Чему равна интенсивность света за экраном в точке, для которой отверстие сделали равным первой зоне Френеля, после чего закрыли по диаметру его половину?\\ 

При всех открытых зонах Френеля амплитуда равна $A_00$
$$I \sim A^2 \Rightarrow I_0 \sim A_{00} ^ 2$$
При первой открытой зоне $A=A_1=2A_{00}$
$$I\sim A^2 \Rightarrow 4A_00 ^2 = 4I_0 $$
При половине первой зоны амплитуда равна:
$$A\sim A_{0.5} = \sqrt{A_{00}^2 + A_{00}^2}=A_{00} \sqrt{2}$$
$$I\sim A^2 = 2A_{00}^2 = 2I_0$$
При закрытой половине первой зоны амплитуда будет равна 
$$A=\frac{A_1}{2}=A_{00} \Rightarrow I \sim A^2 = A_{00}^2 =I_0$$
\end{document}