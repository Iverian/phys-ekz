\documentclass[__main__.tex]{subfiles}

\begin{document}

\qtitle{К}{22}
Продемонстрируйте сохранение квадрата нормы волновой функции во времени. Вектор плотности тока вероятности. Получите уравнение непрерывности для плотности вероятности.\\

У нас имеется уравнение Шрёдингера: $\hat\mathcal{H}\left[\Psi(t,\vec{r})\right]=i\hbar\partial_{t}\Psi(t,\vec{r})$. Мы делаем допущение, что потенциальная энергия в операторе Гамильтона $\hat{\mathcal{H}}$ не зависит явно от времени, и получаем возможность использовать метод разделения переменных Фурье для нахождения функции $\Psi$. В состоянии типа стационарного:
\begin{gather*}
    \Psi(t,\vec r)
    =
    \exp\left(-\frac{i}{\hbar}\varepsilon t\right)\psi(\vec{r})
\end{gather*}
квадрат нормы пси-функции не зависит от времени: $\Vert\Psi\Vert^2=1\cdot\Vert\psi(\vec r)\Vert^2.$
\begin{flalign*}
    &
    \partial_t\Vert\Psi\Vert^2
    =
    \partial_t\left(\Psi,\Psi\right)
    =
    \left(\partial_t\Psi,\Psi\right)+(\Psi,\partial_t\Psi)
    =
    \left|\partial_t\Psi=\frac{1}{i\hbar}\hat\mathcal{H}\Psi\right|
    =\\
    =
    &
    \left(\frac{1}{i\hbar}\hat\mathcal{H}\Psi,\Psi\right)+\left(\Psi,\frac{1}{i\hbar}\hat\mathcal{H}\Psi\right)
    =
    \frac{1}{i\hbar}\left[-(\Psi,\hat{\mathcal H}\Psi)+(\hat{\mathcal H}\Psi,\Psi)\right]
    =\\
    =
    &
    \frac{1}{i\hbar}\left[-(\hat{\mathcal H}^*\Psi,\Psi)+(\hat{\mathcal H}\Psi,\Psi)\right]
    =
    \left|\hat{\mathcal H}^*=\hat{\mathcal H}\right|
    =\\
    =
    &
    0.
\end{flalign*}
Получим вектор плотности тока вероятности $\vec{j}$:
\begin{flalign*}
    &
    \partial_{t}\Psi^*\Psi
    =
    \partial_{t}\Psi^*\Psi+\Psi^*\partial_{t}\Psi
    =
    -\frac{1}{i\hbar}\hat{\mathcal H}\Psi^*\Psi+\Psi^*\frac{1}{i\hbar}\hat{\mathcal H}\Psi
    =
    \left|\hat{\mathcal H}=-\frac{\hbar^2}{2m}\Delta+U(\vec r)\right|
    =\\
    =
    &
    \frac{\hbar}{2im}\left(\Delta\Psi^*\Psi-\Psi^*\Delta\Psi\right)
    =
    -\nabla\operatorname{Re}\left(\Psi^*\frac{\hbar}{im}\nabla\Psi\right)
    =\\
    =
    &
    -\nabla\vec{j}(t,\vec{r}).
\end{flalign*}
Обозначим $\Psi^*(t,\vec r)\cdot\Psi(t,\vec r)=p(t,\vec r)$ -- плотность вероятности, тогда получим уравнение непрерывности:
\begin{gather*}
    \partial_{t}p(t,\vec{r})+\nabla\vec{j}(t,\vec{r})=0.
\end{gather*}

\end{document}