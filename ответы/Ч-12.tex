\documentclass[__main__.tex]{subfiles}

\begin{document}

\qtitle{Ч}{12}
Квантовые статистики: функция распределения Бозе-Эйнштейна, функция распределения Ферми-Дирака. Классический предел квантовых статистик.\\ 

\textbf{Функция распределения Бозе-Эйнштейна}

Функция распределения Бозе-Эйнштейна для частиц с целым (в том числе нулевым) спином имеет вид:

$$f(\varepsilon) = 1/(e^{\frac{\varepsilon - \mu}{kT}} - 1)$$

где $\varepsilon$ - кинетическая энергия частицы, $\mu$ - химический потенциал, зависящий от температуры.

\textbf{Функция распределения Ферми-Дирака}

Функция распределения Ферми-Дирака определяет вероятность заселения (из-за двух возможнх ориентаций спина на каждом уровне энергии могут находиться 2 электрона) уровня с энергией $\varepsilon$ и имеет вид:

$$f(\varepsilon) = 1/(e^{\frac{\varepsilon - \varepsilon_F}{kT}} + 1)$$

Здесь $\varepsilon_F$ - энергия Ферми, параметр, определяемый из очевидного условия, что сумма заселенности всех уровней энергии должна равняться полному числу электронов:

$$\sum f(\varepsilon) = N_\varepsilon$$

\textbf{Классический предел квантовых статистик.}\\

\textit{Распределение Бозе-Эйнштейна}\\

При повышении температуры химический потенциал $\mu$ уменьшается, и может наступить момент, когда отношение $\frac{\varepsilon - \mu}{kT}$ становится заметно больше единицы. Тогда единицей в знаменателе распределения Бозе-Эйнштейна можно пренебречь, и:

$$f(\varepsilon) \approx e^\frac{\mu}{kT} e^{-\frac{\varepsilon}{kT}} \ll 1$$

т.е квантовые свойства системы становятся несущественными и распределение переходит в классическое распределение Больцмана.\\

\textit{Распределение Ферми-Дирака}\\

Пока $kT \ll \varepsilon_F(0)$ справедлива приближенная формула:

$$\varepsilon_F(T) \approx \varepsilon_F(0)\left(1 - \frac{\pi^2}{12}\left(\frac{kT}{\varepsilon_F(0)}\right)^2\right)$$

При большей температура электроны распределяются в широком интервале энергий. При дальнейшем повышении температуры наступает момент, когда энергия Ферми уходит в область, где нет уровней энергии, при этом выполняется условие $\varepsilon - \varepsilon_F(T) \gg kT$ для всех энергетических уровней $\varepsilon$ и тогда:

$$exp\left(\frac{\varepsilon - \varepsilon_F}{kT} \right) \gg 1$$

Тогда единицей в знаменателе функции распределения Ферми-Дирака можно пренебречь и получается классическое распределение Больцмана:

$$f(\varepsilon) \approx e^{\varepsilon_F/kT} e^{-\varepsilon/kT}$$

\end{document}