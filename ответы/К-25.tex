\documentclass[__main__.tex]{subfiles}

\begin{document}

\qtitle{К}{25}
Атом водорода. Собственные значения и собственные функции операторов $\hat{L}^2$ и $\hat{L}_z$, кратность вырождения энергетического уровня, отвечающего заданному главному квантовому числу $n$.\\ 

\textbf{Атом водорода}
Из всего многообразия элементов, мы рассмотрим атом водорода, потому что это самый простой элемент, то есть $H$ имеет один протон и один электрон. Так же, существует закономерность, которая позволяет получать огромное количество спектральных линий, комбинируя значительно меньшее число велечин, называемых \textit{термами}. Для атома водорода:\\
\begin{gather}
T_n = \frac{R_H}{n^2}, \qquad n=1,2...
\end{gather}
где $R_H$ - постоянная Ритберга.\\
\textbf{Собственные значения и собственные функции операторов $\hat{L}^2$ и $\hat{L}_z$}\\
Так как оператор $\hat{L}^2$ коммутирует со всеми проекциями оператора момента количества движения:
\begin{gather}
[\hat{L}^2,\hat{L}_x] = [\hat{L}^2,\hat{L}_y] = [\hat{L}^2,\hat{L}_z] = 0
\end{gather}
Следовательно, у операторов $\hat{L}^2$ и $\hat{L}_z$ должны существовать общие собственные состояния $\left|\lambda,\mu\right>$
\begin{gather}
\begin{cases}
\hat{L}_z \left|\lambda,\mu\right> = \mu\left|\lambda,\mu\right>\\
\hat{L}^2 \left|\lambda,\mu\right> = \lambda\left|\lambda,\mu\right>\\
\end{cases}
\end{gather}
где $\lambda$ - собственное значение квадрата момента количества движения; $\mu$ - собственное значение оператора проекции момента количества движения на координатную ось $z$.\\
Так как
\begin{gather}
\hat{L}_z = x\hat{p}_y-y\hat{p}_x
\end{gather}
Рассмотрим функцию $\Psi = \Psi(r,\theta,\phi)$, зависящую от сферических координат. Тогда при повороте системы на угол $\delta \phi$:
\begin{gather}
\Psi(r,\theta,\phi+\delta\phi) = \Psi(r,\theta,\psi)+\delta \phi \frac{\partial \Psi}{\partial \phi} = \left(1+\delta \phi \frac{\partial}{\partial \phi}\right) \Psi
\end{gather}
По определению оператора поворота:
\begin{gather}
\Psi(r,\theta,\psi+\delta \phi) = \left(1+\frac{i}{\hbar}\delta \phi \hat{L}_z\right)\Psi
\end{gather}
Тогда выражение для $\hat{L_z}$ выглядит следующим образом:
\begin{gather}
\hat{L}_z = -i\hbar\frac{\partial}{\partial \phi}
\end{gather}
Следовательно
\begin{gather}
-i\hbar \frac{\partial \Psi}{\partial \phi} = \mu \Psi \Longrightarrow \frac{\partial \Psi}{\Psi} = \frac{i}{\hbar}\mu \delta \phi\\
\Psi = \Phi(r,\theta)e^{\frac{i}{\hbar}\mu \phi}
\end{gather}
Функция $\Psi$, выраженная таким образом, является собственной функцией проекции момента. Зависимость от угла $\phi$ в сферической системе координат, где
\begin{gather}
0 \leq \phi \leq 2\pi \qquad 0\leq \theta \leq \pi
\end{gather}
выражается экспонентой с мнимым показателем.\\
Рассмотрим уравнение:
\begin{gather}
\hat{H}\Psi(r,\theta,\phi) = \epsilon \Psi(r,\theta,\phi)
\end{gather}
Разделим углавые и радиальную переменные $\Psi(r,\theta,\phi) = \rho(r)Y(\theta,\phi)$
\begin{gather}
\frac{2mr^2}{\hbar^2\rho Y} \left(-Y\frac{\hbar^2}{2mr^2}\partial_r(r^2\partial_r \rho) + \rho \frac{\hat{L}^2Y}{2mr^2} + (V_{(r)}-\epsilon)\rho Y\right) = 0\\
[-\frac{1}{\rho} \partial_r(r^2\partial_r \rho) + \frac{2mr^2}{\hbar^2}(V_{(r)}-\epsilon) ]+[ \frac{\hat{L}^2 Y(\theta,\phi)}{\hbar Y(\theta,\phi)}] = 0
\end{gather}
где выражение в первых квадратных и во вторых квадратных скобках просто числа, тогда $Y(\theta,\phi)$ есть собственная функция оператора $\hat{L}^2$, а поскольку $[\hat{L}^2,\hat{L}_z] = 0$, то является и собственной функцией оператора $\hat{L}_z$.\\
\textbf{Кратность вырождения энергетического уровня, отвечающего заданному главному квантовому числу $n$}\\
\begin{definition}
$l$ - орбитальное квантовое число\\
$m$ - магнитное квантовое число\\
$n$ - главное квантовое число\\
\end{definition}
Состояние с определенной энергией и моментом импульса принято указывать $nl$\\
\begin{gather}
n = l + n_r + 1 \qquad l = 0,1,2...; \ \ n_r = 0,1,2... \Longrightarrow n = 0,1,2...
\end{gather}
где $n_r$ - число узлов радиальной части волновой функции. Теперь зафиксируем $n$ и посмотрим на $n_r$, можно увидеть, что $l = 0,1,2...n-1$. Так же $m = -l,-l+1 ... l-1,l$. Теперь посчитаем кратность вырождения энергетического уровня, отвечающего заданому главному квантовому числу $n$:\\
\begin{gather}
\gamma \sum_{l=0..n-1}(2l+1) = \gamma (2\frac{n(n-1)}{2}+n) = \gamma n^2
\end{gather}
где $\gamma$ - число возможных значений проекции спина на выделенное направление.\\
\end{document}