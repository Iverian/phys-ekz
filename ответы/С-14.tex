\documentclass[__main__.tex]{subfiles}

\begin{document}

\qtitle{С}{14}
Энергия системы точечных зарядов. Ёмкость уединённого проводника. Энергия уединённого заряженного проводника. Ёмкость конденсатора. Энергия конденсатора.\\ 

\emph{Энергия системы точечных зарядов:} рассмотрим систему двух точечных зарядов $q_1$ и $q_2$. Когда заряды бесконечно удалены друг от друга они не взаимодействуют, следовательно, положим их потенциальную энергию равной нулю. Сблизим их на расстояние $r_{12}$, при этом совершится работа для переноса $q_1$:
\begin{gather}
A_{1}=q_{1}\varphi_{1}=q_{1}\frac{1}{4\pi\varepsilon_{0}}\frac{q_2}{r_{12}},
\end{gather}
где $\varphi_{1}$ -- потенциал, создаваемый $q_{2}$ в точке, в которую перемещается $q_{1}$. Аналогично для $q_{2}$:
\begin{gather}
A_{2}=q_{2}\varphi_{2}=q_{2}\frac{1}{4\pi\varepsilon_{0}}\frac{q_{1}}{r_{12}},
\end{gather}
значения $A_{1}$ и $A_{2}$ одинаковы и выражает энергию системы $W$:
\begin{gather}
W=q_{1}\varphi_{1}=q_{2}\varphi_{2}
=
\frac{1}{2}\left(q_{1}\varphi_{1}+q_{2}\varphi_{2}\right),
\llabel{s14:two}
\end{gather}
формула (\lref{s14:two}) выражает \emph{энергию системы двух зарядов}.

Перенесем из бесконечности заряд $q_{3}$ и расположим их на расстоянии $r_{13}$ и $r_{23}$ от $q_{1}$ и $q_{2}$ соответственно, при этом совершим работу:
\begin{gather}
A_{3}=q_{3}\varphi_{3}=q_{3}\frac{1}{4\pi\varepsilon_{0}}\left(\frac{q_1}{r_{13}}+\frac{q_{2}}{r_{23}}\right),
\end{gather}
где $\varphi_{3}$ -- потенциал, создаваемый зарядами $q_1$ и $q_2$ в точке, в которую помещен $q_3$. Сумма работ $A_1$, $A_2$ и $A_3$ равна энергии системы:
\begin{flalign}
\begin{split}
W
=&
\frac{1}{4\pi\varepsilon_{0}}\frac{q_1q_2}{r_{12}}
+
q_{3}\frac{1}{4\pi\varepsilon_{0}}\left(\frac{q_1}{r_{13}}+\frac{q_{2}}{r_{23}}\right)
=\\
=&
\frac{1}{2}\frac{1}{4\pi\varepsilon_{0}}
\left[
q_1\left(\frac{q_2}{r_{12}}+\frac{q_3}{r_{13}}\right)
+
q_2\left(\frac{q_1}{r_{12}}+\frac{q_3}{r_{23}}\right)
+
q_3\left(\frac{q_1}{r_{13}}+\frac{q_2}{r_{23}}\right)
\right]
=\\
=&
\frac{1}{2}\left(q_{1}\varphi_{1}+q_{2}\varphi_{2}+q_{3}\varphi_{3}\right).
\end{split}
\end{flalign}
Индукцией можно показать, что \textbf{энергия системы $k$ свободных зарядов}:
\begin{gather}
W=\frac{1}{2}\sum_{i\in\mathbb{N}_k}q_{i}\varphi_{i},
\qquad
\varphi_{i}=\sum_{j\in\mathbb{N}_{k}\backslash\{i\}}\frac{q_{j}}{r_{ij}},
\llabel{s14:sys}
\end{gather}
где $\mathbb{N}_{k}=\left\{n|n\in\mathbb{N}\wedge{n\le k}\right\}$.

\begin{definition}
    \emph{Емкостью уединенного проводника} с зарядом $q$ называют величину:
    \begin{gather}
        C=\frac{q}{\varphi},
        \llabel{s14:csin}
    \end{gather}
    где $\varphi$ -- потенциал, создаваемого этим проводником электрического поля.
\end{definition}
$C$ зависит только от размеров и формы проводника, а также от окружающего его диэлектрика.

\emph{Энергия уединенного проводника:} заряд $q$, находящийся на уединенном проводнике, можно рассматривать как систему точечных зарядов $\Delta{q}$. Такая система обладает энергией, равной работе требуемой для переноса всех порций $\Delta{q}$ из бесконечности на их позиции на поверхности проводника:
\begin{gather}
\Delta{A}=\varphi\Delta{q}=\frac{q}{C}\Delta{q},
\llabel{s14:rab}
\end{gather}
где $\varphi$ -- потенциал проводника, создаваемый уже имеющимся на нем зарядом $q$. Работа (\lref{s14:rab}) идет на увеличение энергии проводника, тогда:
\begin{flalign}
\begin{split}
&
dW=\frac{1}{C}qdq
\Longrightarrow
W=\frac{1}{C}\int\limits_{0}^{q}\xi{d\xi}=\frac{q^2}{2C},
\end{split}
\end{flalign}
тогда получим из (\lref{s14:csin}):
\begin{gather}
W=\frac{q^2}{2C}=\frac{q\varphi}{2}=\frac{C\varphi^2}{2}.
\end{gather}
Тот же результат можно получить из (\lref{s14:sys}) представив проводник как систему элементарных зарядов $\Delta{q}$ и зная, что поверхность проводника эквипотенциальна $\forall{i\in\mathbb{N}_k}\colon\varphi_{i}=\varphi$.


\begin{definition}
    Конденсатор -- физическое устройство, способное накапливать и сохранять электрический заряд и энергию электрического поля.
\end{definition}
Простейшая конструкция конденсатора -- два заряженных проводника с равными по модулю, но разными по знаку, зарядами, разделенные диэлектриком. Основная характеристика конденсатора -- его емкость $C$:
\begin{gather}
    C=\frac{q}{\varphi_1-\varphi_2}=\frac{q}{U},
    \llabel{s14:ccond}
\end{gather}
где $q$ -- заряд одного проводника (\emph{одной обкладки}), $\varphi_1$, $\varphi_2$ -- потенциалы обкладок.

\emph{Энергия конденсатора:} разделим конденсатор на систему элементарных зарядов, тогда каждый из элементарных зарядов, на которые можно разделить $+q$ находится в точке с потенциалом $\varphi_1$, аналогично $-q$ -- в $\varphi_2$, тогда из (\lref{s14:sys}):
\begin{gather}
W
=
\frac{1}{2}\left[(+q)\varphi_{1}+(-q)\varphi_{2}\right]
=
\frac{1}{2}q(\varphi_1-\varphi_2)=\frac{1}{2}qU
\end{gather}
тогда из (\lref{s14:ccond}) получим:
\begin{gather}
W=\frac{q^2}{2C}=\frac{qU}{2}=\frac{CU^2}{2},
\end{gather}

\end{document}