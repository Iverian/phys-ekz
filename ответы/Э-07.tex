\documentclass[__main__.tex]{subfiles}

\begin{document}

\qtitle{Э}{07}
Индукция магнитного поля в вакууме вблизи плоской поверхности однородного изотропного магнетика (магнитная проницаемость $\mu$) равна $\vec{B}$, причём вектор $\vec{B}$ составляет угол $\alpha$ с нормалью к поверхности. Найдите индукцию магнитного поля в магнетике вблизи поверхности.\\ 

Нам надо найти модуль вектора $\vec{B}'$ -- магнитную индукцию внутри магнетика. Как и у любого вектора, его модуль можно вычислить из нормальной и касательной составляющих:
\begin{gather*}
B'=\sqrt{(B'_n)^2+(B'_\tau)^2}
\end{gather*}
Попробуем разобраться, что из себя представляют эти составляющие:
\begin{gather*}
B'_n=B_n=B\cos\alpha \\
B'_\tau=\mu_0H'_\tau=\mu_0\mu H_\tau=\mu B_\tau=\mu B\sin\alpha, 
\end{gather*}
где $H'_\tau$ -- напряженность магнитного поля в магнетике вблизи к поверхности, $H_\tau$ --- напряженность магнитного поля в вакууме вблизи к поверхности.\\
И тогда:
\begin{gather*}
B'=\sqrt{B^2\cos^2\alpha+\mu^2B^2\sin^2\alpha}\\
B'=B\sqrt{\cos^2\alpha+\mu^2\sin^2\alpha}
\end{gather*}
\end{document}