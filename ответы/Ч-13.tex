\documentclass[__main__.tex]{subfiles}

\begin{document}

\qtitle{Ч}{13}
Электрон находится в состоянии, описываемом кет-вектором $\frac{\left|+\right>+\left|-\right>}{\sqrt{2}}$. Найдите вероятность того, что проекция его спина на направление $\vec{n}$, составляющее угол $\frac{\pi}{4}$ с осями $Ox$, $Oy$, $Oz$, окажется равной $+\frac{\hbar}{2}$.\\ 

Наверное, Никифоров имел ввиду, что $\vec{n}$ имеет равные углы со всеми осями, т.е.
\begin{gather}
\vec{n}=\frac{1}{\sqrt{3}}\vec{i}+\frac{1}{\sqrt{3}}\vec{j}+\frac{1}{\sqrt{3}}\vec{k},
\end{gather}
Распишем $\hat{\sigma}$ в базисе $\vec{i}$, $\vec{j}$, $\vec{k}$.
\begin{flalign}
&
\hat{\sigma}
=
\hat{\sigma}_{x}\vec{i}+\hat{\sigma}_{y}\vec{j}+\hat{\sigma}\vec{k}, \text{ где:}
\\
&
\begin{array}{lll}
\hat{\sigma}_{x}\left|+\right>=\left|-\right>
&
\hat{\sigma}_{y}\left|+\right>=i\left|-\right>
&
\hat{\sigma}_{z}\left|+\right>=\left|+\right>
\\
\hat{\sigma}_{x}\left|-\right>=\left|+\right>
&
\hat{\sigma}_{y}\left|-\right>=-i\left|+\right>
&
\hat{\sigma}_{z}\left|-\right>=-\left|-\right>
\end{array}

\end{flalign}
Применим к нашему состоянию оператор проекции спина на направление $\vec{n}$:
\begin{flalign}
\begin{split}
(\hat{\sigma}\vec{n})\left|\Psi\right>
=&
\left(
n_{x}\hat{\sigma}_{x}+n_{y}\hat{\sigma}_{y}+n_{z}\hat{\sigma}_z
\right)\frac{\left|+\right>+\left|-\right>}{\sqrt{2}}
=\\
=&
\frac{1}{\sqrt{6}}
\left(
\left|-\right>+\left|+\right>
\right)
+
\frac{i}{\sqrt{6}}
\left(
\left|-\right>-\left|+\right>
\right)
+
\frac{1}{\sqrt{6}}
\left(
\left|+\right>-\left|-\right>
\right)
=\\
=&
\left(
\sqrt{\frac{2}{3}}-i\frac{1}{\sqrt{6}}
\right)
\left|+\right>
\left(
\sqrt{\frac{2}{3}}+i\frac{1}{\sqrt{6}}
\right)
\left|-\right>
=\\
=&
a\left|+\right>+b\left|-\right>
\end{split}
\end{flalign}
т.к. $|a|^2=|b|^2$, то вероятность того, что после измерения электрон перейдет в состояние $\left|+\right>$ равна $0.5$.

\end{document}