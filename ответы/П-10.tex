\documentclass[__main__.tex]{subfiles}

\begin{document}

\qtitle{П}{10}
Воспользуйтесь связью полей $\vec{E}$ и $\vec{B}$ с 4-векторным потенциалом $A^\mu$, чтобы отождествить $(0,j)$-ю компоненту симметризованного тензора энергии-импульса электромагнитного поля с $j$-й компонентой вектора Пойнтинга: $\Theta^{0j}=\left(\vec{E}\times\vec{B}\right)^{j}$. Запишите и прокомментируйте теорему Пойнтинга для свободного электромагнитного поля.\\

Воспользуемся выведенным на лекции представлением тензора энергии-импульса ЭМП:
\begin{gather*}
    \Theta^{\mu\nu} = F\indices{^\mu_\rho}F^{\nu\rho}-\frac{1}{4}\eta^{\mu\nu}F_{\alpha\beta}F^{\alpha\beta},\\
    \eta^{\mu\nu}=\operatorname{diag}\left(-1,1,1,1\right)
\end{gather*}
Дальше будет приведено изображение тензора Максвелла с разными индексами. Те элементы, которые должны стоять в решетке совпадают с элементами тензора с нижними индексами. Греческие индексы пробегают значения $0{..}3$.
\begin{gather}
    F_{\mu\nu}
    =
    \left[
        \begin{matrix}
            0   & -E_x & -E_y & -E_z \\
            E_x & 0    & B_z  & -B_y \\
            E_y & -B_z & 0    & B_x  \\
            E_z & B_y  & -B_x & 0    \\
        \end{matrix}
        \right],
    \quad
    F\indices{^\mu_\nu}
    =
    \left[
        \begin{matrix}
            0   & E_x    & E_y    & E_z    \\
            E_x & \ddots & \vdots & \vdots \\
            E_y & \cdots & \ddots & \vdots \\
            E_z & \cdots & \cdots & \ddots \\
        \end{matrix}
        \right],
    \quad
    F^{\mu\nu}
    =
    \left[
        \begin{matrix}
            0    & E_x    & E_y    & E_z    \\
            -E_x & \ddots & \vdots & \vdots \\
            -E_y & \cdots & \ddots & \vdots \\
            -E_z & \cdots & \cdots & \ddots \\
        \end{matrix}
        \right],
\end{gather}

Нам нужно получить $(0;j)$ компоненту, где $j=1,2,3$. Легко заметить, что матрица $\eta$ имеет нули в этих местах, поэтому работаем только с уменьшаемым:
\begin{gather*}
    \begin{cases}
        \Theta^{01} = F\indices{^0_i}F^{1i} = E_yB_z-E_zB_y  \\
        \Theta^{02} = F\indices{^0_i}F^{2i} = -E_xB_z+E_zB_x \\
        \Theta^{03} = F\indices{^0_i}F^{3i} = E_xB_y-E_yB_x  \\
    \end{cases}
    \Rightarrow
    \Theta^{0j}=\left(\vec{E}\times\vec{B}\right)^{j}
\end{gather*}

\begin{theorem}
    $\frac{\partial U}{\partial t}+\nabla\vec{S} = -\left<\vec{j},\vec{E}\right>,$ где $U=\frac{\left<\vec{E},\vec{D}\right>+\left<\vec{B},\vec{H}\right>}{2}$ -- плотность энергии ЭМП, $(0;0)$ компонента тензора энергии-импульса, $\vec{S}=\vec{E}\times\vec{B}$ -- вектор Пойнтинга, $\vec{j}=\sigma\vec{E}$ -- плотность тока, $\vec{D}$ -- электрическая индукция ($\vec{D}\sim\vec{E}$), $\vec{H}$ -- напряженность МП ($\vec{H}\sim\vec{B}$).
\end{theorem}

Смысл этой теоремы в следующем: скорость возрастания электромагнитной энергии внутри некоторого объема в сумме с энергией, вытекающей за единицу времени через поверхность, ограничивающую этот объем, равна с минусом полной работе, совершаемой полем над источниками внутри данного объема.

\end{document}