\documentclass[12pt]{article}

\usepackage[a4paper,
            left=2cm,
            right=1cm,
            top=2cm,
            bottom=2cm]
            {geometry}

\usepackage{fontspec}
\usepackage{polyglossia}
\usepackage{amsmath,
            amssymb,
            unicode-math,
            tensor,
            hyperref,
            multicol,
}

\setdefaultlanguage{russian}
\PolyglossiaSetup{russian}{indentfirst=true}
\setotherlanguage{english}
\setkeys{russian}{babelshorthands=true}

\defaultfontfeatures{Ligatures=TeX}
\setmainfont{STIX Two Text}
\setmathfont{STIX Two Math}
\DeclareSymbolFont{letters}{\encodingdefault}{\rmdefault}{m}{it}

\newfontfamily{\cyrillicfont}{STIX Two Text} 
\newfontfamily{\cyrillicfontrm}{STIX Two Text}
\newfontfamily{\cyrillicfonttt}{Courier New}
\newfontfamily{\cyrillicfontsf}{STIX Two Text}

\everymath{\displaystyle}

\def\twodigits#1{% 
\ifnum#1<10 0\fi 
\number#1}

\newcommand{\qtitle}[2]{
  \addcontentsline{toc}{subsection}{#1-\twodigits{#2}}%
  \paragraph*{#1-\twodigits{#2}}
}

\newcommand{\ssection}[1]{%
  \addcontentsline{toc}{section}{#1}%
  \section*{#1}%
}


\begin{document}

%
\ssection{П Теория поля}
%

\qtitle{П}{1}
4-векторный потенциал и тензор Максвелла. Связь полей $\vec{E}$ и $\vec{B}$ с 4-векторным потенциалом $A^\mu$. Действие для электромагнитного поля. Преобразования полей при переходе из одной инерциальной системы отсчёта в другую. Релятивистские инварианты электромагнитного поля.

\qtitle{П}{2}
Полевая версия теоремы Нётер. Сохраняющийся нётеровский ток. Канонический и симметризованный тензоры энергии импульса электро-магнитного поля. Физический смысл компонент симметризованного тензора энергии-импульса электромагнитного поля. Вектор Пойнтинга и теорема Пойнтинга.

\qtitle{П}{3}
Покажите, что канонический тензор энергии-импульса электромагнитного поля $T^{\mu\nu}=F^{\mu\lambda}\partial^{\nu}A_{\lambda}-\frac{1}{4}F_{\alpha\beta}F^{\alpha\beta}\eta^{\mu\nu}$ не является калибровочно инвариантным в отличие от симметризованного тензора энергии-импульса электромагнитного поля $\Theta^{\mu\nu}=T^{\mu\nu}-F^{\mu\lambda}\partial_{\lambda}A^{\nu}$.

\qtitle{П}{4}
Действие для заряженной частицы в электромагнитном поле. 4-сила Лоренца. Релятивистски инвариантное действие для системы <<электромагнитное поле + заряженные частицы>>.

\qtitle{П}{5}
Воспользуйтесь полевыми уравнениями Эйлера-Лагранжа чтобы получить уравнения Максвелла <<с источниками>>.

\qtitle{П}{6}
Воспользуйтесь преобразованием компонент тензора Максвелла, чтобы получить выражение для магнитного поля равномерно медленно движущегося заряда, с помощью которого покажите справедливость соотношения Био-Савара-Лапласа, укажите границы его применимости.

\qtitle{П}{7}
Постройте канонический тензор энергии-импульса электромагнитного поля $T\indices{^\alpha_\beta}$. Убедитесь в справедливости соотношения $\partial_\alpha T\indices{^\alpha_\beta}=0$, раскройте его физическое содержание.

\qtitle{П}{8}
Покажите, что компоненты тензора энергии-импульса системы заряженных частиц $T\indices{^\mu_\nu}$ в электромагнитном поле удовлетворяет соотношению $\partial^{\nu}T\indices{^\mu_\nu}=F\indices{^\mu_\nu}j^{\nu}$, где $F\indices{^\mu_\nu}$ компоненты тензора Максвелла, $j^\nu$ компоненты 4-вектора плотности тока.

\qtitle{П}{9}
Полевые уравнения Эйлера-Лагранжа. Уравнения Максвелла. Полевая версия теоремы Нётер. Канонический тензор энергии-импульса электромагнитного поля.

\qtitle{П}{10}
Воспользуйтесь связью полей $\vec{E}$ и $\vec{B}$ с 4-векторным потенциалом $A^\mu$, чтобы отождествить $(0,j)$-ю компоненту симметризованного тензора энергии-импульса электромагнитного поля с $j$-й компонентой вектора Пойнтинга: $\Theta^{0j}=\left(\vec{E}\times\vec{B}\right)^{j}$. Запишите и прокомментируйте теорему Пойнтинга для свободного электромагнитного поля.

\qtitle{П}{11}
Симметризованный тензор энергии-импульса электромагнитного поля, вектор Пойнтинга, интенсивность излучения.

\qtitle{П}{12}
Убедитесь в инвариантности тензора Максвелла относительно калибровочных преобразований. Воспользуйтесь уравнениями Максвелла <<с источниками>>, чтобы получить закон сохранения электрического заряда.

\qtitle{П}{13}
Убедитесь в справедливости соотношения $\partial_{\alpha}T^{\alpha\beta}+\Theta^{\alpha\beta}=0$, где $T^{\alpha\beta}$ тензор энергии-импульса системы заряженных частиц и симметризованный тензор энергии-импульса электромагнитного поля. При этом можно воспользоваться $\partial_{\beta}T^{\alpha\beta}=F\indices{^\alpha_\beta}j^\beta$, где $F$ -- тензор Максвелла и $j$ -- 4-вектор плотности тока. Прокомментируйте результат.

\qtitle{П}{14}
Запишите действие для заряженной массивной частицы в электромагнитном поле и, применив принципом экстремального действия, получите её уравнения движения.

\qtitle{П}{15}
Воспользуйтесь связью полей $\vec{E}$ и $\vec{B}$ с 4-векторным потенциалом, чтобы убедиться, что тензорная форма уравнений Максвелла <<без источников>> $\partial_{\alpha}F_{\beta\gamma}+\partial_{\gamma}F_{\alpha\beta}+\partial_{\beta}F_{\gamma\alpha}=0$ эквивалентна уравнениям $\nabla\times\vec{E}=-\partial_{t}\vec{B}$ и $\nabla\cdot\vec{B}=0$. Запишите два последних уравнения в интегральной форме.

\qtitle{П}{16}
Воспользуйтесь связью полей $\vec{E}$ и $\vec{B}$ с 4-векторным потенциалом, чтобы убедиться, что тензорная форма уравнений Максвелла <<с источниками>> (в вакууме) $\partial^{\beta}F_{\alpha\gamma}=-j_{\gamma}$ эквивалентна уравнениям $\nabla\cdot\vec{E}=\rho$ и $\nabla\times\vec{B}=\partial_{t}\vec{E}+\vec{j}$. Запишите два последних уравнения в интегральной форме. Как следует изменить уравнения, чтобы они стали справедливы для материальных сред?

\qtitle{П}{17}
Тензор энергии-импульса системы частиц имеет вид $T^{\alpha\beta}=\sum_{A}p\indices{^\alpha_A}\frac{dx\indices{^\beta_A}}{dt}\delta(\vec{x}-\vec{x}_A)t$, где $A$ индекс частицы, $\delta$ -- дельта-функция Дирака. Убедитесь в справедливости соотношения $\partial_{\beta}T^{\alpha\beta}=\sum_{A}\frac{d\vec{p}_A}{dt}\delta(\vec{x}-\vec{x}_A)$. Интерпретируйте результат. Покажите, что если заряженные частицы находятся в электромагнитном поле, то последнее выражение принимает форму $\partial_{\beta}T\indices{^\alpha^\beta}=F\indices{^\alpha_\beta}j^\beta$, где $j$ -- 4-вектор плотности тока.

\clearpage

%
%
\ssection{О Оптика}
%
%

\qtitle{О}{1}
Принцип Гюйгенса-Френеля. Метод зон Френеля. Классификация дифракционных явлений.

\qtitle{О}{2}
Плоская монохроматическая световая волна с интенсивностью $I_0$ падает нормально на непрозрачный экран с круглым отверстием. Чему равна интенсивность света за экраном в точке, для которой отверстие сделали равным первой зоне Френеля, после чего закрыли по диаметру его половину?

\qtitle{О}{3}
Рассмотрите электромагнитные волны в пространстве, свободном от зарядов и токов. Покажите, что вектора $\vec{k}$ (волновой вектор), $\vec{E}$, $\vec{B}$ образуют правую тройку в некоторой инерциальной системе отсчёта. Объясните, почему в любой другой инерциальной системе отсчёта этот факт, будучи сформулированным для преобразованных векторов, будет также иметь место.

\qtitle{О}{4}
Найти угловое распределение дифракционных минимумов при дифракции на решетке, период которой равен $d$, а ширина щели равна $b$.

\qtitle{О}{5}
Электромагнитные волны в пространстве, свободном от зарядов и токов. Интерференция света: определение, общая схема и условия наблюдения интерференции. Интерференция в тонких пленках.

\qtitle{О}{6}
Степень поляризованности частично поляризованного света составляет $P$. Найдите отношение интенсивности поляризованной составляющей этого света к интенсивности естественной составляющей.

\qtitle{О}{7}
Принцип Гюйгенса-Френеля. Дифракционная решётка, её характеристики.

\qtitle{О}{8}
Принцип Гюйгенса-Френеля. Дифракционный интеграл. Пятно Пуассона-Араго.

\qtitle{О}{9}
Временная и пространственная когерентность электромагнитных волн. Длина и радиус когерентности. Связь временной когерентности со степенью монохроматичности.

\qtitle{О}{10}
Классификация дифракционных явлений. Дифракция Фраунгофера на длинной прямоугольной щели. Влияние ширины щели на дифракционную картину.

\qtitle{О}{11}
Между точечным источником света $S$ и точкой наблюдения $P$ расположили круглую диафрагму, центр которой совпадает с осью $SP$. Покажите, что расстояние от оси, соединяющей точечный источник и точку наблюдения, до границы $n$-й зоны Френеля даётся соотношением $r_n=\sqrt{n\lambda\frac{ab}{a+b}}$, где $\lambda$ -- длина волны, $a$ -- расстояние от источника до диафрагмы, $b$ -- расстояние от диафрагмы до точки наблюдения.

\qtitle{О}{12}
Плоская монохроматическая световая волна падает нормально на диафрагму с двумя узкими щелями, расстояние между которыми составляет $a$. На экране, находящемся на расстоянии $L$ от диафрагмы наблюдается интерференционная картина. На какое расстояние $x$ и в каком направлении произойдет смещение интерференционной картины, если одну щель покрыть стеклом толщины $d$ с показателем преломления $n$. Сформулируйте условия наблюдения интерференции.

\qtitle{О}{13}
Поле диполя Герца, волновая зона, интенсивность излучения, диаграмма направленности.

\qtitle{О}{14}
Точечный источник монохроматического света помещен на расстоянии $a$ от круглой диафрагмы, а экран с противоположной стороны -- на расстоянии $b$ от нее. При каких радиусах диафрагмы $r$ центр дифракционных колец, наблюдаемых на экране, будет темным и при каких -- светлым, если перпендикуляр, опущенный из источника на плоскость диафрагмы, проходит через ее центр?

\qtitle{О}{15}
На поляризатор падает неполяризованное излучение. Известна доля $\alpha$, которую составляет интенсивность прошедшего излучения по отношению к падающему. Если за первым поляризатором поместить второй такой же поляризатор, то доля интенсивности пропущенного через оба поляризатора излучения по отношению к интенсивности излучения, падающего на первый поляризатор, составит $\alpha'$. Найдите угол между плоскостями пропускания поляризаторов.

\qtitle{О}{16}
Цуг и монохроматическая волна. Длина волны, волновой вектор, волновое число. Волновая поверхность, фронт волны, фазовая скорость. Временная и пространственная когерентность электромагнитных волн.

\qtitle{О}{17}
Когерентные источники света, оптическая разность хода лучей, общая схема наблюдения интерференции, условия минимумов и максимумов интенсивности. Условия наблюдения интерференции.

\qtitle{О}{18}
Электромагнитные волны в вакууме. Соотношение между $\vec{E}$ и $\vec{B}$ в бегущей электромагнитной волне, связь интенсивности с объёмной плотностью энергии.

\clearpage

%
%
\ssection{С Электростатика}
%
%

\qtitle{С}{1}
Электрический диполь: электрический дипольный момент, потенциальная энергия диполя в электростатическом поле, момент сил, действующих на диполь в однородном электростатическом поле.

\qtitle{С}{2}
Определите характер зависимости от температуры электрической восприимчивости диэлектрика, состоящего из полярных молекул.

\qtitle{С}{3}
Закон Кулона, напряжённость поля, силовые линии электростатического поля, эквипотенциальные поверхности, электростатическая защита.

\qtitle{С}{4}
Покажите, что напряженность электростатического поля точечного диполя имеет вид $\vec{E} = \frac{1}{4\pi\varepsilon_0}\left( 3\frac{\left<\vec{p}_e, \vec{r}\right>}{r^5}\vec{r} - \frac{\vec{p}_e}{r^3}\right)$, где $\vec{r}$ радиус-вектор, проведённый из диполя в точку наблюдения, $\vec{p}_{e}$ -- электрический дипольный момент.

\qtitle{С}{5}
Покажите, что (а) в неоднородном электростатическом поле на диполь действует сила $\vec{F}=\left<\vec{p}_e,\nabla\right>\vec{E}$, (б) момент сил, действующий на электрический диполь в однородном электростатическом поле $\vec{E}$ равен $\vec{M}=\vec{p}_e\times\vec{E}$, где $\vec{p}_e$ -- электрический дипольный момент.

\qtitle{С}{6}
Потенциал электростатического поля, эквипотенциальные поверхности. Вычисление потенциала по известной напряжённости поля и определение конфигурации поля по заданному потенциалу.

\qtitle{С}{7}
Вычислите ёмкость сферического конденсатора, представленного двумя концентрическими обкладками, радиусы которых $R_1$ и $R_2$, диэлектрическая проницаемость вещества, заполняющего пространство между ними $\varepsilon$.

\qtitle{С}{8}
Примените теорему Гаусса к расчету электростатических полей: найдите поле, порождаемое бесконечной равномерно заряженной (поверхностная плотность заряда $\sigma$) плоскостью, бесконечной равномерно заряженной нитью (линейная плотность заряда $\kappa$).

\qtitle{С}{9}
Оценить среднюю объёмную плотность электрических зарядов в атмосфере, если известно, что напряженность электрического поля на поверхности Земли составляет примерно $130\text{В/м}$, а на высоте $1\text{км}$ -- примерно $40\text{В/м}$

\qtitle{С}{10}
Работа в электростатическом поле, потенциальность электростатического поля. Потенциальная энергия точечного заряда $q$ в поле, создаваемом системой точечных зарядов $Q_j$.

\qtitle{С}{11}
Механизмы поляризации диэлектриков. Теорема Гаусса для вектора $\vec{D}$. Условия на границе раздела диэлектриков.

\qtitle{С}{12}
Электрический дипольный момент, поляризованность, теорема Гаусса для вектора поляризованности, вектор электрического смещения, условия на границе раздела диэлектриков.

\qtitle{С}{13}
Проводники в электростатическом поле. Теорема Ирншоу. Поле вблизи поверхности проводника. Ёмкость уединённого проводника. Конденсатор.

\qtitle{С}{14}
Энергия системы точечных зарядов. Ёмкость уединённого проводника. Энергия уединённого заряженного проводника. Ёмкость конденсатора. Энергия конденсатора.

\qtitle{С}{15}
Внутри диэлектрика известны его поляризованность $\vec{P}(x,y,z)=2\alpha x\vec{e}_1+4y\vec{e}_2+6z\vec{e}_3$ и напряжённость поля $\vec{E}(x,y,z)=\frac{\alpha}{\varepsilon_0}x\vec{e}_1+2y\vec{e}_2+3z\vec{e}_3$, где $\alpha=const$. Найдите плотность связанных и сторонних зарядов внутри диэлектрика, а также диэлектрическую проницаемость вещества.

\clearpage

%
\ssection{Э Электричество и магнетизм}
%

\qtitle{Э}{1}
Укажите границы области применимости закона Ома и выполните переход от его дифференциальной формы к интегральной форме.

\qtitle{Э}{2}
Гипотеза молекулярных токов Ампера. Теорема о циркуляции вектора $\vec{H}$. Условия на границе раздела магнетиков.

\qtitle{Э}{3}
Сила Ампера. Магнитное поле прямого постоянного тока. Сила взаимодействия двух коллинеарных постоянных токов.

\qtitle{Э}{4}
Электрический ток. Плотность тока. Закон сохранения электрического заряда. Дифференциальная и интегральная формы закона Ома, границы его области применимости.

\qtitle{Э}{5}
Система уравнений Максвелла-Лоренца. Материальные уравнения. Условия на границе раздела двух диэлектриков. Условия на границе раздела двух магнетиков.

\qtitle{Э}{6}
Переменный синусоидальный ток частоты $\omega$ течёт по обмотке прямого соленоида, радиус сечения которого $R$. Найдите отношение амплитудных значений электрической и магнитной энергий внутри соленоида.

\qtitle{Э}{7}
Индукция магнитного поля в вакууме вблизи плоской поверхности однородного изотропного магнетика (магнитная проницаемость $\mu$) равна $\vec{B}$, причём вектор $\vec{B}$ составляет угол $\alpha$ с нормалью к поверхности. Найдите индукцию магнитного поля в магнетике вблизи поверхности.

\qtitle{Э}{8}
Теорема Бора-ван Лёвена. Парамагнетики, закон Кюри. Намагниченность, магнитная восприимчивость, магнитная проницаемость. Теорема о циркуляции вектора $\vec{H}$.

\qtitle{Э}{9}
Электромагнитная индукция, закон Фарадея, правило Ленца, явление самоиндукции, индуктивность контура.

\qtitle{Э}{10}
Найдите поле $\vec{B}$, порождаемое тороидом (число витков $N$), по которому течёт постоянный ток $I$. Воспользуйтесь полученным результатом, чтобы посчитать поле $\vec{B}$, порождаемое бесконечным соленоидом (число витков на единицу длины $n$), по которому течёт ток $I$.

\qtitle{Э}{11}
Найдите магнитное поле постоянного кругового (радиус $a$) тока $I$ на его оси. Чему равен магнитный момент этого витка стоком?

\qtitle{Э}{12}
Сила Ампера. Момент сил, действующих на виток с током в однородном магнитном поле. Магнитный момент тока. Гипотеза молекулярных токов Ампера. Намагниченность.

\qtitle{Э}{13}
Воспользуйтесь законом Био-Савара-Лапласа, чтобы вычислить магнитное поле прямого постоянного тока $I$ на расстоянии $R$ от него. Проверьте результат с помощью теоремы о циркуляции $\vec{B}$.

\clearpage

%
%
\ssection{К Квантовая механика}
%
%

\qtitle{К}{1}
Тепловое излучение и люминесценция. Равновесное тепловое излучение: свойства, спектральная плотность энергии, температура.

\qtitle{К}{2}
Абсолютно чёрное тело: испускательная способность, энергетическая светимость. Закон Стефана-Больцмана.

\qtitle{К}{3}
Вычислите постоянную Стефана-Больцмана, воспользовавшись формулой Планка.

\qtitle{К}{4}
Фотоэффект. Эффект Комптона. Интерференция электрона на двух щелях.

\qtitle{К}{5}
Ультрафиолетовая катастрофа. Формула Рэлея-Джинса. Формула Планка.

\qtitle{К}{6}
Вычислите коммутаторы $\hat{x}$, $\hat{p}_x$, $\left[\hat{L}_z,\hat{L}^2\right]$. Прокомментируйте результаты.

\qtitle{К}{7}
Воспользуйтесь соотношением Вина для спектральной плотности энергии, чтобы получить закон смещения Вина.

\qtitle{К}{8}
Сформулируйте теорему Эренфеста; прокомментируйте её на примере электрона в одномерном потенциале.

\qtitle{К}{9}
Рассмотрите задачу об электроне в бесконечно глубокой потенциальной яме. Чему равна минимальная кинетическая энергия электрона? Какова вероятность обнаружить электрон в интервале $\frac{L}{6}<x<\frac{L}{3}$ (где $L$ ширина ямы) во втором возбуждённом состоянии?

\qtitle{К}{10}
Коммутационные соотношения и спектр операторов $\hat{J}^2$ и $\hat{J}_z$ ( $\hat{J}$ -- оператор полного момента импульса).

\qtitle{К}{11}
Рассмотрите задачу об электроне в трёхмерной бесконечно глубокой потенциальной яме. Какова минимальная кинетическая энергия электрона? Чему равна кратность вырождения энергетического уровня $27\frac{\pi^2\hbar^2}{2mL^2}+U_0$ (где $L$ ширина ямы)?

\qtitle{К}{12}
Исходя из обобщённого соотношения неопределённостей Хайзенберга, получите соотношение неопределённостей <<время-энергия>>, приведите примеры использования, дайте интерпретацию результата.

\qtitle{К}{13}
Квантовомеханический гармонический осциллятор: представление чисел заполнения, энергетический спектр.

\qtitle{К}{14}
Найдите волновые функции основного и первого возбуждённого состояний квантовомеханического гармонического осциллятора.

\qtitle{К}{15}
Рассмотрите задачу об одномерном потенциальном барьере бесконечной ширины для случая, когда энергия микрообъекта превышает высоту потенциального порога, найдите коэффициенты отражения и прохождения.

\qtitle{К}{16}
Временная эволюция классической величины и временная эволюция квантовомеханического среднего. Интеграл движения в классической и квантовой механике.

\qtitle{К}{17}
Получите матричные элементы оператора импульса в координатном представлении $\left<x'|\hat{p}|x\right>$ и оператора координаты в импульсном представлении $\left<p'|\hat{x}|p\right>$, прокомментируйте результаты.

\qtitle{К}{18}
Рассмотрите задачу о движении электрона в одномерном потенциале, представляющем собой ступеньку высоты $U$ бесконечной ширины для случая, когда энергия электрона (а) $\varepsilon>U$, (б) $\varepsilon<U$. Прокомментируйте результаты.

\qtitle{К}{19}
Квантовомеханическое среднее и его временная эволюция. Классические уравнения Гамильтона и теорема Эренфеста.

\qtitle{К}{20}
Для квантовомеханического гармонического осциллятора, состояние которого задаётся кет-вектором $\left|n\right>$, вычислите $\Delta{x}\Delta{p_x}$, прокомментируйте результат.

\qtitle{К}{21}
Временное и стационарное уравнения Шрёдингера. Электрон в бесконечно глубокой одномерной потенциальной яме.

\qtitle{К}{22}
Продемонстрируйте сохранение квадрата нормы волновой функции во времени. Вектор плотности тока вероятности. Получите уравнение непрерывности для плотности вероятности.

\qtitle{К}{23}
Постулаты квантовой механики: о квантовых состояниях, о физических величинах, об измерениях, динамический постулат.

\qtitle{К}{24}
Покажите, что собственные значения оператора $\hat{J}_z$ принимают значения в интервале $-j\le m\le +j$, где $\hat{J}_z\left|jm\right>=\hbar m\left|jm\right>$, $\hat{J}^2\left|jm\right>=\hbar^2 j(j+1)\left|jm\right>$.

\qtitle{К}{25}
Атом водорода. Собственные значения и собственные функции операторов $\hat{L}^2$ и $\hat{L}_z$, кратность вырождения энергетического уровня, отвечающего заданному главному квантовому числу $n$.

\qtitle{К}{26}
Сформулируйте постулат квантовой механики о физических величинах. Покажите, что все собственные значения эрмитова оператора суть вещественные числа. Сформулируйте постулат квантовой механики об измерениях. Покажите, что собственные функции эрмитова оператора, отвечающие различным его собственным значениям, ортогональны.

\qtitle{К}{27}
Энергетическая светимость, испускательная способность, поглощательная способность. Закон Кирхгофа.

\qtitle{К}{28}
Покажите, что операторы $\hat{p}_x$ и $\hat{L}_z=\hat{x}\hat{p}_y-\hat{y}\hat{p}_x$ эрмитовы. Вычислите коммутаторы $\left[\hat{x},\hat{p}_x\right]$ и $\left[\hat{L}_z,\hat{L}^2\right]$. Покажите, что собственные значения квадрата эрмитова оператора неотрицательны. Прокомментируйте результаты.

\qtitle{К}{29}
Электрон находится в одномерной прямоугольной потенциальной яме с бесконечно высокими стенками. Определите ширину ямы, если известно, что разность энергии между первым и вторым возбуждёнными состояниями составляет $0.3\text{эВ}$.

\qtitle{К}{30}
<<Старая>> квантовая теория: постулаты Бора и комбинационное правило Ритберга-Ритца. Спектральные серии атома водорода.

\qtitle{К}{31}
Электрон находится в состоянии $\Psi(t,x)$, описываемом суперпозицией двух стационарных состояний, $\Psi_m(t,x)$ и $\Psi_n(t,x)$ с весами $c_m$ и $c_n$ соответственно ($m\neq n$, энергии $m$ и $n$ известны). Найдите квантовомеханическое среднее гамильтониана. Какова вероятность в результате измерения энергии получить значение $\varepsilon_m$?

\qtitle{К}{32}
Частица массы $m$ находится в трёхмерной кубической потенциальной яме (ребро куба $l$) с абсолютно непроницаемыми стенками. Найти разность энергий второго и третьего возбуждённого уровней. Указать их кратность вырождения.

\qtitle{К}{33}
Получите с помощью формулы Планка приближённые выражения для объёмной спектральной плотности излучения в области (а) $\hbar\omega\ll k_{B}T$, (б) $\hbar\omega\ll k_{B}T$. Преобразуйте формулу Планка для объемной спектральной плотности излучения $u_{\omega}(T)$ от переменной $\omega$ к переменной $\lambda$.

\qtitle{К}{34}
Временное и стационарное уравнения Шрёдингера. Задача об электроне в трёхмерной потенциальной яме с абсолютно непроницаемыми стенками.

\qtitle{К}{35}
Получите формулу Планка. Сопоставьте вычисленную с её помощью объёмную спектральную плотность энергии равновесного теплового излучения с предсказанием Рэлея-Джинса. Прокомментируйте результат.

\qtitle{К}{36}
Укажите условие, при соблюдении которого возможно произвести разделение временной и пространственных переменных в нерелятивистском временном уравнении Шрёдингера. Реализуйте процедуру разделения переменных и получите временную зависимость для стационарных состояний.

\qtitle{К}{37}
Рассмотрите стационарное уравнение Шрёдингера для электрона в сферически симметричном поле. Реализуйте процедуру разделения переменных. Охарактеризуйте функции $Y_{im}(\theta,\phi)$ содержащие зависимость от угловых переменных.

\qtitle{К}{38}
Квантовомеханический гармонический осциллятор: представление чисел заполнения, энергетический спектр, волновые функции основного и первого возбуждённого состояний.

\qtitle{К}{39}
Рассмотрите стационарное уравнение Шрёдингера для электрона в сферически симметричном поле. Реализуйте процедуру разделения переменных. Охарактеризуйте свойства радиальной составляющей волновой функции $\rho_{nl}(r)$.

\qtitle{К}{40}
Покажите, что два наблюдаемых оператора коммутируют тогда и только тогда, когда имеют общую систему собственных функций. Приведите примеры совместных и несовместных наблюдаемых величин.

\clearpage

%
%
\ssection{Ч Элементарные частицы}
%
%

\qtitle{Ч}{1}
Докажите, что собственными значениями эрмитова оператора, квадрат которого равен тождественному оператору, могут быть только $\pm 1$. Воспользуйтесь доказанным утверждением, чтобы ответить на вопрос о возможных результатах измерения проекции спина электрона на выделенное направление?

\qtitle{Ч}{2}
Сформулируйте принцип неразличимости частиц одного сорта. Продемонстрируйте работу постулата симметризации на системе из двух тождественных (а) бозонов, (б) фермионов.

\qtitle{Ч}{3}
Докажите, что собственными значениями эрмитова оператора, квадрат которого равен тождественному оператору, могут быть только $\pm 1$. Покажите, что оператор перестановки частиц удовлетворяет вышеперечисленным требованиям.

\qtitle{Ч}{4}
Гильбертово пространство состояний частицы заданного сорта, полная абелева совокупность, принцип дополнительности Бора, получите и прокомментируйте обобщенное соотношение неопределённостей Хайзенберга величин $A$ и $B$: $\Delta{A}\Delta{B}\ge\frac{\left|\left<\left[\hat{A},\hat{B}\right]\right>\right|}{2}$.

\qtitle{Ч}{5}
Гильбертово пространство состояний системы из двух частиц (а) разных сортов, (б) одного сорта. Принцип тождественности частиц. Постулат симметризации, принцип запрета Паули.

\qtitle{Ч}{6}
Покажите, что для системы из двух электронов, находящейся в факторизованном состоянии существует пара направлений (по одному для каждого электрона), в которых проекция спинов соответствующих электронов с определённостью положительна.

\qtitle{Ч}{7}
Факторизованные и запутанные состояния. Проекторная техника. Неравенства Белла.

\qtitle{Ч}{8}
Принципы работы лазера. Коэффициенты Эйнштейна и их взаимосвязь, распространение излучения в среде с резонансными уровнями, инверсия заселённостей, активная среда, роль резонатора.

\qtitle{Ч}{9}
Покажите, что для системы из двух электронов, находящейся в запутанном состоянии $\left|Est\right>=\frac{\left|+-\right>-\left|-+\right>}{\sqrt{2}}$ не существует пары направлений (по одному для каждого электрона), в которых проекция спинов соответствующих электронов с определённостью положительна.

\qtitle{Ч}{10}
Эксперимент Штерна-Герлаха. Гипотеза спина электрона. Оператор проекции спина на выделенное направление. Эксперименты с поляризованным пучком электронов.

\qtitle{Ч}{11}
Электрон в сферически симметричном потенциале. Главное, орбитальное, магнитное квантовые числа. Опыт Штерна-Герлаха. Гипотеза спина электрона.

\qtitle{Ч}{12}
Квантовые статистики: функция распределения Бозе-Эйнштейна, функция распределения Ферми-Дирака. Классический предел квантовых статистик.

\qtitle{Ч}{13}
Электрон находится в состоянии, описываемом кет-вектором $\frac{\left|+\right>+\left|-\right>}{\sqrt{2}}$. Найдите вероятность того, что проекция его спина на направление $\vec{n}$, составляющее угол $\frac{\pi}{4}$ с осями $Ox$, $Oy$, $Oz$, окажется равной $+\frac{\hbar}{2}$.


\qtitle{Ч}{15}
Магнитный момент атома, связанный с орбитальным моментом импульса электрона: квазиклассическое и квантовомеханическое рассмотрения.

\qtitle{Ч}{16}
Установка состоит из двух приборов Штерна-Герлаха. Неполяризованный пучок электронов направляется на прибор Штерна-Герлаха, ориентированный вдоль оси $Oz$, после прохождения которого выделяют электроны, состояние которых описывается кет-вектором $\left|+\right>$, и направляют их на следующий прибор Штерна-Герлаха, ориентация ко-
торого составляет $\frac{\pi}{4}$ с осью $Oz$. Предскажите, что будет наблюдаться
на выходе из установки (т.е. какая доля электронов исходного пучка окажется в каком состоянии).

\qtitle{Ч}{17}
Найти при $T=0$ (а) максимальную кинетическую энергию свободных электронов в металле, если известна их концентрация $\nu$, (б) определить отношение средней кинетической энергии к максимальной.

\qtitle{Ч}{18}
Электрон в сферически симметричном потенциале. Главное, орбитальное, магнитное квантовые числа. Опыт Штерна-Герлаха. Гипотеза спина электрона.

\qtitle{Ч}{19}
Постулат симметризации для частиц с полуцелым спином. Определитель Фока-Слэтера. Принцип запрета Паули. Функция распределения Ферми-Дирака. Газ невзаимодействующих электронов в основном состоянии, энергия Ферми.

\qtitle{Ч}{20}
Электрон находится в состоянии, описываемом кет-вектором $\left|st\right>=\frac{\left|+\right>-\left|-\right>}{\sqrt{2}}$. Найдите направление, при измерении проекции спина на которое результат с определённостью будет оказываться равным $+\frac{\hbar}{2}$.

\qtitle{Ч}{21}
Атомная цепочка. Электрон в периодическом потенциале. Теорема Блоха. Зона Бриллюэна. Циклические граничные условия (Борна-Кармана). Зонная структура кристаллов.

\clearpage

%
%
\ssection{Таблица соответствия}
%
%

\begin{multicols}{3}

\begin{tabular}{c|rrrr}
\textbf{1}  & {С-01} & {К-01} & {С-02} & {Ч-01} \\
\textbf{2}  & {С-03} & {К-02} & {С-04} & {К-03} \\
\textbf{3}  & {П-01} & {К-04} & {С-05} & {К-40} \\
\textbf{4}  & {П-02} & {К-05} & {П-03} & {К-06} \\
\textbf{5}  & {П-04} & {Ч-02} & {П-05} & {Ч-03} \\
\textbf{6}  & {О-01} & {Ч-04} & {О-02} & {К-07} \\
\textbf{7}  & {С-06} & {К-08} & {О-03} & {К-09} \\
\textbf{8}  & {П-06} & {К-10} & {О-04} & {К-11} \\
\textbf{9}  & {О-05} & {Ч-05} & {Э-01} & {Ч-06} \\
\textbf{10} & {Э-02} & {Ч-07} & {С-07} & {К-12} \\
\textbf{11} & {Э-03} & {Ч-08} & {С-08} & {Ч-09} \\
\textbf{12} & {Э-04} & {К-13} & {С-09} & {К-14} \\
\end{tabular}

\begin{tabular}{c|rrrr}
\textbf{13} & {Э-05} & {Ч-10} & {Э-06} & {К-15} \\
\textbf{14} & {С-10} & {К-16} & {О-06} & {К-17} \\
\textbf{15} & {О-07} & {Ч-11} & {Э-07} & {К-18} \\
\textbf{16} & {О-08} & {К-19} & {П-06} & {К-20} \\
\textbf{17} & {С-11} & {К-21} & {П-07} & {К-22} \\
\textbf{18} & {О-09} & {К-23} & {П-08} & {К-24} \\
\textbf{19} & {П-09} & {К-25} & {П-10} & {К-26} \\
\textbf{20} & {О-10} & {К-27} & {О-11} & {К-28} \\
\textbf{21} & {Э-08} & {Ч-12} & {О-12} & {Ч-13} \\
\textbf{22} & {О-13} & {Ч-07} & {О-14} & {К-29} \\
\textbf{23} & {П-11} & {Ч-15} & {П-12} & {Ч-16} \\
\textbf{24} & {С-12} & {К-30} & {П-13} & {Ч-17} \\
\end{tabular}

\begin{tabular}{c|rrrr}
\textbf{25} & {Э-09} & {Ч-18} & {О-15} & {К-31} \\
\textbf{26} & {О-16} & {Ч-19} & {П-14} & {К-32} \\
\textbf{27} & {О-17} & {Ч-08} & {Э-10} & {К-33} \\
\textbf{28} & {О-18} & {Ч-07} & {Э-11} & {Ч-20} \\
\textbf{29} & {С-13} & {К-34} & {П-15} & {К-35} \\
\textbf{30} & {С-14} & {Ч-12} & {П-16} & {К-12} \\
\textbf{31} & {Э-05} & {Ч-21} & {П-17} & {К-36} \\
\textbf{32} & {Э-12} & {К-23} & {Э-13} & {К-37} \\
\textbf{33} & {П-09} & {К-38} & {П-17} & {К-39} \\
\textbf{34} & {О-13} & {Ч-05} & {С-15} & {К-24} \\
\end{tabular}

\end{multicols}

\clearpage

%
%
\setcounter{tocdepth}{1}
\tableofcontents
%
%

\end{document}