\documentclass[__main__.tex]{subfiles}

\begin{document}

\qtitle{П}{06}
Воспользуйтесь преобразованием компонент тензора Максвелла, чтобы получить выражение для магнитного поля равномерно медленно движущегося заряда, с помощью которого покажите справедливость соотношения Био-Савара-Лапласа, укажите границы его применимости.\\ 

Рассмотрим заряд $q$, помещенный в начало координат ИСО $K_1$, движущейся относительно ИСО $K$ вдоль оси $Ox$ с быстротой $\nu=const$, тогда координаты заряда в $K$ примут вид:
\begin{gather*}
x^1 = c\nu t, \quad x^2 = 0, \quad x^3 = 0,
\end{gather*}
где $\nu = const$ -- быстрота $K_1$ относительно $K$.
Из закона Кулона в ИСО $K_1$ потенциал поля, порождаемого зарядом $q$:
$$
A'_{\alpha} = (\varphi_1, \vec{A}_1) = (\frac{q}{x_1},\vec{0}),
$$
где $x_1=\sqrt{\left(x^1_1\right)^2+\left(x^2_1\right)^2+\left(x^3_1\right)^2}$.
Тогда найдем потенциал поля частицы в ИСО $K$ из преобразований Лоренца:
\begin{gather}
\label{e-05-potlorenz}
\left(
\begin{matrix}
\varphi\\ A^1\\ A^2\\ A^3
\end{matrix}
\right) = 
\left(
\begin{matrix}
\gamma    & \gamma\nu & 0 & 0 \\
\gamma\nu &    \gamma & 0 & 0 \\
0 &         0 & 1 & 0 \\
0 &         0 & 0 & 1
\end{matrix}
\right)
\left(
\begin{matrix}
\varphi_1 \\ 0\\ 0\\ 0
\end{matrix}
\right) \qquad \Rightarrow \qquad
\left(
\begin{matrix}
\varphi\\ A^1\\ A^2\\ A^3
\end{matrix}
\right) =
\left(
\begin{matrix}
\gamma \varphi_1\\ \gamma \nu \varphi_1\\ 0\\ 0
\end{matrix}
\right),
\end{gather}
теперь найдем координаты $x_1$ в $K$:
$$
\left(
\begin{matrix}
ct_1 \\ x^1_1 \\x^2_1 \\ x^3_1
\end{matrix}
\right) = 
\left(
\begin{matrix}
\gamma & -\gamma\nu & 0 & 0 \\
-\gamma\nu &     \gamma & 0 & 0 \\
0 &          0 & 1 & 0 \\
0 &          0 & 0 & 1
\end{matrix}
\right)
\left(
\begin{matrix}
ct \\ x^1 \\ x^2 \\ x^3
\end{matrix}
\right) \qquad \Rightarrow \qquad
\left(
\begin{matrix}
ct_1 \\ x^1_1 \\x^2_1 \\ x^3_1
\end{matrix}
\right) =
\left(
\begin{matrix}
\gamma ct - \gamma\nu x^1 \\
-\gamma\nu ct + \gamma x^1 \\
x^2 \\
x^3
\end{matrix}
\right),
$$
тогда из $x_1$ примет вид:
\begin{gather}
\label{e-05-rasst}
\left(x_1\right)^2 = \left(x^1_1\right)^2+\left(x^2_1\right)^2+\left(x^3_1\right)^2 =
\gamma^2\left(x^1-\nu ct\right)^2+\left(x^2\right)^2+\left(x^3\right)^2.
\end{gather}
Находим вектор магнитной индукции $\vec{B} $, порождаемый зарядом $q$ из
(\ref{e-05-potlorenz}) и (\ref{e-05-rasst}):
\begin{flalign*}
\begin{split}
\vec{B} &=
\nabla \times \vec A =
\frac{\partial A^1}{\partial x^3}\vec{e}_2 - \frac{\partial A^1}{\partial x^2}\vec{e}_3 =
-\frac{\gamma\nu q x^3}{\left(x_1\right)^3}\vec{e}_2 + \frac{\gamma\nu q x^2}{\left(x_1\right)^3}\vec{e}_3 = \\
&= \frac{\gamma\nu q}{\left(x_1\right)^3}(-x^3\vec{e}_2+x^2\vec{e}_3) =
\frac{\gamma q}{\left(x_1\right)^3}\vec \nu \times \vec x,
\end{split}
\end{flalign*}
где 
$\displaystyle \vec \nu = \frac{1}{c}\vec v = \nu\vec{e}_1 = \frac{1}{c}v\vec{e}_1$ -- вектор быстроты ИСО $K_1$ относительно $K$.
Положим $\nu \ll c$, тогда $\gamma \approx 1$ и $\vec{B}$ примет вид:
$$
\vec{B}=
\frac{q}{c\left(\left(x^1-vt\right)^2+\left(x^2\right)^2+\left(x^3\right)^2\right)^{3/2}}
\vec v \times \vec x = 
\frac{q}{c\left| \vec x - \vec v t \right|^3}\vec v \times \vec x 
$$ 
\textbf{(Здесь идёт догадка автора)}
Продифференцируем обе части уравнения (по времени?). Знаем, что производная заряда по времени это ток, а производная координаты по времени это скорость. Тогда получим:

$$d\vec{B} = \frac{1}{c} \frac{I [\vec{v},\vec{r}]}{r^3}.$$

Полученный результат есть ни что иное как закон Био-Савара-Лапласа.

%%

\end{document}