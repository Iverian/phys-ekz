\documentclass[__main__.tex]{subfiles}

\begin{document}

\qtitle{С}{11}
Механизмы поляризации диэлектриков. Теорема Гаусса для вектора $\vec{D}$. Условия на границе раздела диэлектриков.\\ 

\begin{wrapfigure}{r}{.3\linewidth}
	\centering
	\input{img/e-03.pdf_tex}
	\caption{электрический диполь}
	\llabel{e03:fig:dipol}
\end{wrapfigure}

Рассмотрим молекулу диэлектрика как электрический диполь: систему, состоящую из двух одинаковых по модулю точечных зарядов $+q$ и $-q$, находящихся на расстоянии $l$ друг от друга (см. Рис. \lref{e03:fig:dipol}). Поместим заряды в центры <<тяжести>> положительных и отрицательных зарядов молекулы. Введем величину $\vec{p}$ -- \emph{дипольный момент молекулы}:
\begin{gather}
	\vec{p} = q\vec{l}.
\end{gather}
Различают следующие виды молекул диэлектриков:
\begin{itemize}
	\item
	\emph{неполярными} называют молекулы, у которых центры <<тяжести>> положительных и отрицательных зарядов совпадают в отсутствие внешнего электрического поля $\vec{l}~=~\vec{0}\Rightarrow\vec{p}=\vec{0}$;
	\item
	ко второй группе относят вещества с асимметричным строением молекул (центры тяжести зарядов сдвинуты относительно друг друга), в отсутствие внешнего эл. поля они обладают дипольным моментом. Но, вследствие теплового движения молекул, в отсутствие внешнего эл. поля их дипольные моменты хаотично ориентированы и, следовательно, их результирующий равен нулю;
	\item
	третья группа -- вещества с ионным строением. Ионные кристаллы -- это пространственные решетки с правильным чередованием ионов разных знаков. При внесении таких кристалов во внешнее электрическое поле, кристаллическая решетка деформируется, возникает дипольный момент.
\end{itemize}

В зависимости от характера химической связи различают 3 \textbf{механизма поляризации диэлектриков}: электронный, ионный и дипольный (ориентационный).

\begin{itemize}
\item
\textbf{Электронная поляризация} присуща всем диэлектрикам и превалирует в кристаллах с ковалентной связью. Под действием внешнего электрического поля $P$ происходит смещение электронов атома относительно его ядра (деформация его электронной оболочки) и возникают индуцированные диполи. Диэлектрические свойства индуцированных диполей относятся к числу резонансных явлений.

Электронный механизм поляризации является наименее инерционным, т.к. масса электрона значительно меньше массы частиц, участвующих в процессе поляризации. Время установления электронной поляризации составляет $\approx$ 10-15 с, что сравнимо с периодом световых колебаний.

\item
\textbf{Ионная поляризация} наблюдается в ионных кристаллах и происходит в результате возникновения диполей вследствие относительного смещения (сдвига) положительных и отрицательных ионов под влиянием электрического поля. При этом имеет место также деформация электронных оболочек ионов, что порождает электронную поляризацию. Время установления ионной поляризации примерно на порядок больше ( $\approx$ 10-14 с).

\item
\textbf{Дипольная (ориентационная) поляризация} наблюдается в полярных диэлектриках. Существующие в отсутствии электрического поля электрические диполи ориентированы хаотично. При включении поля диполи приобретают преимущественную ориентацию. Этот процесс и называют дипольный или ориентационной поляризацией.
\end{itemize}

\begin{definition}
	\emph{Поляризацией} диэлектрика называется явление ориентации его диполей при внесении во внешнее электрическое поле.
\end{definition}

\begin{definition}
	\emph{Поляризованностью} вещества $\vec{P}$ назовем дипольный момент единицы объема этого вещества: если $\Delta V$ -- бесконечно малый объем вещества, а $\vec{p}_V = \sum_{i}\vec{p}_i$ -- сумма дипольных моментов молекул, заключенных в этом объеме, то:
	\begin{gather}
		\vec{P} = \frac{1}{\Delta V}\sum_{i}\vec{p}_i,
	\end{gather}
	если же рассматривать диэлектрик как сплошное тело, то $\vec{P}=\frac{d\vec{p}}{dV}$.
\end{definition}

Для большого класса диэлектриков справедливо:
\begin{gather}
	\vec{P} = \kappa\varepsilon_0\vec{E},
\end{gather}
где $\kappa$ -- диэлектрическая восприимчивость диэлектрика ($[\kappa]=1$).

Заряды, входящие в состав молекул диэлектрика, назовем \emph{связанными}. Заряды, находящиеся в пределах диэлектрика, но не входящие в состав его молекул, а так же заряды, расположенные за пределами диэлектрика, будем называть \emph{сторонними}.

\begin{definition}
	Вектором электрической индукции (электрического смещения) $\vec{D}$ называют вектор
	\begin{gather}
		\vec{D} = \varepsilon_0\vec{E}+\vec{P},
	\end{gather}
\end{definition}

В вакууме $\vec{D}$ вычисляется по формулам:
\begin{gather}
	\vec{D} = \varepsilon_0\vec{E}_0 = \varepsilon_0\varepsilon\vec{E},\llabel{eq:deq}
\end{gather}
где $\varepsilon=\kappa+1$, $\vec{E}$ -- вектор напряженности электрического поля в диэлектрике.

\begin{theorem}[Гаусса для вектора электрической индукции]
	Поток вектора электрического смещения в диэлектрике сквозь замкнутую поверхность $S$ равен сумме заключенных внутри нее \emph{сторонних} электрических зарядов $Q^{\text{(стор)}}$.
	\begin{gather}
		\oiint\limits_{S}(\vec{D},\vec{n})d\sigma = Q^{\text{(стор)}} = \sum_{i}^{n}q_i^{\text{(стор)}}.
	\end{gather}
\end{theorem}

\begin{wrapfigure}{r}{.3\linewidth}
	\centering
	\def\svgwidth{1\linewidth}
	\input{img/e-03-1.pdf_tex}
	\caption{контур $C$}
	\llabel{e03:fig:abcd}
\end{wrapfigure}

\textbf{Условия на границе раздела диэлектриков:} рассмотрим вектор напряженности электрического поля $\vec{E}$ и вектор электрического смещения $\vec{D}$ на границе раздела двух однородных изотропных (симметрия относительно поворота в пространстве) диэлектриков с диэлектрической проницаемостью $\varepsilon_1$ и $\varepsilon_2$, при отсутствии на границе свободных зарядов.

\textbf{Резюме:} \emph{Если на границе раздела двух однородных изотропных диэлектриков сторонних зарядов нет, то при переходе этой границы $(\vec{E},\vec{\tau})$ и $(\vec{D},\vec{n})$ изменяются непрерывно, без скачка; составляющие $(\vec{E},\vec{n})$ и $(\vec{D},\vec{\tau})$ испытывают скачок.}

Построим внутри границы раздела диэлектриков \textbf{1} и \textbf{2} прямоугольный контур $C = MNPK$ длины $l$, ориентировав его как показано на Рис. \lref{e03:fig:abcd}. Условия на границе получим с помощью теоремы Гаусса и теоремы о циркуляции:
\begin{flalign}
	& \oint\limits_{C}(\vec{E},\vec{\tau})dl = 0, \llabel{eq:circ} \\
	& \oiint\limits_{S}(\vec{D},\vec{n})d\sigma = Q^{\text{(стор)}}. \llabel{eq:gauss}
\end{flalign}

Пусть напряженность поля около границы в \textbf{1} равна $\vec{E}_1$ и $\vec{E}_2$ в $\textbf{2}$. Сторона контура должна иметь такую длину, чтобы в ее пределах поле было однородным, т.е. можно было бы $\vec{E}$ считать одинаковым. Согласно (\lref{eq:circ}) $(\vec{E}_1,\vec{\tau})l + (\vec{E}_2,\vec{\tau})l = 0$. Т.к. знаки интегралов по $KM$ и $PN$ различны, а значения интегралов на $MN$ и $PK$ ничтожно малы, то $(\vec{E}_1,\vec{\tau})l-(\vec{E}_2,\vec{\tau})l=0$, следовательно, получим:
\begin{gather}
	(\vec{E}_1,\vec{\tau}) = (\vec{E}_2,\vec{\tau}),
\end{gather}
где $\vec{\tau} = \vec{j}$. Получим, что \emph{Касательная составляющая вектора $\vec{E}$ оказывается одинаковой по обе стороны границы раздела (т.е. не испытывает скачка при переходе через границу раздела).} Для $\vec{D}$ из (\lref{eq:deq}) получим из $\varepsilon_1\neq\varepsilon_2$:
\begin{gather}
	\begin{cases}
		(\vec{D}_1,\vec{\tau}) = \varepsilon_1\varepsilon_0(\vec{E}_1,\vec{\tau}) \\
		(\vec{D}_2,\vec{\tau}) = \varepsilon_2\varepsilon_0(\vec{E}_2,\vec{\tau})
	\end{cases},
\end{gather}
получим:
\begin{gather}
	\frac{(\vec{D}_1,\vec{\tau})}{(\vec{D}_2,\vec{\tau})} = \frac{\varepsilon_1}{\varepsilon_2},
\end{gather}
т.е. \emph{Тангенциальная составляющая $\vec{D}$ при переходе границы раздела диэлектриков испытывает скачок.}

%\begin{wrapfigure}{r}{.3\linewidth}
%\centering
%\def\svgwidth{1\linewidth}
%\input{img/e-03-2.pdf_tex}
%\caption{контур $C$}
%\llabel{e03:fig:cylynder}
%\end{wrapfigure}

Пусть на границе раздела двух диэлектриков имеются сторонние заряды. Замкнутую поверхность выберем в виде цилиндра малой высоты $h$, расположим его на границе раздела двух диэлектриков так, чтобы одно основание было в диэлектрике \textbf{1}, а второе – в \textbf{2}. Сечение цилиндра возьмем таким, чтобы в пределах каждого его торца вектор $\vec{D}$ был одинаков. Тогда, учитывая, что поток через боковую поверхность можно представить в виде $\left<(\vec{D},\vec{n})\right>S_{\text{бок}}$, где $S_{\text{бок}}$ -- площадь боковой поверхности, а $\left<(\vec{D},\vec{n})\right>$ -- среднее значение $(\vec{D},\vec{n})$ на ней, по (\lref{eq:gauss}) для $\vec{D}$ получим:
\begin{gather}
	\Psi_{\vec{D}} = (\vec{D}_1,\vec{n}_1)\Delta S + (\vec{D}_2,\vec{n}_2)\Delta S + \left<(\vec{D},\vec{n})\right>S_{\text{бок}} = \sigma\Delta S,
\end{gather}
при $h\rightarrow 0\colon S_{\text{бок}}\rightarrow 0$ и $h\rightarrow 0\colon\Psi_{\vec{D}}\rightarrow (\vec{D}_1,\vec{n}_1)\Delta S + (\vec{D}_2,\vec{n}_2)\Delta S = \sigma\Delta S$. Тогда $(\vec{D}_1,\vec{n}_1)+(\vec{D}_2,\vec{n}_2)=\sigma$, тогда получим $(\vec{D}_2,\vec{n})-(\vec{D}_1,\vec{n})=\sigma$, где $\vec{n}$ -- общая нормаль к $S$. Следовательно, получим при $\sigma=0$ (отсутствии внешних зарядов) $(\vec{D}_1,\vec{n})=(\vec{D}_2,\vec{n})$; согласно (\lref{eq:deq}) получим:
\begin{gather}
	\frac{(\vec{E}_1,\vec{n})}{(\vec{E}_2,\vec{n})}=\frac{\varepsilon_2}{\varepsilon_1}.
\end{gather}
%%

\end{document}