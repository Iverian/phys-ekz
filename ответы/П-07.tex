\documentclass[__main__.tex]{subfiles}

\begin{document}

\qtitle{П}{07}
Постройте канонический тензор энергии-импульса электромагнитного поля $T\indices{^\alpha_\beta}$. Убедитесь в справедливости соотношения $\partial_\alpha T\indices{^\alpha_\beta}=0$, раскройте его физическое содержание.\\ 

\emph{Канонический тензор энергии - импульса} поля имеет вид:
\begin{gather}
T\indices{^\alpha_\beta}
=
\frac{\partial\Lambda}{\partial(\partial_\alpha{q^i})}\partial_\beta{q^i}
-
\Lambda\delta\indices{^\alpha_\beta},
\end{gather}
где $q^i$ -- обобщенные координаты поля, $\Lambda(q^i,\partial_\alpha{q^i})$ -- <<плотность лагранжиана>> поля. Функция $\Lambda$ удовлетворяет \emph{полевым уравнениям Лагранжа}:
\begin{gather}
\partial_\alpha\left(\frac{\partial\Lambda}{\partial(\partial_\alpha{q^i})}\right)
-
\frac{\partial\Lambda}{\partial{q^i}}=0
\end{gather}
Теперь рассмотрим выражение $\partial_\alpha T\indices{^\alpha_\beta}$:
\begin{flalign}
\begin{split}
\partial_\alpha T\indices{^\alpha_\beta}
=&
\partial_\alpha\left(\frac{\partial\Lambda}{\partial(\partial_\alpha{q^i})}\partial_\beta{q^i}\right)
-
\partial_\alpha\Lambda\delta\indices{^\alpha_\beta}
=
\partial_\alpha\left(\frac{\partial\Lambda}{\partial(\partial_\alpha{q^i})}\partial_\beta{q^i}\right)
-
\partial_\beta\Lambda
\Longrightarrow\\
\Longrightarrow&
\partial_\beta\Lambda
=
\frac{\partial\Lambda}{\partial{q^i}}\partial_\beta{q^i}
+
\frac{\partial\Lambda}{\partial(\partial_\alpha{q^i})}\partial_{\alpha\beta}{q^i}
\Longrightarrow\\
\Longrightarrow
\partial_\alpha T\indices{^\alpha_\beta}
=&
\partial_\alpha\left(\frac{\partial\Lambda}{\partial(\partial_\alpha{q^i})}\right)
+
\frac{\partial\Lambda}{\partial(\partial_\alpha{q^i})}\partial_{\alpha\beta}{q^i}
-
\frac{\partial\Lambda}{\partial{q^i}}\partial_\beta{q^i}
-
\frac{\partial\Lambda}{\partial(\partial_\alpha{q^i})}\partial_{\alpha\beta}{q^i}
=\\
=&
\partial_\alpha\left(\frac{\partial\Lambda}{\partial(\partial_\alpha{q^i})}\right)
-
\frac{\partial\Lambda}{\partial{q^i}}
=\\
=&
0,
\end{split}
\end{flalign}
получили $\partial_{\alpha}T\indices{^\alpha_\beta}=0$, осталось лишь построить \emph{канонический тензор энергии - импульса электромагнитного поля}: для этого нужно определить вид $q^{i}$ и $\Lambda$. Оказывается что обобщенными координатами $q^{i}$ электромагнитного поля выступает 4-вектор потенциала $A^\mu$, а $\Lambda$ имеет вид:
\begin{gather}
\Lambda
=
\frac{1}{4}F^{\alpha\beta}F_{\alpha\beta},
\end{gather}
где $F_{\alpha\beta}=\partial_\alpha{A_\beta}-\partial_\beta{A_\alpha}$ -- тензор Максвелла.
\begin{flalign}
\begin{split}
T\indices{^\mu_\nu}
=&
\frac{\partial\Lambda}{\partial(\partial_\mu{A_\lambda})}\partial_\nu{A_\lambda}
-
\Lambda\delta\indices{^\mu_\nu}
=\\
=&
\frac{\partial\Lambda}{\partial(\partial_\mu{A_\lambda})}\partial_\nu{A_\lambda}
-\frac{1}{4}F^{\alpha\beta}F_{\alpha\beta}\delta\indices{^\mu_\nu}
\end{split}
\end{flalign}
Вычислим отдельно первое слагаемое:
\begin{flalign*}
\begin{split}
\frac{\partial\Lambda}{\partial(\partial_\mu{A_\lambda})}
=&
\frac{\eta^{\gamma\alpha}\eta^{\nu\beta}}{4}
\frac{\partial}{\partial(\partial_\mu{A_\lambda})}
\left[
\left(
\partial_\alpha{A_\beta}
-
\partial_\beta{A_\alpha}
\right)
\left(
\partial_\gamma{A_\nu}
-
\partial_\nu{A_\gamma}
\right)
\right]
=\\
=&
\frac{\eta^{\gamma\alpha}\eta^{\nu\beta}}{4}
\left[
\left(
\delta\indices{^\mu_\alpha}\delta\indices{^\lambda_\beta}
-
\delta\indices{^\mu_\beta}\delta\indices{^\lambda_\alpha}
\right)
\left(
\partial_\gamma{A_\nu}
-
\partial_\nu{A_\gamma}
\right)
+
\left(
\partial_\alpha{A_\beta}
-
\partial_\beta{A_\alpha}
\right)
\left(
\delta\indices{^\mu_\gamma}\delta\indices{^\lambda_\nu}
-
\delta\indices{^\mu_\nu}\delta\indices{^\lambda_\gamma}
\right)
\right]
=\\
=&
\frac{1}{4}
\left[
\left(
\eta^{\gamma\mu}\eta^{\nu\lambda}
-
\eta^{\gamma\lambda}\eta^{\nu\mu}
\right)
\left(
\partial_\gamma{A_\nu}
-
\partial_\nu{A_\gamma}
\right)
+
\left(
\partial_\alpha{A_\beta}
-
\partial_\beta{A_\alpha}
\right)
\left(
\eta^{\alpha\mu}\eta^{\beta\lambda}
-
\eta^{\alpha\lambda}\eta^{\beta\mu}
\right)
\right]
=\\
=&
\frac{1}{4}
\left[
\left(
\partial^\mu{A^\lambda}
-
\partial^\lambda{A^\mu}
-
\partial^\lambda{A^\mu}
+
\partial^\mu{A^\lambda}
\right)
+
(
\cdots
)
\right]
=\\
=&
F^{\mu\lambda}
\end{split}
\end{flalign*}

Получим выражение для \emph{канонического тензора энергии - импульса электромагнитного поля}:
\begin{gather}
\underset{emf}{T\indices{^\mu_\nu}}
=
F^{\mu\lambda}\partial_{\nu}A_{\lambda}
-
\frac{1}{4}F_{\alpha\beta}F^{\alpha\beta}\delta\indices{^\mu_\nu}
\end{gather}
\end{document}