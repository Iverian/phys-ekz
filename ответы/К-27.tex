\documentclass[__main__.tex]{subfiles}

\begin{document}

\qtitle{К}{27}
Энергетическая светимость, испускательная способность, поглощательная способность. Закон Кирхгофа.\\ 

\begin{definition}
Энергетическая светимость $R$ (интегральная плотность потока энергии излучения) — это энергия, испускаемая единицей площади поверхности тела в единицу времени.
\end{definition}
\begin{gather*}
R= \int\limits_0^\infty r_\omega d\omega
\end{gather*}

\begin{definition}
Испускательная способность тела $r_\omega$ (спектральная плотность потока энергии излучения) — это количество энергии, испускаемой единицей площади поверхности тела в единицу времени единичном интервале частот.
\end{definition}

\begin{definition}
Поглощательная способность $\alpha_\omega$ (спектральный коэффициент поглощения) —
это отношение энергии поглощенной поверхностью тела к энергии, падающей на поверхность тела. Обе энергии (падающая и поглощенная) берутся в расчете на единицу площади поверхности тела, единицу времени и единичный интервал частот.
\end{definition}
\textbf{Закон Кирхгофа}
\begin{statement}
Отношение излучательной способности любого тела к его поглощательной способности одинаково для всех тел при данной температуре для данной частоты и не зависит от их формы и химической природы.
\end{statement}
Т.е. отношение испускательной и поглощательной способностей не зависит от природы тела и является универсальной функцией частоты (длины волны) и температуры:
\begin{gather*}
\frac{1}{4}CU_\omega(T)=\frac{r_\omega}{\alpha_\omega}
\end{gather*}
\end{document}