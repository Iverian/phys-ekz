\documentclass[__main__.tex]{subfiles}

\begin{document}

\qtitle{Ч}{21}
Атомная цепочка. Электрон в периодическом потенциале. Теорема Блоха. Зона Бриллюэна. Циклические граничные условия (Борна-Кармана). Зонная структура кристаллов.\\ 

Рассмотрим электроны  в идеальном кристалле, где атомы расположены в строгом порядке , определяемой решеткой Браве.\\
\begin{definition}
	\label{orp1}
	Решеткой или системой трансляций Браве называется набор элементарных трансляций или трансляционная группа, которыми может быть получена вся бесконечная кристаллическая решётка.
\end{definition}

 Следовательно, потенциал взаимодействия также обладает периодичностью решетки Браве:
 \begin{gather}
 	\label{c21-1}
  U(r+R)=U(r) 
 \end{gather}
 
 где R - любой из векторов решетки Браве.
 
 Уравение Шредингера ля электрона в периодической решетке :
 
 \begin{gather}
 	\label{c21-2}
 	  H \psi=(-(\hbar^2/2m) \nabla^2+U(r))\psi=E\psi(r)),  
 \end{gather}

где потенциал обладает свойством \ref{c21-1}.

 Одним из основных следствий периодичности потенциала является теорема Блоха.\\
 \begin{theorem}
 	Собственные волновые функции $\psi$ одноэлектронного гамильтониана
 	
 	 $$H=(-(\hbar^2/2m) \nabla^2+U(r)$$
 	 
 	 где U(r+R)=U(r) при всех R принадлежащих решетке Браве, могут быть выбраны в форме плоской волны , умноженной на функцию с периодиностью  решетки Браве, т.е.:
 	 
 	 $$\psi_{nk} (r)=e^{ikr}*u_{nk}(r),$$
 	 
 	 где $$u_{nk}(r+R)=u_{nk}(r),$$
 	 
 	 для всех R, принадлежащих решетке Браве. Здесь n - номер зоны, появление котoрого связано с тем, что для данного k имеется множество решений.
 	 В иной записи теорема Блоха имеет вид:
 	 
 	$$\psi(r+R)=e^{ikR}*\psi(r)$$
 \end{theorem}

\begin{definition}
	Граничное условие Борна-Кармана для периодического потенциала. Выберем "ящик" соразмерный элементарной ячейке соответствующей решетки Браве. Запишем его в виде:

$$\psi_{nk}(r+N_i a_i)=\psi_{nk}(r), i=1,2,3 $$

	Здесь $a_i$ - тройка основных векторов, а все $N_i$- целые числа достигающие величин порядка $N^{\frac{1}{3}} $, где $N=N_{1}N_{2}N_{3}$ - полное число элементарных ячеек в кристалле.
\end{definition}
	 Применяя к граничным условиям теорему Блоха, находим:\\
 \begin{gather}
	 	\label{ch1}
	 		\psi_{nk}(r+N_i a_i)=e^{iN_{i}ka_{i}} \psi_{nk}(r) ,i=1,2,3
	\end{gather}	 
Соотношение \ref{ch1} может выполняться только при условии (как вектора обратной решетки или решетки в пространстве Фурье)
$$	e^{iN_{i}ka_{i}}=1 ,i=1,2,3 $$
$$т.е. N_{i}ka_{i}=2\pi m_{j}$$
где $m_{j}$- целые числа.

Представим k в виде разложения по базису векторов $b_{j}$  обратной решетки 
$$k=\sum_{j=1,2,3}x_{j} b_{j},$$

где $x_{j}$ - действительные числа, $b_{j}a_{i}=2\pi \delta_{ij}.$

или запишем иначе:
$$k=\sum_{j=1,2,3}m_{j}/N_{j} b_{j},$$

Из предыдущей записи следует, что обьем $\delta k$ в k-пространстве , приходящийся на одно разрешенное k равен обьему параллелепипеда:

$$\delta k=(1/N)b_{1}(b_{2}\times b_{3})$$

Эта формула значит, что  число разрешенных волновых векторов, содержащихся в одной элементарной ячейке обратной решетки, равно числу ячеек в кристалле.



\begin{definition}
	Зоны Бриллюэна.
	Вектор k называют квазиволновым т.к. изменение вектора k на вектор обратной решетки не приводит к изменению состояния.
	Область $-\frac{b_{i}}{2}<k<\frac{b_{i}}{2}$ называется зоной Бриллюэна.
\end{definition}

Соответственно разделению к-пространства на зоны Бриллюэна, энергетический спектр электронов разделен на \textbf{энергетические зоны} : 1-й з.Бриллюэна. соответствует 1-я энергетическая зона, и т.д. Т.е., о зонах Бриллюэна говорим, когда имеем дело с к-пространством, об энергетических зонах, когда анализируем Е(к). 
\end{document}