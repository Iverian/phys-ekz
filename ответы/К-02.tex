\documentclass[__main__.tex]{subfiles}

\begin{document}

\qtitle{К}{02}
Абсолютно чёрное тело: испускательная способность, энергетическая светимость. Закон Стефана-Больцмана.\\ 


\begin{definition}
    Энергетическая светимость $R$ (интегральная плотность потока энергии излучения) — равна энергии, испускаемой в единицу времени единицей поверхности излучающего тела по всем направлениям.
\end{definition}


В системе СИ энергетическая светимость измеряется в $\frac{Вт}{м^2}$\\

Энергетическая светимость зависит от температуры тела. Тепловое излучение состоит из волн различных частот. Для характеристики теплового излучения важно знать, какая энергия, в каком диапазоне частот излучается телом. Поэтому вводят дифференциальную характеристику $r_ω(ω,T)$  , называемую испускательной способностью тела, являющуюся  спектральной плотностью потока энергии излучения.\\

\begin{definition}
    Испускательная способность тела (спектральная плотность потока энергии излучения) — это количество энергии, испускаемой в единицу времени единицей поверхности тела в единичном интервале частот по всем направлениям.
\end{definition}   
Чтобы получить энергетическую светимость тела, надо проинтегрировать испускательную способность по всем частотам:
$$R = \int\limits_{0}^{\infty} r_ω dω$$

В системе СИ испускательная способность тела (спектральная плотность потока энергии излучения) измеряется в $\frac{Дж}{м^2}$.\\

Нагретое тело не только испускает энергию, но и поглощает ее. Для описания способности тела поглощать энергию падающего на его поверхность излучения вводится величина, которая так и называется: поглощательная способность.\\
\begin{definition}
    Поглощательная способность $α_ω$  (спектральный коэффициент поглощения) — равна отношению энергии поглощенной поверхностью тела к энергии, падающей на поверхность тела. Обе энергии (падающая и поглощенная) берутся в расчете на единицу площади, единицу времени и единичный интервал частот.
\end{definition}
Поглощательная способность равна той доли, которую — в заданном спектральном интервале $dω$  — поглощенная энергия излучения $dE_ω^{ПОГЛ}$  составляет от падающей $dE_ω^{ПАД}$ энергии излучения. Другими словами:

$$α_ω = \frac{dE_ω^{ПОГЛ}}{dE_ω^{ПАД}} $$

Очевидно, что поглощательная способность тела является безразмерной величиной, не превышающей единицу. \\

\begin{definition}
    Абсолютно черное тело — это тело, способное поглощать при любой температуре все падающее на него излучение всех частот.
\end{definition}

Для абсолютно черного тела $α_ω ≡ 1$

Тел с такими свойствами в природе не бывает, это очередная физическая идеализация.\\

Будем поочередно помещать в полость различные тела. Все они находятся в одинаковых условиях, в окружении одного и того же излучения. Обозначим $f(ω,T)$  энергию, падающую в единицу времени на единицу поверхности тела в единичном интервале частот. Согласно определению поглощательной способности тело поглощает энергию $α_ω*f(ω,T)$   В состоянии равновесия эта энергия должна быть равна испущенной телом энергии:

\begin{gather}
r_ω=α_ω*f(ω,T)
\end{gather}

Различные тела в полости имеют разную поглощательную способность, следовательно, у них будет и разная испускательная способность, так что отношение $\frac{r_ω}{α_ω}$   не зависит от конкретного тела, помещенного в полость:

\begin{gather}
\llabel{_8:kirh}
\frac{r_ω}{α_ω}=f(ω,T)
\end{gather}

С другой стороны, испускательная способность тела не зависит от полости, в которую оно помещено, но лишь от свойств тела. Таким образом, функция $f(ω,T)$ есть универсальная функция частоты и температуры, не зависящая ни от свойств полости, ни от характеристик тела в ней. Соотношение (\lref{_8:kirh}) выражает закон Кирхгофа.\\

Отношение испускательной и поглощательной способности тела не зависит от природы тела. Для всех тел функция  $f(ω,T)$ есть универсальная функция частоты и температуры (функция Кирхгофа).\\

Строго говоря, сформулированное выше утверждение справедливо в условиях термодинамического равновесия, наличие которого здесь и ниже всегда предполагается.
Для абсолютно черного тела $α_ω≡1$
откуда следует физическая интерпретация универсальной функции Кирхгофа $f(ω,T)$: она представляет собой испускательную способность абсолютно черного тела, то есть

$$r_ω^{*} = f(ω,T)$$

(Характеристики абсолютно черного тела будем помечать звездочкой, а само тело называть нередко просто «черным», а не абсолютно черным).

Установим теперь связь между испускательной способностью черного тела $f(ω,T)$ и спектральной плотностью  $u(ω,T)$ стандартного излучения в полости (выше мы назвали его излучением черного тела). Сравнивая размерности этих величин, видим, что отношение $\frac{f}{u}$ имеет размерность скорости. Единственная величина, имеющая размерность скорости, которая ассоциируется с электромагнитными волнами в вакууме, — это скорость света $c$. Поэтому искомое соотношение должно иметь вид:

$$f(ω,T)=βcu(ω,T)$$

Итак, энергетическая светимость черного тела  $R^{*}(T)$  связана с плотностью энергии в полости $U(T)$ соотношением:

\begin{gather}
R^{*} (T)=\frac{cU(T)}{4}  
\end{gather}


Закон Стефана-Больцмана — Энергетическая светимость абсолютно черного тела пропорциональна четвертой степени абсолютной температуры.\\

$$E_σ = \int\limits_{0}^{\infty} f(ω,T)dω = σ\cdot T^4$$

Из формулы видно, что при повышении температуры светимость тела не просто возрастает — она возрастает в значительно большей степени.


\end{document}