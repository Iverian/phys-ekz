\documentclass[__main__.tex]{subfiles}

\begin{document}

\qtitle{Э}{05}
Система уравнений Максвелла-Лоренца. Материальные уравнения. Условия на границе раздела двух диэлектриков. Условия на границе раздела двух магнетиков.\\

Вывод для Гениев!\\
Формулы для пацанов!\\

\textbf{Уравнения Максвела-Лоренца в дифференциальной форме}
\begin{gather*}
    \begin{cases}
        \operatorname{rot} \vec{E} 
        =
        -\frac{1}{c}\frac{\partial \vec{H}}{\partial t},           \\
        \operatorname{div} \vec{H} 
        = 
        0,                                                                   \\
        \operatorname{div}\vec{E}
        =
        4\pi\rho,                                                            \\
        \operatorname{rot}\vec{H}=
        \frac{1}{c}\frac{\partial\vec{E}}{\partial t}+\frac{4\pi}{c}\vec{j}, \\
    \end{cases}
\end{gather*}

\textbf{Уравнения Максвела-Лоренца в интегральной форме}
\begin{gather*}
	\begin{cases}
		\oint \vec{E} dl = \frac{1}{c}\frac{\partial}{\partial t}\int \vec{H} df,\\
		\oint \vec{H} df = 0, \\
		\oint \vec{H} dl = \frac{1}{c}\frac{\partial}{\partial t}\int \vec{E} df + \frac{4\pi}{c}\int \vec{j} df,\\
		\oint \vec{E} df = 4\pi\int \rho dV
	\end{cases}
\end{gather*}

\textbf{Материальные уравнения}

\begin{gather*}
    \begin{cases}
        \vec{D}=\varepsilon_{0}\varepsilon\vec{E}=\varepsilon_{0}(1+\chi_{e})\vec{E} \\
        \vec{B}=\mu_{0}\mu\vec{H}=\mu_{0}(1+\chi _{m})\vec{H}                        \\
    \end{cases}
\end{gather*}
где $\varepsilon-$ относительная диэлектрическая проницательность, $\mu-$ относительная магнитная проницаемость, $\chi_{e}$ -- диэлектрическая восприимчивость, $\chi _{m}$ -- магнитная восприимчивость.

\textbf{Граничные условия для диэлектриков (самое необходимое)} \emph{Если на границе раздела двух однородных изотропных диэлектриков сторонних зарядов нет, то при переходе этой границы $(\vec{E},\vec{\tau})$ и $(\vec{D},\vec{n})$ изменяются непрерывно, без скачка; составляющие $(\vec{E},\vec{n})$ и $(\vec{D},\vec{\tau})$ испытывают скачок.}\\

\textbf{Граничные условия для магнетиков (все, что есть (никто не поверит что вы знаете док-во))} \emph{Если на границе раздела двух однородных изотропных магнетиков сторонних зарядов нет, то при переходе этой границы $(\vec{H},\vec{\tau})$ и $(\vec{B},\vec{n})$ изменяются непрерывно, без скачка; составляющие $(\vec{H},\vec{n})$ и $(\vec{B},\vec{\tau})$ испытывают скачок.}\\

\textbf{Вывод Системы уравнений Максвелла-Лоренца в вакууме(для лохов)}\\

\begin{gather*}
    \vec{H}=\nabla\times\vec{A},
    \quad
    \vec{E}=-\frac{1}{c}\frac{\partial\vec{A}}{\partial t}-\nabla\varphi,
\end{gather*}
где $\vec{H}$ -- напряжённость магнитного поля, $\vec{E}$ -- напряжённость электрического поля, $\vec{A}$ - векторный потенциал поля, $\varphi$ -- скалярный потенциал.

Определим $\nabla\times\vec{E}$:
\begin{gather*}
    \nabla\times\vec{E}=-\frac{1}{c}\frac{\partial}{\partial t}\nabla\times\vec{A}-\nabla\times\nabla\varphi.
\end{gather*}

Ротор градиента равен нулю.
\begin{gather}
    \llabel{e-02-rotE}
    \nabla\times\vec{E}=-\frac{1}{c}\frac{\partial\vec{H}}{\partial t}.
\end{gather}

Возьмём дивергенцию от обеих частей уравнения $rot A = H$ (дивергенция ротора равна нулю):
\begin{gather}
    \llabel{e-02-divH}
    \nabla\vec{H}=0
\end{gather}

Уравнения $\lref{e-02-rotE}$ и  $\lref{e-02-divH}$ составляют первую пару уравнений Максвелла.

В интегральной форме:
\begin{gather*}
    \oint\vec{E}d\vec{l}
    =
    \frac{1}{c}\frac{\partial}{\partial t}\int\vec{H}d\vec{f},\\
    \oint\vec{H}d\vec{f}=0.
\end{gather*}

Для вывода второй пары уравнений Максвелла нам нужно знать, что действие:
\begin{gather}
    \llabel{e-02-S}
    S = -\sum\int mcds - \frac{1}{c^2}\int A_ij^id\Omega - \frac{1}{16\pi c}\int F_{ik}F^{ik}d\Omega.
\end{gather}

При нахождении уравнений поля их принципа наименьшего действия мы должны считать заданным движения зарядов и должны варьировать только потенциалы поля(играющие здесь роль "координат" системы); при нахождении уравнений движения мы, наоборот, считали поле заданным и варьировали траекторию частицы.
Поэтому вариация первого члена в $\lref{e-02-S}$ равна теперь нулю, а во втором не должен варьироваться ток $j^i$. Таким образом,
\begin{gather*}
    \delta S = -\frac{1}{c}\big[\frac{1}{c}j^i\delta A_i + \frac{1}{8\pi}F^{ik}\delta F_{ik}\big]d\Omega = 0
\end{gather*}
(при варьировании во втором члене учтено, что $F^{ik}\delta F_{ik} \equiv F_{ik}\delta F^{ik}$). Подставляя
\begin{gather*}
    F_{ik} = \frac{\partial A_k}{\partial x^i} - \frac{\partial A_i}{\partial x^k},
\end{gather*}
имеем:
\begin{gather*}
    \delta S = -\frac{1}{c}\int \big\{\frac{1}{c}j^i\delta A_i + \frac{1}{8\pi}F^{ik}\frac{\partial}{\partial x^i}\delta A_k - \frac{1}{8\pi}F^{ik}\frac{\partial}{\partial x^k}\delta A_i\big\}d\Omega.
\end{gather*}
Во втором члене меняем местами индексы $i$ и $k$, по которым производится суммирование, и, кроме того, заменяем $F_{ki}$ на $-F_{ik}$.\\
Тогда мы получим:
\begin{gather*}
    \delta S = -\frac{1}{c}\int\big\{\frac{1}{c}j^i\delta A_i - \frac{1}{4\pi}F^{ik}\frac{\partial}{\partial x^k}\delta A_i\big\}d\Omega.
\end{gather*}
Второй из этих интегралов берём по частям, т.е. применяем теорему Гаусса:
\begin{gather}
    \llabel{e-02-deltaS}
    \delta S = -\frac{1}{c}\int\big\{\frac{1}{c}j^i + \frac{1}{4\pi}\frac{\partial F^{ik}}{\partial x^k}\big\}\delta A_id\Omega - \frac{1}{4\pi c}\int F^{ik}\delta A_idS_k\big|.
\end{gather}
Во втором члене  мы должны взять его значение на пределах интегрирования. Пределами интегрирования по координатам является бесконечность, где поле исчезает. На пределах же интегрирования по времени, т.е. в заданные начальный и конечный моменты времени, вариация потенциалов равна нулю, так как по смыслу принципа наименьшего действия потенциалы в эти моменты заданы. Таким образом, второй член в $\lref{e-02-deltaS}$ равен нулю, и мы находим:
\begin{gather*}
    \int(\frac{1}{c}j^i + \frac{1}{4\pi}\frac{\partial F^{ik}}{\partial x^k})\delta A_id\Omega = 0.
\end{gather*}
Ввиду того, что по смыслу принципа наименьшего действия вариации $\delta A_i$ произвольны, нулю должен равняться коэффициент при $\delta A_i$, т.е.
\begin{gather}
    \llabel{e-02-deltaF}
    \frac{\partial F^{ik}}{\partial x^k} = -\frac{4\pi}{c}j^i.
\end{gather}
Перепишем эти четыре ($i = 0, 1, 2, 3$) уравнения в трёхмерной форме. При $i=1$ имеем:
\begin{gather*}
    \frac{1}{c}\frac{\partial F^{10}}{\partial t} + \frac{\partial F^{11}}{\partial x} + \frac{\partial F^{12}}{\partial y} + \frac{\partial F^{13}}{\partial z} = -\frac{4\pi}{c}j^1.
\end{gather*}
Подставляя значения составляющих тензора $F^{ik}$, находим:
\begin{gather*}
    \frac{1}{c}\frac{\partial E_x}{\partial t} - \frac{\partial H_z}{\partial y} + \frac{\partial H_y}{\partial z} = -\frac{4\pi}{c}j_x.
\end{gather*}
Вместе с двумя следующими $(i = 2, 3)$ уравнениями они могут быть записаны как одно векторное:
\begin{gather}
    \llabel{e-02-rotH}
    \nabla\times\vec{H}=\frac{1}{c}\frac{\partial\vec{E}}{\partial t} + \frac{4\pi}{c}\vec{j}.
\end{gather}
Наконец, уравнение с $i=0$ даёт:
\begin{gather}
    \llabel{e-02-divE}
    \nabla\vec{E}=4\pi\rho.
\end{gather}
Уравнения $\lref{e-02-rotH}$ и $\lref{e-02-divE}$ и составляют вторую пару уравнения Максвелла.\\
В интегральной форме:\\
\begin{gather*}
    \oint\vec{H}d\vec{l}=\frac{1}{c}\frac{\partial}{\partial t}\int\vec{E}d\vec{f}+\frac{4\pi}{c}\int\vec{j}d\vec{f},\\
    \oint\vec{E}d\vec{f}=4\pi\int\rho{dV}.
\end{gather*}
Уравнения Максвелла являются основными уравнениями электродинамики.\\
\begin{wrapfigure}{r}{.3\linewidth}
    \centering
    \def\svgwidth{1\linewidth}
    \input{img/e-03-1.pdf_tex}
    \caption{контур $C$}
    \llabel{e03:fig:abcd}
\end{wrapfigure}

\textbf{Условия на границе раздела диэлектриков:} рассмотрим вектор напряженности электрического поля $\vec{E}$ и вектор электрического смещения $\vec{D}$ на границе раздела двух однородных изотропных (симметрия относительно поворота в пространстве) диэлектриков с диэлектрической проницаемостью $\varepsilon_1$ и $\varepsilon_2$, при отсутствии на границе свободных зарядов.

Построим внутри границы раздела диэлектриков \textbf{1} и \textbf{2} прямоугольный контур $C = MNPK$ длины $l$, ориентировав его как показано на Рис. \lref{e03:fig:abcd}. Условия на границе получим с помощью теоремы Гаусса и теоремы о циркуляции:
\begin{flalign}
    & \oint\limits_{C}(\vec{E},\vec{\tau})dl = 0, \llabel{eq:circ} \\
    & \oiint\limits_{S}(\vec{D},\vec{n})d\sigma = Q^{\text{(стор)}}. \llabel{eq:gauss}
\end{flalign}

Пусть напряженность поля около границы в \textbf{1} равна $\vec{E}_1$ и $\vec{E}_2$ в $\textbf{2}$. Сторона контура должна иметь такую длину, чтобы в ее пределах поле было однородным, т.е. можно было бы $\vec{E}$ считать одинаковым. Согласно (\lref{eq:circ}) $(\vec{E}_1,\vec{\tau})l + (\vec{E}_2,\vec{\tau})l = 0$. Т.к. знаки интегралов по $KM$ и $PN$ различны, а значения интегралов на $MN$ и $PK$ ничтожно малы, то $(\vec{E}_1,\vec{\tau})l-(\vec{E}_2,\vec{\tau})l=0$, следовательно, получим:
\begin{gather}
    (\vec{E}_1,\vec{\tau}) = (\vec{E}_2,\vec{\tau}),
\end{gather}
где $\vec{\tau} = \vec{j}$. Получим, что \emph{Касательная составляющая вектора $\vec{E}$ оказывается одинаковой по обе стороны границы раздела (т.е. не испытывает скачка при переходе через границу раздела).} Для $\vec{D}$ из (\lref{eq:deq}) получим из $\varepsilon_1\neq\varepsilon_2$:
\begin{gather}
    \begin{cases}
        (\vec{D}_1,\vec{\tau}) = \varepsilon_1\varepsilon_0(\vec{E}_1,\vec{\tau}) \\
        (\vec{D}_2,\vec{\tau}) = \varepsilon_2\varepsilon_0(\vec{E}_2,\vec{\tau})
    \end{cases},
\end{gather}
получим:
\begin{gather}
    \frac{(\vec{D}_1,\vec{\tau})}{(\vec{D}_2,\vec{\tau})} = \frac{\varepsilon_1}{\varepsilon_2},
\end{gather}
т.е. \emph{Тангенциальная составляющая $\vec{D}$ при переходе границы раздела диэлектриков испытывает скачок.}

Пусть на границе раздела двух диэлектриков имеются сторонние заряды. Замкнутую поверхность выберем в виде цилиндра малой высоты $h$, расположим его на границе раздела двух диэлектриков так, чтобы одно основание было в диэлектрике \textbf{1}, а второе – в \textbf{2}. Сечение цилиндра возьмем таким, чтобы в пределах каждого его торца вектор $\vec{D}$ был одинаков. Тогда, учитывая, что поток через боковую поверхность можно представить в виде $\left<(\vec{D},\vec{n})\right>S_{\text{бок}}$, где $S_{\text{бок}}$ -- площадь боковой поверхности, а $\left<(\vec{D},\vec{n})\right>$ -- среднее значение $(\vec{D},\vec{n})$ на ней, по (\lref{eq:gauss}) для $\vec{D}$ получим:
\begin{gather}
    \Psi_{\vec{D}} = (\vec{D}_1,\vec{n}_1)\Delta S + (\vec{D}_2,\vec{n}_2)\Delta S + \left<(\vec{D},\vec{n})\right>S_{\text{бок}} = \sigma\Delta S,
\end{gather}
при $h\rightarrow 0\colon S_{\text{бок}}\rightarrow 0$ и $h\rightarrow 0\colon\Psi_{\vec{D}}\rightarrow (\vec{D}_1,\vec{n}_1)\Delta S + (\vec{D}_2,\vec{n}_2)\Delta S = \sigma\Delta S$. Тогда $(\vec{D}_1,\vec{n}_1)+(\vec{D}_2,\vec{n}_2)=\sigma$, тогда получим $(\vec{D}_2,\vec{n})-(\vec{D}_1,\vec{n})=\sigma$, где $\vec{n}$ -- общая нормаль к $S$. Следовательно, получим при $\sigma=0$ (отсутствии внешних зарядов) $(\vec{D}_1,\vec{n})=(\vec{D}_2,\vec{n})$; согласно (\lref{eq:deq}) получим:
\begin{gather}
    \frac{(\vec{E}_1,\vec{n})}{(\vec{E}_2,\vec{n})}=\frac{\varepsilon_2}{\varepsilon_1}.
\end{gather}

\end{document}