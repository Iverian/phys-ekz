\documentclass[__main__.tex]{subfiles}

\begin{document}

\qtitle{Ч}{04}
Гильбертово пространство состояний частицы заданного сорта, полная абелева совокупность, принцип дополнительности Бора, получите и прокомментируйте обобщенное соотношение неопределённостей Хайзенберга величин $A$ и $B$: $\Delta{A}\Delta{B}\ge\frac{\left|\left<\left[\hat{A},\hat{B}\right]\right>\right|}{2}$.\\ 


\textbf{принцип дополнительности Бора}\\
Получение экспериментальной информации об одних физических величинах, описывающих микрообъект, неизбежно связано с потерей информации о некоторых других величинах, дополнительных к первым. Такими взаимно дополнительными величинами являются, например, координата частицы и ее импульс (или скорость), потенциальная и кинетическая энергии и др.
\textbf{обобщенное соотношение неопределённостей Хайзенберга}\\
Квантовая механика позволяет нам определить вероятность того или иного результата эксперемента и среднее значение физической величины.\\
Пусть $\hat{A}\left|j\right> = a_j \left|j\right>$. Тогда если состояние системы определяется вектором $\left|c\right>$, то\\
\begin{gather}
\left<A\right> = \left<c|\hat{A}|c\right> = \sum_j \left<c|\hat{A}|c\right>\left<j|c\right> = \sum_j \left<c|j\right>\left<j|c\right>a_j = \sum_j |c_j|^2 a_j
\end{gather}
где $|c_j|^2$ - вероятность того, что если до изменения система находится в состоянии $\left|c\right>$, то сразу после измерения она окажется в состоянии $\left|j\right>$.\\
\begin{gather}
\Delta A \equiv \sigma[A] \equiv \sqrt{\left<(A-\left<A\right>)^2\right>}\\
\left<(A-\left<A\right>)^2\right> = \left<A^2-2A\left<A\right>+\left<A\right>^2\right> = \left<A^2\right> - \left<A\right>^2\\
(\Delta A)^2 = \left<(A-\left<A\right>)^2\right> = \left< c|(A-\left<A\right>)(A-\left<A\right>)|c\right> = \left<d|d\right>
\end{gather}
Аналогично $(\Delta B)^2 = \left<f|f\right>$.\\
\begin{gather}
(\Delta A)^2(\Delta B)^2 = \left<d|d\right>\left<f|f\right> \geq \frac{|\left<d|f\right>|^2}{\left<d|f\right>\left<f|d\right>}\\
\left<d|f\right> = \left<c|(\hat{A}-\left<A\right>)(\hat{B} - \left<B\right>)|c\right> = \left<AB\right> - \left< A\right> \left< B\right>\\
\left<d|f\right>|^2 = Re^2+Im^2 \geq Im^2 = \left<\frac{[A,B]}{2i}\right>^2\\
\end{gather}
Взглянув на определение неопределенностей мы можем осозновать, что соотношение неопределенностей - это некие утверждения о средних.\\
\end{document}