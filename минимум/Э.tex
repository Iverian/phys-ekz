\documentclass[__minimum__.tex]{subfiles}

\begin{document}

\begin{comment}
+ 1) Закон Киргофа
+ 2) Закон смещения Вина
+ 3) Закон Стефана Больцмана
+ 4) Формула Рэлея-Джинса
+ 5) Формула Планка
+ 6) Соотношение неопределенностей Гейзенберга
+ 7) Постулаты квантовой механики
+ 8) Коммутатор
+ 9) Эрмитов оператор
+ 10) Постулат о вещественных велечинах
+ 11) Постулат об измерениях
+ 12) Наблюдаемый оператор
+ 13) Равенство Парсевалля
+ 14) Квантомеханические средние
+ 15) Временное и стационарное уравнения Шредингера
+ 16) Вектор плотности тока вероятности
+ 17) Теорема Эренфеста
+ 18) Дуальное пространство
+ 19) Коммутационные соотношения
+ 20) Кванто-механический гармонический осцилятор
+ 21) Постулаты Бора
+ 22) Полный и собственный момент импульса
23) Матрицы Паули
\end{comment}

\ssection{Э Электрическтво и Магнентизм}

\textbf{Теорема Бора-ван Лёвена}\\
\begin{theorem}
В состоянии термодинамического равновесия система электрически заряженных частиц (электронов, атомных ядер и т. п.), помещённая в постоянное магнитное поле, не могла бы обладать магнитным моментом, если бы она строго подчинялась законам классической физики.
\end{theorem}

Одной из основных характеристик любого магнетика является намагниченность $\vec{M}$, представляющая собой магнитный момент единицы объема вещества:
\begin{gather*}
\vec{M}=\frac{\vec{P}_m}{V}
\end{gather*}
Намагниченность возрастает с увеличением напряженности магнитного поля:
\begin{gather*}
\vec{M}=\chi\vec{H}=\chi\frac{\vec{B}}{\mu\mu_0},
\end{gather*}
где $\chi$ --- магнитная восприимчивость, $\mu$ --- магнитная проницаемость.\\
Магнитная индукция, создаваемая в присутствии вещества, описывается соотношением:
\begin{gather*}
\vec{B}=\mu_0(\vec{H}+\vec{M}).
\end{gather*} 
И так как $\vec{B}=\mu\mu_0 \vec{H}$:
\begin{gather*}
\chi=\mu-1
\end{gather*}
Магнитная восприимчивость может быть как положительной, так и отрицательной. Вещества с положительной магнитной восприимчивостью, которые усиливают магнитное поле, называются парамагнетиками.\\

Внешнее магнитное поле стремится установить магнитные моменты атомов вдоль $\vec{B}$ в то время, как тепловое движение – разбросать их равномерно по всем направлениям. В результате устанавливается некоторая преимущественная ориентация магнитных моментов атомов вдоль поля. Пьер Кюри экспериментально установил, что магнитная восприимчивость парамагнетика зависит от температуры согласно закону (закон Кюри):
\begin{gather*}
\chi=\frac{C}{T}
\end{gather*}
где $C$ – постоянная Кюри, зависящая от рода вещества.\\

\textbf{Теорема о циркуляции вектора $\vec{H}$}\\
\begin{theorem}
Циркуляция вектора напряженности магнитного поля равна алгебраической сумме токов проводимости, которые охвачены замкнутым контуром, по которому рассматривается циркуляция:
\begin{gather*}
\oint\limits_L \vec{H}d\vec{r}=\sum I_m
\end{gather*}
\end{theorem}

\begin{definition}
Электромагнитная индукция --- явление возникновения электрического тока в замкнутом проводящем контуре при изменении во времени магнитного потока, пронизывающего контур.
\end{definition}
\begin{definition}
Магнитным потоком $\Phi$ через площадь $S$ контура называется величина 
\begin{gather*}
\Phi = BS\cos\alpha,
\end{gather*}
где $B$ -- модуль вектора магнитной индукции, $\alpha$ -- угол между вектором магнитной индукции и нормалью к плоскости контура.
\end{definition}

\textbf{Закон Фарадея}\\
При изменении магнитного потока в проводящем контуре возникает ЭДС индукции $\mathcal E_{\text{ИНД}}$, равная скорости изменения магнитного потока через поверхность, ограниченную контуром, взятой со знаком минус:
\begin{gather*}
\mathcal E_{\text{ИНД}}=-\frac{\Delta \Phi}{\Delta t}
\end{gather*}
\textbf{Правило Ленца}\\
Индукционный ток, возбуждаемый в замкнутом контуре при изменении магнитного потока, всегда направлен так, что создаваемое им магнитное поле препятствует изменению магнитного потока, вызывающего индукционный ток.\\

\begin{definition}
Явление самоиндукции - это возникновение в проводящем контуре ЭДС, создаваемой вследствие изменения силы тока в самом контуре.
\end{definition}
\begin{definition}
Индуктивность (или коэффициент самоиндукции) — коэффициент пропорциональности между электрическим током, текущим в каком-либо замкнутом контуре, и полным магнитным потоком, называемым также потокосцеплением, создаваемым этим током через поверхность, краем которой является этот контур.

Индуктивность является электрической инерцией, подобной механической инерции тел. А вот мерой этой электрической инерции как свойства проводника может служить служить ЭДС самоиндукции. Характеризуется свойством проводника противодействовать появлению, прекращению и всякому изменению электрического тока в нём.
\begin{gather*}
\Phi=LI,
\end{gather*}
где $L$ -- индуктивность контура, $I$ -- сила тока в контуре.
\end{definition}

\end{document}