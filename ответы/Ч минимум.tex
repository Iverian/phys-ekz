\documentclass[__main__.tex]{subfiles}

\begin{document}
	\begin{definition}
	Выражение $\hat{H}\psi = E\psi$ называется стационарным уравнением Шредингера, где $E = \hbar \omega$ - соотношение Эйнштейна.\\
	\end{definition}
	\begin{definition}
	Симметричная волновая функция описывает частицы, называемые бозонами.\\
	Бозоны - частицы с целым спином $(s = 0,1,2...)$\\
	Антисимметричная волновая функция описывает частицы, называемые фермионами.\\
	Фермионы - частицы с полуцелым спином $(s = \frac{1}{2},\frac{3}{2},\frac{5}{2}...)$\\
	\end{definition}
	\textbf{Эрмитов оператор}\\
	$\hat{A}\left|j\right> = a_j\left|j\right>$\\
	\textbf{Принцип неразичимости частиц одного сорта}\\
	В классической механике всегда есть возможность различить (пронумеровать объекты и сохранить нумерацию в течении всего времени их существования) одинаковые объекты, в то время как квантовомеханическое описание делает невозможным различить одинаковые объекты.
	\begin{gather}
	<x|\hat{A}|x> = <\widetilde{x}|\hat{A}|\widetilde{x}>
	\end{gather}
	- это математическая формулировка принципа неразличимости частиц. Не существует наблюдаемых отличий между системой в состоянии $|x>$ и состемой в состоянии $|\widetilde{x}> = \hat{P}|x>, где \hat{P} - \textit{оператор перестановки}
	\begin{defenition}
	\hat{P} -оператор перестановки, в основе его определения лежит действие на базисные векторы $|i> |j>$. Это означает, что под действием оператора на 2-хчастичное состояние, в котором 1-я частица находилась в состоянии $|i>$, а 2-я в состоянии $|j>$, получается 2-хчастичное состояние, в котором 1-я частица находится в состоянии $|j>$, а 2-я в состоянии $|i>$.$\hat{P}$ - эрмитов оператор и так же унитарен.\\
	\end{defenition}
	\textbf{Постулат симметризации}
	Состояния системы, содержащей N тождественных частиц, будут все либо симметричными, либо антисимметричными относительно перестановок этих N частиц.\\
	\textbf{Принцип запрета Паули}\\
	Система из двух одинаковых фермионов не может находится в состоянии $\Psi$ содержащих одинаковые одночастичные состояния $\phi$ - Принцип запрета Паули.\\
	\textbf{Обобщенное соотношение неопределённостей Хайзенберга}\\
	Квантовая механика позволяет нам определить вероятность того или иного результата эксперемента и среднее значение физической величины.\\
	Пусть $\hat{A}\left|j\right> = a_j \left|j\right>$.\\
	$\Delta{A}\Delta{B}\ge\frac{\left|\left<\left[\hat{A},\hat{B}\right]\right>\right|}{2}$\\
	\textbf{Факторизованное состояние}\\
	Общее состояние составной системы $ \left|Gst\right>$в некоторых случаях можно представить в форме $\left|fst\right> = \left|st\right>_1\left|stt\right>_2$
	\textif{факторизованного состояния}  (в виде тензорного произведения (знак опу-
	щен) состояний каждого элемента системы).\\
	\textbf{Запутанные системы}\\
	Не все состояния системы можно представить в виде произведениясостояний системы. Это приводит к наличию \textit{запутанных состояний}.\\
	\textbf{Эксперимент Штерна-Герлаха}\\
	На атом обладающий магнитным моментом $\vec{\mu}$, в неоднородном магнитном поле $\vec{B}$ должна действовать сила
	\begin{gather*}
		\vec{f} = \left(\vec{\mu}\cdot\nabla\right)\vec{B}
	\end{gather*}
	Чтоб рассуждать было проще, пусть $B_x, B_y = 0$  и $B_z \neq 0$, тогда
	\begin{gather*}
		\vec{f} = \left(\mu_x\partial_xB_z+\mu_y\partial_yB_z+\mu_z\partial_zB_z\right)\vec{e}_z
	\end{gather*}
	направлена либо по либо против $z$. Пусть неоднородность создана лишь вдоль $z$ тогда $\partial_xB_z=0; \partial_yB_z=0$\\\\
	Опыт выглядит довольно просто: направим пучок атомов в область неоднородного магнитного поля и посмотрим что будет на выходе из этой области.
	В области неоднородного магнитного поля на атомы с $\mu_z \neq 0$ действует сила $f_z \sim \mu_z$; в результате они отклоняются от первоначального направления; величина отклонения тем больше, чем больше $|\mu_z|$; вверх или вниз зависит от знака $\mu_z$.
	Согласно квантовой механике $\mu_z$ квантуется $\Rightarrow$ исходный пучок обязан расщепиться на число пучков, равное числу разрешенных $\mu_z$. В итоге должно получится что-то вроде(\textbf{пространственное квантование} -- набор эквидистантных пятен на экране):
	\begin{figure}[h]
		\center{\includegraphics[width=0.9\linewidth]{ch-18}}
	\end{figure}
	\textbf{Гипотеза спина электрона}\\
	Подсчитаем число пучков на выходе из неоднородного $\vec{B}$. Вопрос: Сколько возможных $\mu_z$ ?
	\begin{gather*}
		\mu_z = -\mu_Bm
	\end{gather*}
	где $\mu_B$ - магнетон Бора.\\
	При фиксированном $l$ число возможных $m$ составляет $2l+1$ и учитывая, что $l,n \in N$, приходим к выводу, что число возможных $m$ - нечетное. Проверим это:\\
	Для этого пропустим пучок атомов водорода с $l=0$ через область неоднородного магнитного поля. Поскольку $l = 0 \Rightarrow m = 0 \Rightarrow \mu_z = 0 \Rightarrow$ пучок должен проследовать прямо в центр экрана, НО это не так - он расщепляется на две составляющие! Таким образом приходиться предположить, что электрон обладает собственным моментом импульса или спином, которому соответствует некоторый магнитный момент. Приходим к определению полного момента импульса частицы:
	\begin{gather*}
		\hat{J} = \hat{L}+\hat{S}
	\end{gather*} 
	где $\hat{L}$ - орбитальный момент, $\hat{S}$ - спин.\\\\
	Ввиду того (как оказалось), что электрон обладает внутренней степенью свободы(спином), то теперь для того, чтобы охарактеризовать его стационарные состояния в поле ядра нам потребуется уже не 3 квантовых числа а четыре (добавляется спиновое квантовое число - $m_s$)\\\\
	\textbf{Функция распределения Бозе-Эйнштейна}
	Функция распределения Бозе-Эйнштейна для частиц с целым (в том числе нулевым) спином имеет вид:
	$$f(\varepsilon) = 1/(e^{\frac{\varepsilon - \mu}{kT}} - 1)$$
	где $\varepsilon$ - кинетическая энергия частицы, $\mu$ - химический потенциал, зависящий от температуры.
	\textbf{Функция распределения Ферми-Дирака}
	Функция распределения Ферми-Дирака определяет вероятность заселения (из-за двух возможнх ориентаций спина на каждом уровне энергии могут находиться 2 электрона) уровня с энергией $\varepsilon$ и имеет вид:
	$$f(\varepsilon) = 1/(e^{\frac{\varepsilon - \varepsilon_F}{kT}} + 1)$$
	Здесь $\varepsilon_F$ - энергия Ферми, параметр, определяемый из очевидного условия, что сумма заселенности всех уровней энергии должна равняться полному числу электронов:
	$$\sum f(\varepsilon) = N_\varepsilon$$
\end{document}