\documentclass[__main__.tex]{subfiles}

\begin{document}

\qtitle{О}{15}
На поляризатор падает неполяризованное излучение. Известна доля $\alpha$, которую составляет интенсивность прошедшего излучения по отношению к падающему. Если за первым поляризатором поместить второй такой же поляризатор, то доля интенсивности пропущенного через оба поляризатора излучения по отношению к интенсивности излучения, падающего на первый поляризатор, составит $\alpha'$. Найдите угол между плоскостями пропускания поляризаторов.\\

\begin{theorem}[Закон Малюса]
    Интенсивность прошедшего через поляризатор линейно - поляризованного света интенсивности $I_0$ выражается соотношением:
    \begin{gather}
        I=k_{a}I_0\cos^2\varphi,
        \llabel{o15:mal}
    \end{gather}
    где $k_{a}$ -- коэффициент пропускания поляризатора, $\varphi$ -- разность между углами поляризации поляризатора и падающего света.
\end{theorem}
Получим из (\lref{o15:mal}) и условия:
\begin{flalign}
    & I_{1}={\alpha}I_{0}  \llabel{o15:a} \\
    & I_{2}={\alpha'}I_{0} \llabel{o15:b} \\
\end{flalign}
где $I_{1}$ -- интенсивность света, прошедшего через первый поляризатор, $I_{2}$ -- через второй. Т.к. для естественного света:
\begin{gather}
    I_{1}=\frac{1}{2}k_{a}I_{0},
\end{gather}
т.к. для неполяризованного света все направления поляризации равновероятны, т.е. получим из (\lref{o15:a}):
\begin{gather}
    k_{a}=2\alpha,
\end{gather}
Из (\lref{o15:b}) получим:
\begin{flalign}
    \begin{split}
        &
        \alpha'=k_{a}\cos^2\varphi
        \Longrightarrow
        \sqrt{\frac{\alpha'}{2\alpha}}=\cos\varphi
        \Longrightarrow\\
        \Longrightarrow&
        \varphi=\arccos\sqrt{\frac{\alpha'}{2\alpha}}
    \end{split}
\end{flalign}

\end{document}