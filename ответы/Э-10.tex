\documentclass[__main__.tex]{subfiles}

\begin{document}

\qtitle{Э}{10}
Найдите поле $\vec{B}$, порождаемое тороидом (число витков $N$), по которому течёт постоянный ток $I$. Воспользуйтесь полученным результатом, чтобы посчитать поле $\vec{B}$, порождаемое бесконечным соленоидом (число витков на единицу длины $n$), по которому течёт ток $I$.\\ 

Тороид представляет собой тонкий провод, навитый на каркас, имеющий форму тора. Он эквивалентен системе одинаковых круговых токов, центр которых расположены по окружности. Возьмем контур в виде окружности радиуса $r$, центр которой совпадает с центром тороида. В сил симметрии вектор $\vec{B}$ в каждой точке должен быть направлен по касательной к контуру. Следовательно:
\begin{gather*}
\oint B_ldl=B\cdot 2\pi r,
\end{gather*}
где $B$ --- магнитная индукция в тех точках, где проходит контур.\\
Если контур проходит внутри тороида, он охватывает ток $2\pi R NI$ ($R$ --- радиус тороида). В этом случае 
\begin{gather*}
B\cdot 2\pi r=\mu_02\pi RNI,
\end{gather*}
откуда 
\begin{gather*}
B=\mu_0NI\frac{R}{r}.
\end{gather*}
Контур, проходящий вне тороида, токов не охватывает, поэтому для него $B\cdot 2\pi r=0$. Таким образом, вне тороида магнитная индукция равна нулю.\\
Для тороида, радиус которого $R$ значительно превосходит радиус витка, отношение $\dfrac{R}{r}$ для всех точек внутри тороида мало отличается от единицы и тогда получается такая же формула, как для бесконечно длинного соленоида:
\begin{gather*}
B=\mu_0NI.
\end{gather*}
В этом случае поле можно считать однородным в каждом из сечений тороида (условно, имея в виду модуль вектора магнитной индукции).
\end{document}