\documentclass[__main__.tex]{subfiles}

\begin{document}

\qtitle{Э}{08}
Теорема Бора-ван Лёвена. Парамагнетики, закон Кюри. Намагниченность, магнитная восприимчивость, магнитная проницаемость. Теорема о циркуляции вектора $\vec{H}$.\\ 

\textbf{Теорема Бора-ван Лёвена}\\
\begin{theorem}
В состоянии термодинамического равновесия система электрически заряженных частиц (электронов, атомных ядер и т. п.), помещённая в постоянное магнитное поле, не могла бы обладать магнитным моментом, если бы она строго подчинялась законам классической физики.
\end{theorem}

Одной из основных характеристик любого магнетика является намагниченность $\vec{M}$, представляющая собой магнитный момент единицы объема вещества:
\begin{gather*}
\vec{M}=\frac{\vec{P}_m}{V}
\end{gather*}
Намагниченность возрастает с увеличением напряженности магнитного поля:
\begin{gather*}
\vec{M}=\chi\vec{H}=\chi\frac{\vec{B}}{\mu\mu_0},
\end{gather*}
где $\chi$ --- магнитная восприимчивость, $\mu$ --- магнитная проницаемость.\\
Магнитная индукция, создаваемая в присутствии вещества, описывается соотношением:
\begin{gather*}
\vec{B}=\mu_0(\vec{H}+\vec{M}).
\end{gather*} 
И так как $\vec{B}=\mu\mu_0 \vec{H}$:
\begin{gather*}
\chi=\mu-1
\end{gather*}
Магнитная восприимчивость может быть как положительной, так и отрицательной. Вещества с положительной магнитной восприимчивостью, которые усиливают магнитное поле, называются парамагнетиками.\\

Внешнее магнитное поле стремится установить магнитные моменты атомов вдоль $\vec{B}$ в то время, как тепловое движение – разбросать их равномерно по всем направлениям. В результате устанавливается некоторая преимущественная ориентация магнитных моментов атомов вдоль поля. Пьер Кюри экспериментально установил, что магнитная восприимчивость парамагнетика зависит от температуры согласно закону (закон Кюри):
\begin{gather*}
\chi=\frac{C}{T}
\end{gather*}
где $C$ – постоянная Кюри, зависящая от рода вещества.\\

\textbf{Теорема о циркуляции вектора $\vec{H}$}\\
\begin{theorem}
Циркуляция вектора напряженности магнитного поля равна алгебраической сумме токов проводимости, которые охвачены замкнутым контуром, по которому рассматривается циркуляция:
\begin{gather*}
\oint\limits_L \vec{H}d\vec{r}=\sum I_m
\end{gather*}
\end{theorem}
\end{document}