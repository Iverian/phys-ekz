\documentclass[__main__.tex]{subfiles}

\begin{document}

\qtitle{К}{14}
Найдите волновые функции основного и первого возбуждённого состояний квантовомеханического гармонического осциллятора.\\


$\forall n\in\mathbb{N}\cup\{0\}\colon|n\rangle$ образуют дискретное представление, называемое \textit{представлением чисел заполнения}. Вектора этого базиса можно получить из $|0\rangle$. Чтобы показать это, выведем рекуррентное соотношения для векторов базиса с учетом нормировки:

Пусть $\langle n|n\rangle=1$, тогда для $|n+1\rangle$:
\begin{gather*}
    \langle n+1|n+1\rangle = \langle n|\hat{b}\hat{b}^\dagger|n\rangle = \langle n|\hat{n}+1|n\rangle
    = n+1
\end{gather*}
Тогда $|n+1\rangle = \frac{1}{\sqrt{n+1}}\hat{b}^\dagger|n\rangle$ или
\begin{gather*}
    |n\rangle = \frac{1}{\sqrt{n!}}\left(\hat{b}^\dagger\right)^n|0\rangle
\end{gather*}

Тогда из уравнения $\hat{b}|0\rangle = 0$ найдем $|0\rangle$. Согласно (\lref{_10:b}) и т.к. $\hat{p}=-i\hbar\frac{d}{dx}$, получим:
\begin{gather*}
    \hat{b}
    =
    \left.
    \sqrt{\frac{m\omega}{2\hbar}}x+\sqrt{\frac{\hbar}{2m\omega}}\frac{d}{dx}
    \right|_{x_0=\sqrt{\frac{\hbar}{m\omega}}}
    =
    \frac{x}{\sqrt{2}x_0}+\frac{x_0}{\sqrt{2}}\frac{d}{dx}
\end{gather*}
Таким образом уравнение $\hat{b}|0\rangle = 0$ примет вид:
\begin{gather*}
    \frac{d}{dx}|0\rangle + \frac{x}{x_0^2}|0\rangle = 0
\end{gather*}
Решением этого линейного ОДУ первого порядка является функция:
\begin{gather*}
    |0\rangle = C\exp\left(-\frac{x^2}{2x_0^2}\right)
\end{gather*}
Константу $C$ найдем из условия нормировки:
\begin{gather*}
    C^2\int\limits_{-\infty}^{\infty}\exp\left(-\frac{x^2}{2x_0^2}\right)dx=1,
\end{gather*}
Этот интеграл можно вычислить из вероятностных соображений, сведением к плотности нормального распределения $N(0,\sigma)$: $p(x)=\frac{1}{\sqrt{2\pi}\sigma}\exp\left(-\frac{x^2}{2\sigma^2}\right)$: $\int\limits_{-\infty}^{\infty}p(x)dx=1$.
В итоге получим
\begin{gather*}
    |0\rangle = \frac{1}{\pi^{1/4}x_0^{1/2}}\exp\left(-\frac{x^2}{x_0^2}\right)
\end{gather*}
Вычислим $|1\rangle = \hat{b}^\dagger|0\rangle$
\begin{gather*}
    \hat{b}^\dagger
    =
    \frac{x}{\sqrt{2}x_0}-\frac{x_0}{\sqrt{2}}\frac{d}{dx}
\end{gather*}
Тогда
\begin{gather*}
    |1\rangle
    =
    \frac{1}{\sqrt{2x_0\sqrt{\pi}}}\left(\frac{x}{x_0}-x_0\frac{d}{dx}\right)\exp\left(-\frac{x^2}{x_0^2}\right)
    =
    \frac{\sqrt{2}}{\pi^{1/4}x_0^{3/2}}x\exp\left(-\frac{x^2}{x_0^2}\right) + \frac{x}{x_0}
\end{gather*}

В итоге получили, что собственный базис гамильтониана осциллятора образует \textit{представлением чисел заполнения} $\forall n\in\mathbb{N}\cup\{0\}\colon|n\rangle$, где $|n\rangle = \frac{1}{\sqrt{n!}}\left(\hat{b}^\dagger\right)^n|0\rangle$, а $|0\rangle$ находится из $\hat{b}|0\rangle = 0$.\\

Волновая функция основного состояния: $|0\rangle = \frac{1}{\sqrt{x_0\sqrt{\pi}}}\exp\left(-\frac{x^2}{x_0^2}\right)$, первого возбужденного: $|1\rangle = \frac{\sqrt{2}}{\pi^{1/4}x_0^{3/2}}x\exp\left(-\frac{x^2}{x_0^2}\right) + \frac{x}{x_0}$

\end{document}