\documentclass[__main__.tex]{subfiles}

\begin{document}

\qtitle{К}{35}
Получите формулу Планка. Сопоставьте вычисленную с её помощью объёмную спектральную плотность энергии равновесного теплового излучения с предсказанием Рэлея-Джинса. Прокомментируйте результат.\\

\begin{definition}
    Закон Рэлея-Джинса -- закон излучения для спектральной плотности мощности излучения (спектральной плотности энергетической светимости) $u(\omega, T)$ абсолютно чёрного тела, полученный в рамках классической статистики.
    \begin{flalign}
        \llabel{k35:form}
        u(\omega,T)=kT\cdot\frac{\omega^2}{\pi^2c^3},
    \end{flalign}
\end{definition}
\begin{definition}
    Формула Планка — выражение для спектральной плотности мощности излучения (спектральной плотности энергетической светимости) абсолютно чёрного тела, которое было получено Максом Планком для плотности энергии излучения $u(\omega,T)$
\end{definition}
Для получения формулы Планка положим, что излучение испускается в виде отдельных порций с энергией, пропорциональной частоте $\omega$:
\begin{gather}
    \varepsilon=\hbar\omega,
\end{gather}
тогда энергия излучения примет вид:
\begin{gather}
    \varepsilon_n=n\hbar\omega,
\end{gather}
где $n$ -- количество порций (\emph{квантов}). Согласно распределению Больцмана вероятность $P_n$ того, что энергия излучения равна $\varepsilon_n$ определяется как:
\begin{gather}
    P_n
    =
    A\exp\left(-\frac{\varepsilon_n}{kT}\right)
\end{gather}
где $A$ определим из условия нормировки $\sum_{n}P_n=1$:
\begin{gather}
    A=\frac{1}{\sum_{n}\exp\left(-\frac{\varepsilon_n}{kT}\right)},
\end{gather}
математическое ожидание энергии излучения $\overline\varepsilon$ тогда примет вид:
\begin{flalign}
    \overline{\varepsilon}
    =&
    \sum_{n}\varepsilon_{n}P_{n}
    =
    \left.
    \frac{\sum_{n}n\hbar\omega\exp\left(-n\hbar\omega/kT\right)}{\sum_{n}\exp\left(-n\hbar\omega/kT\right)}
    \right|_{x=\hbar\omega/kT}
    =
    \hbar\omega\frac{\sum_{n}n\exp(-nx)}{\sum_{n}\exp(-nx)}
    =\\
    =&
    -\hbar\omega\frac{d}{dx}\ln\sum_{n}\exp(-nx)
\end{flalign}
По формуле суммы бесконечно убывающей геом. прогрессии (знаменатель $e^{-x}<1$):
\begin{gather}
    \sum_{n}\exp(-nx)=\frac{1}{1-e^{-x}},
\end{gather}
итого получим:
\begin{gather}
    \overline\varepsilon
    =
    -\hbar\omega{d_x}\ln\frac{1}{1-e^{-x}}
    =
    \frac{\hbar\omega}{e^{x}-1}
    =
    \frac{\hbar\omega}{\exp(\hbar\omega/kT) - 1}
\end{gather}
заметим, что при $\hbar\rightarrow{0}\colon\overline{\varepsilon}\rightarrow{kT}$, т.е. если энергия могла бы принимать непрерывный ряд значений (классический случай), то ее среднее значение имело бы вид $kT$, из этих соображений заменим $kT$ в (\lref{k35:form}) на полученное $\overline{\varepsilon}$ и получим формулу Планка
\begin{flalign}
    u(\omega,T)
    =
    \frac{\omega^2}{\pi^2c^3}\cdot\frac{\hbar\omega}{\exp(\hbar\omega/kT)-1}
\end{flalign}

\end{document}