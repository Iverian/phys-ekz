\documentclass[__main__.tex]{subfiles}

\begin{document}

\qtitle{Ч}{01}
Докажите, что собственными значениями эрмитова оператора, квадрат которого равен тождественному оператору, могут быть только $\pm 1$. Воспользуйтесь доказанным утверждением, чтобы ответить на вопрос о возможных результатах измерения проекции спина электрона на выделенное направление?\\ 

\begin{proof}
\begin{gather*}
\hat{A}|j\rangle = a_j|j\rangle\\
\hat{1}|j\rangle =\hat{A}^2|j\rangle = a_j\hat{A}|j\rangle = (a_j)^2|j\rangle \Longrightarrow (a_j)^2=1 \Longrightarrow a_j=\pm 1
\end{gather*}
\end{proof}
Полезное обобщение:
\begin{gather*}
(\hat{\sigma}\cdot \vec{n})^2=\hat{1} \Longrightarrow \mbox {СЗ оператора } \hat{\sigma}\cdot \vec{n} \mbox{ равны } \pm 1
\end{gather*}

\begin{statement}
Какое направление $\vec{n}$ ни возьми (в спиновом пространстве состояний), существует такое состояние, в котором проекция спина на это направление с определенностью положительна, то есть вероятность обнаружить проекцию спина по $\vec{n}$ есть 1.
\end{statement}
\begin{proof}
Для существования ненулевого решения квадратной СЛАУ
\begin{gather*}
\left(\hat{\sigma}\vec{n}\right)\left(
\begin{matrix}
\lambda \\
\rho \\
\end{matrix}
\right) = +\left(
\begin{matrix}
\lambda \\
\rho \\
\end{matrix}
\right) \qquad \mbox{или} \qquad \left(
\begin{matrix}
n_z-1 & n_x-in_y \\
n_x+in_y & -n_z-1 \\
\end{matrix}
\right)\left(
\begin{matrix}
\lambda \\
\rho \\
\end{matrix}
\right) = 0
\end{gather*}
Необходимо и достаточно, чтобы:
\begin{gather*}
\det{(\hat{\sigma}\vec{n}-1)}=0
\end{gather*}
Проверим, выполняется ли это условие:
\begin{gather*}
(n_z-1)(-n_z-1)-(n_x-in_y)(n_x+in_y)=1-n_z^2-n_x^2-n_y^2
\end{gather*}
Не забываем, что $n$ -- единичный вектор (модуль равен 1), поэтому 
\begin{gather*}
1-n_z^2-n_x^2-n_y^2=0 \Longrightarrow \det{(\hat{\sigma}\vec{n}-1)}=0 
\end{gather*}
\end{proof}

\end{document}