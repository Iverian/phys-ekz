\documentclass[__main__.tex]{subfiles}

\begin{document}

\qtitle{К}{38}
Квантовомеханический гармонический осциллятор: представление чисел заполнения, энергетический спектр, волновые функции основного и первого возбуждённого состояний.\\ 

Рассмотрим классический гармонический осциллятор: его полная механическая энергия примет вид
\begin{gather*}
E = \frac{kx^2}{2}+\frac{p^2}{2m},
\end{gather*}
для перехода к квантовому аналогу заменим жесткость пружины $k$ (какие еще пружины в микромире?) на ее выражение через собственную частоту $\omega$: $k=m\omega^2$, теперь по аналогии с классическим выражением запишем квантовомеханический гамильтониан (оператор полной энергии системы):
\begin{gather*}
\hat{H} = \frac{m\omega^2\hat{x}^2}{2}+\frac{\hat{p}^2}{2m}
\end{gather*}
Теперь для нахождения \textit{энергетического спектра} $E_n$ нужно найти собственные значения $\hat{H}$:
\begin{gather*}
\hat{H}|v\rangle = E|v\rangle
\end{gather*}
Для решения этой задачи воспользуемся \textit{оператором числа элементарных возбуждений} $\hat{n}=\hat{b}^\dagger\hat{b}$, где $\hat{b}$ и $\hat{b}^\dagger$ -- операторы такие, что $[\hat{b},\hat{b}^\dagger]=1$. Введем для этой задачи оператор $\hat{b}$: 
\begin{gather}
\hat{b} = \sqrt{\frac{m\omega}{2\hbar}}\hat{x} + i\sqrt{\frac{1}{2m\omega\hbar}}\hat{p}
\llabel{_10:b}
\end{gather}
Т.к. $\hat{x}$ и $\hat{p}$ эрмитово самосопряженные, то
\begin{gather*}
\hat{b} = \sqrt{\frac{m\omega}{2\hbar}}\hat{x} - i\sqrt{\frac{1}{2m\omega\hbar}}\hat{p}
\end{gather*}
Проверим $[\hat{b},\hat{b}^\dagger]=1$:
\begin{gather*}
[\hat{b},\hat{b}^\dagger]
=
\left.
-\frac{i}{2\hbar}[\hat{x},\hat{p}]+\frac{i}{2\hbar}[\hat{p},\hat{x}]
\right|_{[\hat{x},\hat{p}]=-[\hat{p},\hat{x}]}
=
\left.
-\frac{i}{\hbar}[\hat{x},\hat{p}]
\right|_{[\hat{x},\hat{p}]=i\hbar}
=
1
\end{gather*}
Оператор $\hat{n}$ примет вид:
\begin{gather*}
\hat{n}=\hat{b}^\dagger\hat{b}=\frac{m\omega\hat{x}^2}{2\hbar}+\frac{\hat{p}^2}{2m\omega\hbar}-\frac{1}{2}
\end{gather*}
Тогда получим:
\begin{gather*}
\hbar\omega\left(\hat{n}+\frac{1}{2}\right)
=
\frac{m\omega^2\hat{x}^2}{2}+\frac{\hat{p}^2}{2m}
=
\hat{H}
\end{gather*}
Обозначим за $|n\rangle$ собственную функцию оператора $\hat{n}$, отвечающую значению $n$ тогда:
\begin{gather*}
\hat{H}|n\rangle
=
\hbar\omega\left(\hat{n}+\frac{1}{2}\right)|n\rangle
=
\hbar\omega\left(n+\frac{1}{2}\right)|n\rangle
=
E_n|n\rangle
\end{gather*}
Получим, что собственные функции $\hat{H}$ совпадают с СФ оператора $\hat{n}$. Покажем, что СФ $\hat{n}$ -- натуральные числа. а его собственные значения (стационарные значения энергии осциллятора) $E_n=\hbar\omega\left(n+\frac{1}{2}\right),n\in\mathbb{N}\cup\{0\}$. Покажем, что индекс $n$ дискретный и СЗ $\hat{n}$ -- натуральные числа.

\begin{theorem}
	Собственные значения $n$ оператора $|n\rangle$ -- неотрицательные действительные числа.
	\llabel{_10:th-nneg}
\end{theorem}
\begin{proof}
	Рассмотрим собственные функции $\hat{n}$ -- $|n\rangle$, отвечающие собственным значениям $n$:
	\begin{gather*}
	\hat{n}|n\rangle = n|n\rangle
	\Longrightarrow
	\left< n| \hat{n}|n\right> = n \left<n|n\right>, \left<n|n\right> \ge 0,
	\end{gather*}
	с другой стороны
	\begin{gather*}
	\left< n| \hat{n}|n\right> = (\left< n\right|\hat{b})(\hat{b}^\dagger\left|n\right>) =
	\left< n'|n'\right> \ge 0
	\end{gather*}
	Получим, что $n\ge 0$, причем $n=0\Leftrightarrow \hat{b}|n\rangle = 0$.
\end{proof}
\begin{theorem}
	Если $|n\rangle$ -- собственная функция оператора $\hat{n}$, отвечающая значению $n$, то $\hat{b}^\dagger|n\rangle$ -- собственная функция $\hat{n}$, отвечающая $n+1$:
	$$
	\hat{n}|n\rangle=n|n\rangle\Longrightarrow\hat{n}\hat{b}^\dagger|n\rangle=(n+1)\hat{b}^\dagger|n\rangle
	$$
	\llabel{_10:th-seq}
\end{theorem}
\begin{proof}
	\begin{gather*}
	\hat{n}\hat{b}^\dagger|n\rangle
	|_{\hat{n}=\hat{b}^\dagger\hat{b}}
	=
	\hat{b}^\dagger\hat{b}\hat{b}^\dagger|n\rangle
	\end{gather*}
	Т.к. $[\hat{b},\hat{b}^\dagger]=\hat{b}\hat{b}^\dagger-\hat{b}^\dagger\hat{b}=1\Longrightarrow\hat{b}\hat{b}^\dagger=\hat{b}^\dagger\hat{b}+1=\hat{n}+1$, то:
	\begin{gather*}
	\hat{b}^\dagger\hat{b}\hat{b}^\dagger|n\rangle
	=
	\hat{b}^\dagger(\hat{n}+1)|n\rangle
	=
	(n+1)\hat{b}^\dagger|n\rangle
	\end{gather*}
\end{proof}
\begin{theorem}
	Если $|n\rangle$ -- собственная функция оператора $\hat{n}$, отвечающая значению $n$, то $\hat{b}|n\rangle$ -- собственная функция $\hat{n}$, отвечающая $n-1$:
	$$
	\hat{n}|n\rangle=n|n\rangle\Longrightarrow\hat{n}\hat{b}|n\rangle=(n-1)\hat{b}|n\rangle
	$$
	\llabel{_10:th-min}
\end{theorem}
\begin{proof}
	Аналогично (\lref{_10:th-seq})
\end{proof}
Тогда (\lref{_10:th-nneg}) и (\lref{_10:th-min}) вместе образуют некоторое противоречие: применяя последовательно $\hat{b}$ к $|n\rangle$ мы рано или поздно придем к собственной функции, отвечающей отрицательному значению $n-k<0,k\in\mathbb{N}$. Из этого можно заключить, что такой процесс приближения всегда должен приводить к $0$, следовательно все собственные значения оператора $\hat{n}$ -- натуральные числа.

Т.е. существует вектор $|0\rangle$:
\begin{gather*}
\hat{n}|0\rangle = 0
\Longrightarrow
\langle 0|\hat{n}|0\rangle = \langle 0|\hat{b}^\dagger\hat{b}|0\rangle = \langle 0'|0'\rangle = 0
\Longrightarrow
|0'\rangle = \hat{b}|0\rangle = 0
\end{gather*}

Таким образом $\forall n\in\mathbb{N}\cup\{0\}\colon|n\rangle$ образуют дискретное представление, называемое \textit{представлением чисел заполнения}. Вектора этого базиса можно получить из $|0\rangle$. Чтобы показать это, выведем рекуррентное соотношения для векторов базиса с учетом нормировки:

Пусть $\langle n|n\rangle=1$, тогда для $|n+1\rangle$:
\begin{gather*}
\langle n+1|n+1\rangle = \langle n|\hat{b}\hat{b}^\dagger|n\rangle = \langle n|\hat{n}+1|n\rangle
= n+1
\end{gather*}
Тогда $|n+1\rangle = \frac{1}{\sqrt{n+1}}\hat{b}^\dagger|n\rangle$ или
\begin{gather*}
|n\rangle = \frac{1}{\sqrt{n!}}\left(\hat{b}^\dagger\right)^n|0\rangle
\end{gather*}

Тогда из уравнения $\hat{b}|0\rangle = 0$ найдем $|0\rangle$. Согласно (\lref{_10:b}) и т.к. $\hat{p}=-i\hbar\frac{d}{dx}$, получим:
\begin{gather*}
\hat{b}
=
\left.
\sqrt{\frac{m\omega}{2\hbar}}x+\sqrt{\frac{\hbar}{2m\omega}}\frac{d}{dx}
\right|_{x_0=\sqrt{\frac{\hbar}{m\omega}}}
=
\frac{x}{\sqrt{2}x_0}+\frac{x_0}{\sqrt{2}}\frac{d}{dx}
\end{gather*}
Таким образом уравнение $\hat{b}|0\rangle = 0$ примет вид:
\begin{gather*}
\frac{d}{dx}|0\rangle + \frac{x}{x_0^2}|0\rangle = 0
\end{gather*}
Решением этого линейного ОДУ первого порядка является функция:
\begin{gather*}
|0\rangle = C\exp\left(-\frac{x^2}{2x_0^2}\right)
\end{gather*}
Константу $C$ найдем из условия нормировки:
\begin{gather*}
C^2\int\limits_{-\infty}^{\infty}\exp\left(-\frac{x^2}{2x_0^2}\right)dx=1,
\end{gather*}
Этот интеграл можно вычислить из вероятностных соображений, сведением к плотности нормального распределения $N(0,\sigma)$: $p(x)=\frac{1}{\sqrt{2\pi}\sigma}\exp\left(-\frac{x^2}{2\sigma^2}\right)$: $\int\limits_{-\infty}^{\infty}p(x)dx=1$.
В итоге получим
\begin{gather*}
|0\rangle = \frac{1}{\pi^{1/4}x_0^{1/2}}\exp\left(-\frac{x^2}{x_0^2}\right)
\end{gather*}
Вычислим $|1\rangle = \hat{b}^\dagger|0\rangle$
\begin{gather*}
\hat{b}^\dagger
=
\frac{x}{\sqrt{2}x_0}-\frac{x_0}{\sqrt{2}}\frac{d}{dx}
\end{gather*}
Тогда
\begin{gather*}
|1\rangle
=
\frac{1}{\sqrt{2x_0\sqrt{\pi}}}\left(\frac{x}{x_0}-x_0\frac{d}{dx}\right)\exp\left(-\frac{x^2}{x_0^2}\right)
=
\frac{\sqrt{2}}{\pi^{1/4}x_0^{3/2}}x\exp\left(-\frac{x^2}{x_0^2}\right) + \frac{x}{x_0}
\end{gather*}

В итоге получили, что собственный базис гамильтониана осциллятора образует \textit{представлением чисел заполнения} $\forall n\in\mathbb{N}\cup\{0\}\colon|n\rangle$, где $|n\rangle = \frac{1}{\sqrt{n!}}\left(\hat{b}^\dagger\right)^n|0\rangle$, а $|0\rangle$ находится из $\hat{b}|0\rangle = 0$.\\
Энергетический спектр гармонического осциллятора примет вид $\forall n\in\mathbb{N}\cup\{0\}E_n=\hbar\omega\left(n+\frac{1}{2}\right)$.\\
Волновая функция основного состояния: $|0\rangle = \frac{1}{\sqrt{x_0\sqrt{\pi}}}\exp\left(-\frac{x^2}{x_0^2}\right)$, первого возбужденного: $|1\rangle = \frac{\sqrt{2}}{\pi^{1/4}x_0^{3/2}}x\exp\left(-\frac{x^2}{x_0^2}\right) + \frac{x}{x_0}$


\end{document}