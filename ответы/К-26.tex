\documentclass[__main__.tex]{subfiles}

\begin{document}

\qtitle{К}{26}
Сформулируйте постулат квантовой механики о физических величинах. Покажите, что все собственные значения эрмитова оператора суть вещественные числа. Сформулируйте постулат квантовой механики об измерениях. Покажите, что собственные функции эрмитова оператора, отвечающие различным его собственным значениям, ортогональны.\\ 

\textbf{Постулат квантовой механики о физических величинах}\\
Каждой физической величине ставится в соответствие эрмитов оператор обладающий полной системой собственных функций.
\begin{definition}
	Оператор $\hat{A}^+$ называется сопряженным оператором, если $(\hat{A}^+ \psi_2, \psi_1) = (\psi_2, \hat{A} \psi_1)$.
\end{definition}

\begin{definition}
	Эрмитовым оператором наз-ся линейный и самосопряженный оператор (т.е $\hat{A}^+ = \hat{A}$).
\end{definition}

\begin{statement}
	Все собственные значения эрмитова оператора $\hat{A}$ - вещественные числа.
\end{statement}
\begin{proof}
	Пусть $\psi_a$ - собственная функция оператора $\hat{A}$, тогда при воздействии оператора $\hat{A}$ на эту функцию получим эту же функцию домноженную на какое-то число $a$:
	\begin{gather*}
		\hat{A}\psi_a = a\psi_a
	\end{gather*}
	Необходимо доказать, что $a$ - вещественное. Для этого воспользуемся утверждением того, что если $\hat{A}$ - эрмитов оператор, то скалярное произведение $(\psi_a,\hat{A}\psi_a)$ - вещественное.
	\begin{gather*}
		(\psi_a,\hat{A}\psi_a) = (\psi_a,a\psi_a) = a(\psi_a,\psi_a)
	\end{gather*}
	Ввиду того, скалярное произведение по определению $(\psi_a,\psi_a) = \vert\vert\psi_a\vert\vert^2$ - вещественное число, то "$a$" ничего не остается как быть только вещественным т.к левая часть уравнения по утверждению выше  - вещественная.
\end{proof}

\textit{Постулат квантовой механики об измерениях:}
Пусть измеряется некоторая величина $A$ и ей в соответствие ставится оператор $\hat{A}$. Если он эрмитов ${\hat{A}^+=\hat{A}}$; $\phi_n(x)$ - его собственные функции, и функция состояния, в котором находится система, задана как:
$$\psi(x)=\sum_{n=0}c_n\phi_n(x)$$
Тогда в момент измерения система перейдет в состояние, описываемое собственной функцией оператора $\hat{A}$, причем вероятность перейти в состояние $\phi_n(x)$ пропорциональна ${\left|c_n\right|}^2$. В результате измерения получится соответствующие собственное значение оператора.


	
	СФ самосопряженного оператора $\hat F$ ортогональны, то есть:
	\begin{gather*}
	\int \psi^{\star}_m(x) \psi_n(x)dx = 0\quad при\quad m\ne n
	\end{gather*}
	
	Запишем два уравнения двух различных собственных значений $f_n$ и $f_m$, взяв от второго комплексное сопряжение:
	\begin{gather*}
	\hat{F}\psi_n(x)=f_n\psi_n(x),\\
	\hat{F}^\star \psi^\star_m(x)=f_m\psi^\star_m(x).
	\end{gather*}
	
	Первое ур-е умножим слева на $\psi^\star_m(x)$, второе - на $\psi_n(x)$, затем проинтегрируем и вычтем из первого ур-я второе:
	\begin{gather*}
	\int \psi^\star_m(x)\hat{F}\psi_n(x)dx - 
	\int \psi_n(x)\hat{F}^\star \psi^\star_m(x)dx=
	(f_n-f_m)\int \psi^\star_m(x)\psi_n(x)dx.
	\end{gather*}
	
	Левая часть ур-я равна нулю, поскольку вычитаемое сводится к уменьшаемому после опреции транспонирования и использования св-ва самосопряжения оп-ра $\hat{F}$. Учитывая, что $f_n\ne f_m$, получаем:
	\begin{gather*}
	0=\int \psi^\star_m(x)\psi_n(x)dx=
	\langle\psi_m\mid\psi_n\rangle
	\end{gather*}
	
	Объединив результат с условием нормировки, можно написать так:
	\begin{gather*}
	\langle\psi_m\mid\psi_n\rangle=\delta_{mn}
	\end{gather*}


\end{document}