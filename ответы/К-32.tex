\documentclass[__main__.tex]{subfiles}

\begin{document}

\qtitle{К}{32}
Частица массы $m$ находится в трёхмерной кубической потенциальной яме (ребро куба $l$) с абсолютно непроницаемыми стенками. Найти разность энергий второго и третьего возбуждённого уровней. Указать их кратность вырождения.\\

Рассмотрим кубическую потенциальную яму $\mathbf{K}$ с шириной ребра $l$, где потенциальная энергия $U(\vec{r})$ имеет вид:
\begin{gather*}
    U(\vec{r})
    =
    \begin{cases}
        0      & \vec{r}\in\mathbf{K}     \\
        \infty & \vec{r}\not\in\mathbf{K} \\
    \end{cases},
\end{gather*}
тогда уравнение Шредингера примет вид:
\begin{gather*}
\begin{cases}
\Delta\psi+k^2\psi=0\\
\left.\psi\right|_\mathbf{K}=0
\end{cases}
\qquad
k^2=\frac{2m}{\hbar^2}T
\end{gather*}
Решение этого уравнения примет вид с учетом краевых условий:
\begin{gather*}
\psi(\vec{r})
=
A\sin\frac{\pi{n_1}x}{l}\sin\frac{\pi{n_2}y}{l}\sin\frac{\pi{n_3}z}{l}, 
\end{gather*}
Коэффициент $A$ найдем из условия нормировки $\int\limits_{V}\psi(\vec{r})=1$:
\begin{gather*}
A=\sqrt{\frac{8}{l^3}},
\end{gather*}
подставив волновую функцию в уравнение найдем кинетическую энергию, получим спектр энергии:
\begin{gather*}
T_{n_1,n_2,n_3}=\frac{\pi^2\hbar^2}{2ml^2}\left(n_1^2+n_2^2+n_3^2\right),
\end{gather*} 
Тогда уровень возбуждения определяется суммой квадратов $n_1^2+n_2^2+n_3^2$, второй уровень: $n_1^2+n_2^2+n_3^2=2^2+1^2+1^2=6$, третий: $n_1^2+n_2^2+n_3^2=2^2+2^2+1^2=9$.Получим:
\begin{gather*}
T^{(3)}-T^{(2)}
=
\frac{\pi^2\hbar^2}{2ml^2}(9-6)
=
\frac{\pi^2\hbar^2}{2ml^2}(3)
=
T^{(1)}
\end{gather*} 

\end{document}