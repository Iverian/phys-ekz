\documentclass[__main__.tex]{subfiles}

\begin{document}

\qtitle{К}{29}
Электрон находится в одномерной прямоугольной потенциальной яме с бесконечно высокими стенками. Определите ширину ямы, если известно, что разность энергии между первым и вторым возбуждёнными состояниями составляет $0.3\text{эВ}$.\\ 

\textbf{Решение}
Энергия частицы
$$
T_n = \frac{\hbar^2k_n^2}{2m} = \frac{\mathcal{\pi}^2\hbar^2}{2ml^2}n^2
$$
Для первого и второго возбуждённых состояний:
$$
T_2 = \frac{4\mathcal{\pi}^2\hbar^2}{2ml^2} \hspace{0.5cm}
T_1 = \frac{\mathcal{\pi}^2\hbar^2}{2ml^2}
$$
По условию
\begin{gather*}
T = T_2 - T_1 = \frac{4\mathcal{\pi}^2\hbar^2}{2ml^2} - \frac{\mathcal{\pi}^2\hbar^2}{2ml^2} = 0.3\\ 
\frac{3\mathcal{\pi}^2\hbar^2}{2ml^2} = 0.3
\end{gather*}
Ширина ямы:
$l = \sqrt{\frac{3\mathcal{\pi}^2\hbar^2}{0.6m}}$
\end{document}