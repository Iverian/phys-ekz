\documentclass[__minimum__.tex]{subfiles}

\begin{document}

\ssection{П Теория поля}

\paragraph{4-вектор потенциала поля:} потенциал $A^\mu=\left(\varphi, \vec{A}\right)$.

\paragraph{Векторные характеристики поля:} электромагнитное поле характеризуется $\vec{E}$ -- \emph{напряженностью электрического поля}, $\vec{B}$ -- \emph{индукцией магнитного поля}, где:
\begin{flalign}
    &\vec{E}=-\partial_{t}\vec{A}-\nabla\varphi,\\
    &\vec{B}=\nabla\times\vec{A},
\end{flalign}

\paragraph{Тензор Максвелла:} основной тензор электродинамики $F_{\alpha\beta}$:
\begin{gather}
    F_{\alpha\beta}=\partial_{\alpha}A_\beta-\partial_{\beta}A_\alpha,
\end{gather}
в матричной форме получим:
\begin{gather}
    F_{\alpha\beta}
    =
    \left[
        \begin{matrix}
            0    & E_x  & E_y  & E_z  \\
            -E_x & 0    & -B_z & B_y  \\
            -E_y & B_z  & 0    & -B_x \\
            -E_z & -B_y & B_y  & 0    \\
        \end{matrix}
        \right],
    \qquad
    F^{\alpha\beta}
    =
    \left[
        \begin{matrix}
            0   & -E_x & -E_y & -E_z \\
            E_x & 0    & -B_z & B_y  \\
            E_y & B_z  & 0    & -B_x \\
            E_z & -B_y & B_x  & 0    \\
        \end{matrix}
        \right],
\end{gather}

\paragraph{Инварианты электромагнитного поля:} коэффициенты характеристического многочлена тензора Максвелла:
\begin{gather}
    I_{1}
    =
    F^{\alpha\beta}F_{\alpha\beta}
    =
    2(\vec{E}^2-\vec{B}^2),
    \qquad
    I_{2}
    =
    \operatorname{det}F_{\alpha\beta}
    =
    \left<\vec{E},\vec{B}\right>^2,
\end{gather}

\paragraph{Уравнение движения заряда в поле: }
\begin{gather}
    m\frac{du_\alpha}{dt}
    =
    qF_{\alpha\beta}u^\beta
\end{gather}

\paragraph{Действие для заряда $q$ в электромагнитном поле:}
\begin{gather}
    \mathscr{A}
    =
    \int
    \left[
    -mds-qA_{i}dx^{i}
    \right],
\end{gather}

\paragraph{Уравнения Максвелла:} в тензорной форме:
\begin{gather}
    \begin{cases}
        \partial_{\gamma}F_{\alpha\beta}+\partial_{\beta}F_{\gamma\alpha}+\partial_{\alpha}F_{\beta\gamma}=0 \\
        \partial^{\alpha}F_{\alpha\beta}=-j_{\beta}
    \end{cases}
\end{gather}

в векторной форме:
\begin{gather}
    \begin{cases}
        \nabla\times\vec{E}=-\partial_{t}\vec{B} \\
        \nabla\vec{B}=0
    \end{cases}
    \qquad
    \begin{cases}
        \nabla\vec{E}=\rho \\
        \nabla\times\vec{B}=\partial_{t}\vec{E}+\vec{j}
    \end{cases}
\end{gather}

\paragraph{Закон сохранения электрического заряда:}
Из второго уравнения Максвелла $\partial^{\alpha}F_{\alpha\beta}=-j_{\beta}$:
\begin{flalign*}
    \begin{split}
        &
        \partial^{\alpha}F_{\alpha\beta}=-j_{\beta}
        \Longrightarrow
        \partial^{\alpha\beta}F_{\alpha\beta}=-\partial^{\alpha}j_{\alpha}
        \Longrightarrow\\
        \Longrightarrow&
        \partial^{\alpha}j_{\alpha}=0,
    \end{split}
\end{flalign*}
$\partial^{\alpha\beta}F_{\alpha\beta}=0$ т.к. это свертка кососимметрического тензора $F_{\alpha\beta}$ и симметрического $\partial^{\alpha\beta}$. В векторной форме имеет вид:
\begin{gather}
    \partial_{t}\rho + \nabla\vec{j}=0,
\end{gather}

\paragraph{Действие для поля:}
\begin{gather}
    \mathscr{A}
    =
    \int{Ldt}
    =
    \int\limits_{\Omega}\Lambda(q_{i},\partial_{\mu}q_{i}){dV}{dt},
\end{gather}
где $\Lambda(q_{i},\partial_{\mu}q_{i})$ -- \emph{<<плотность лагранижиана>>}. Для электромагнитного поля функция $\Lambda$ примет вид:
\begin{gather}
    \underset{emf}\Lambda
    =
    \frac{1}{4}F_{\alpha\beta}F^{\alpha\beta},
    \label{p:lemf}
\end{gather}
для свободных зарядов и токов $\Lambda$ примет вид:
\begin{gather}
    \underset{int}\Lambda
    =
    -j^{\alpha}A_{\alpha},
\end{gather}
где $j^\mu=\rho\frac{dx^{\mu}}{dt}$ -- \emph{4-вектор плотности тока}.

\paragraph{Полевые уравнения Лагранжа:}
\begin{gather}
    \frac{\partial{\Lambda}}{\partial{q^{i}}}
    -
    \partial_\beta\frac{\partial\Lambda}{\partial(\partial_\beta{q^i})}
    =
    0
\end{gather}

\paragraph{Сохраняющийся Нетеровский ток (относительно симметрии трансляции)}

\begin{theorem}[Нётер]
    Инвариантность действия относительно некоторой непрерывной группы симметрии приводит к соответствующему закону сохранения.
\end{theorem}

\begin{gather}
    \mathscr{J}^{\mu}
    =
    \left[
        \Lambda\delta\indices{^\mu_\nu}
        -
        \frac{\partial\Lambda}{\partial(\partial_\mu{A})}
        \right]\delta{x^\nu}
    +
    \frac{\partial\Lambda}{\partial(\partial_\mu{A})}
    \delta{A}
\end{gather}

\paragraph{Канонический тензор энергии - импульса}
\begin{gather}
    T\indices{^\alpha_\beta}
    =
    \frac{\partial\Lambda}{\partial(\partial_\alpha{q^i})}\partial_\beta{q^i}
    -
    \Lambda\delta\indices{^\alpha_\beta},
\end{gather}
для электромагнитного поля получим из (\ref{p:lemf}):
\begin{gather}
    \underset{emf}{T\indices{^\mu_\nu}}
    =
    F^{\mu\lambda}\partial_{\nu}A_{\lambda}
    -
    \frac{1}{4}F_{\alpha\beta}F^{\alpha\beta}\delta\indices{^\mu_\nu},
\end{gather}
\emph{канонический тензор энергии - импульса не инвариантен относительно калибровочных преобразований}, чтобы сделать его таковым вводят \emph{симметризованный тензор энергии - импульса}


Запишем \emph{канонический тензор энергии - импульса системы частицы}:
\begin{gather}
    \underset{prt}{T}^{\mu\nu}
    =
    \sum_{i}
    p\indices{^\mu_i}\frac{dx\indices{^\nu_i}}{dt}\delta\left(\vec{x}-\vec{x}_i(t)\right)
\end{gather}

\paragraph{Симметризованный тензор энергии - импульса электромагнитного поля:}
\begin{gather}
    \underset{emf}{\Theta\indices{^\mu_\nu}}
    =
    F^{\mu\lambda}F\indices{_\nu_\lambda}
    -
    \frac{1}{4}F_{\alpha\beta}F^{\alpha\beta}\delta\indices{^\mu_\nu}
\end{gather}

В матричном виде имеет вид:
\begin{gather*}
    \underset{emf}{\Theta^{\mu\nu}} =
    \left[
        \begin{array}{cccc}
            W   & S_x         & S_y         & S_z         \\
            S_x & \sigma_{xx} & \sigma_{xy} & \sigma_{xz} \\
            S_y & \sigma_{yx} & \sigma_{yy} & \sigma_{yz} \\
            S_z & \sigma_{zx} & \sigma_{zy} & \sigma_{zz} \\
        \end{array}
        \right],
\end{gather*}
где $\vec{S}=\vec{E}\times\vec{B}$ -- вектор Пойнтинга, $W$ -- плотность энергии, $\sigma_{ij}$ -- тензор напряжений.

\begin{theorem}[Теорема Пойнтинга]
    \begin{gather}
        \frac{\partial\varepsilon}{\partial{t}}+\nabla\vec{S}=-\left<\vec{j},\vec{E}\right>,
    \end{gather}
    где $\varepsilon=\frac{1}{2}\left(\vec{E}^2+\vec{B}^2\right)$ -- плотность энергии, $\vec{S}=\vec{E}\times\vec{B}$ -- вектор Пойнтинга.
\end{theorem}

\end{document}