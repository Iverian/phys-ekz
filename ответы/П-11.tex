\documentclass[__main__.tex]{subfiles}

\begin{document}

\qtitle{П}{11}
Симметризованный тензор энергии-импульса электромагнитного поля, вектор Пойнтинга, интенсивность излучения.\\ 

\begin{definition}
	Вектор Пойтинга: $\vec{S} = [\vec{E} \vec{B}]$ 
\end{definition}

Канонический тензор энергии-импульса имеет вид:
\begin{gather*}
\underset{emf}{T}^{\mu\nu} = F^{\mu\lambda}\partial^\nu A_\lambda - \frac{1}{4}F_{\alpha\beta}F^{\alpha\beta}\eta^{\mu\nu}
\end{gather*}
\begin{gather*}
\underset{emf}{T^{\mu\nu}} - F^{\mu\lambda}\partial_\lambda A^\nu = F^{\mu\lambda}(\partial^\nu A_x - \partial_\lambda A^\nu) - \frac{1}{4}F_{\alpha\beta}F^{\alpha\beta}\eta^{\mu\nu} = F^{\mu\lambda}F^\nu_\lambda - \frac{1}{4}F_{\alpha\beta}F^{\alpha\beta}\eta^{\mu\nu}\equiv \underset{emf}{\Theta}^{\mu\nu}
\end{gather*}
$\underset{emf}{\Theta}^{\mu\nu}$ - симметризованый тензор энергии-импульса электромагнитного поля.\\
Выразим компоненты тензора $T^{\mu\nu}$ через напряжённости электрического и магнитного полей.  
$F_{\mu\nu}= 
\begin{bmatrix}
0 & E_x & E_y & E_z\\
-E_x & 0 & -H_z & H_y\\
-E_y & H_z & 0 & -H_x\\
-E_z & -H_y & H_x & 0
\end{bmatrix},
\quad F^{\mu\nu} = 
\begin{bmatrix}
0 & -E_x & -E_y & -E_z\\
E_x & 0 & -H_z & H_y\\
E_y & H_z & 0 & -H_x\\
E_z & -H_y & H_x & 0
\end{bmatrix}
$

\begin{definition}
	Интенсивность излучения\\
	$I(t)=\frac{1}{T}\int^{t+T}_{t} |\vec{S}(t)| dt$
	\end{definition}

\end{document}