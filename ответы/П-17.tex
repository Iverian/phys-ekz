\documentclass[__main__.tex]{subfiles}

\begin{document}

\qtitle{П}{17}
Тензор энергии-импульса системы частиц имеет вид $T^{\alpha\beta}=\sum_{A}p\indices{^\alpha_A}\frac{dx\indices{^\beta_A}}{dt}\delta(\vec{x}-\vec{x}_A)t$, где $A$ индекс частицы, $\delta$ -- дельта-функция Дирака. Убедитесь в справедливости соотношения $\partial_{\beta}T^{\alpha\beta}=\sum_{A}\frac{d\vec{p}_A}{dt}\delta(\vec{x}-\vec{x}_A)$. Интерпретируйте результат. Покажите, что если заряженные частицы находятся в электромагнитном поле, то последнее выражение принимает форму $\partial_{\beta}T\indices{^\alpha^\beta}=F\indices{^\alpha_\beta}j^\beta$, где $j$ -- 4-вектор плотности тока.\\ 

%%

\end{document}