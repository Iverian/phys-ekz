\documentclass[__main__.tex]{subfiles}

\begin{document}

\qtitle{П}{13}
Убедитесь в справедливости соотношения $\partial_{\beta}\left(T^{\alpha\beta}+\Theta^{\alpha\beta}\right)=0$, где $T^{\alpha\beta}$ тензор энергии-импульса системы заряженных частиц и симметризованный тензор энергии-импульса электромагнитного поля. При этом можно воспользоваться $\partial_{\beta}T^{\alpha\beta}=F\indices{^\alpha_\beta}j^\beta$, где $F$ -- тензор Максвелла и $j$ -- 4-вектор плотности тока. Прокомментируйте результат.\\ 

Запишем \emph{симметризованный тензор энергии - импульса для электромагнитного поля}:
\begin{gather}
\underset{emf}{\Theta}^{\mu\nu}
=
F^{\mu\lambda}F\indices{^\nu_\lambda}
-
\frac{1}{4}F_{\alpha\beta}F^{\alpha\beta}\eta^{\mu\nu}
\end{gather}
Тогда рассматриваемое соотношение имеет вид:
\begin{flalign}
\begin{split}
\partial_\nu
\left(
\underset{prt}{T}^{\mu\nu}
+
\underset{emf}{\Theta}^{\mu\nu}
\right)
=&
F\indices{^\mu_\nu}j^{\nu}
+
\partial_{\nu}
\left(
F^{\mu\lambda}F\indices{^\nu_\lambda}
-
\frac{1}{4}F_{\alpha\beta}F^{\alpha\beta}\eta^{\mu\nu}
\right)
=\\
=&
F\indices{^\mu_\nu}j^{\nu}
+
\partial_{\nu}
\left(
F^{\mu\lambda}F\indices{^\nu_\lambda}
\right)
-
\frac{1}{4}\partial^{\mu}
\left(
F_{\alpha\beta}F^{\alpha\beta}
\right)
=\\
=&
F\indices{^\mu_\nu}j^{\nu}
+
F\indices{^\nu_\lambda}\partial_{\nu}F^{\mu\lambda}
+
F^{\mu\lambda}\partial_{\nu}F\indices{^\nu_\lambda}
-
\frac{1}{4}\partial^{\mu}
\left(
F_{\alpha\beta}F^{\alpha\beta}
\right)
=\\
=&
F\indices{^\mu_\nu}j^{\nu}
+
F\indices{^\nu_\lambda}\partial_{\nu}F^{\mu\lambda}
-
F^{\mu\lambda}j_{\lambda}
-
\frac{1}{4}\partial^{\mu}
\left(
F_{\alpha\beta}F^{\alpha\beta}
\right)
=\\
=&
F\indices{^\nu_\lambda}\partial_{\nu}F^{\mu\lambda}
-
\frac{1}{4}\partial^{\mu}
\left(
F_{\alpha\beta}F^{\alpha\beta}
\right)
=
F\indices{^\nu_\lambda}\partial_{\nu}F^{\mu\lambda}
-
\frac{1}{4}F^{\alpha\beta}\partial^{\mu}F_{\alpha\beta}
-
\frac{1}{4}F_{\alpha\beta}\partial^{\mu}F^{\alpha\beta}
=\\
=&
F\indices{^\nu_\lambda}\partial_{\nu}F^{\mu\lambda}
-
\frac{1}{2}F_{\alpha\beta}\partial^{\mu}F^{\alpha\beta}
\end{split}
\llabel{p13:1}
\end{flalign}
Покажем, что второе слагаемое (\lref{p13:1}) равно первому:
\begin{flalign}
\begin{split}
\frac{1}{2}F_{\alpha\beta}\partial^{\mu}F^{\alpha\beta}
=&
\frac{1}{2}F_{\alpha\beta}
\left(
\partial^{\mu\alpha}A^{\beta}-\partial^{\mu\beta}A^{\alpha}
\right)
=
\frac{1}{2}F_{\alpha\beta}
\left(
\partial^{\alpha\mu}A^\beta-\partial^{\alpha\beta}A^\mu
+
\partial^{\alpha\beta}A^\mu-\partial^{\beta\mu}A^\alpha
\right)
=\\
=&
\frac{1}{2}F_{\alpha\beta}
\left(
\partial^{\alpha}F^{\mu\beta}
+
\partial^{\beta}F^{\alpha\mu}
\right)
=
\left.
\frac{1}{2}
\left(
F_{\alpha\beta}\partial^{\alpha}F^{\mu\beta}
+
F_{\beta\alpha}\partial^{\alpha}F^{\beta\mu}
\right)
\right|_{F_{\alpha\beta}=-F_{\beta\alpha}}
=\\
=&
\frac{1}{2}
\left(
F_{\alpha\beta}\partial^{\alpha}F^{\mu\beta}
+
F_{\alpha\beta}\partial^{\alpha}F^{\mu\beta}
\right)
=
F_{\alpha\beta}\partial^{\alpha}F^{\mu\beta}
=\\
=&
F\indices{^\nu_\lambda}\partial_{\nu}F^{\mu\lambda}
\end{split}
\end{flalign}

\end{document}