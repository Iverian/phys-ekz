\documentclass[__main__.tex]{subfiles}

\begin{document}

\qtitle{П}{08}
Покажите, что компоненты тензора энергии-импульса системы заряженных частиц $T\indices{^\mu_\nu}$ в электромагнитном поле удовлетворяет соотношению $\partial^{\nu}T\indices{^\mu_\nu}=F\indices{^\mu_\nu}j^{\nu}$, где $F\indices{^\mu_\nu}$ компоненты тензора Максвелла, $j^\nu$ компоненты 4-вектора плотности тока.\\ 


Запишем \emph{канонический тензор энергии - импульса системы частицы}:
\begin{gather}
\underset{prt}{T}^{\mu\nu}
=
\sum_{i}
p\indices{^\mu_i}\frac{dx\indices{^\nu_i}}{dt}\delta\left(\vec{x}-\vec{x}_i(t)\right)
\end{gather}
Тогда искомое выражение примет вид:
\begin{flalign}
\partial_{\nu}\underset{prt}{T}^{\mu\nu}
=&
\sum_{i}
\partial_{\nu}
\left(
p\indices{^\mu_i}\frac{dx\indices{^\nu_i}}{dt}
\right)
\delta\left(\vec{x}-\vec{x}_i(t)\right)
\llabel{p08:1}
\end{flalign}
Рассмотрим действие электромагнитного поля $F^{\mu\nu}$ на систему частиц, с такой стороны можно записать в сумму плотность силы Лоренца:
\begin{gather}
\mathscr{F}^\mu
=
{q}F\indices{^\mu_\nu}u^\nu,
\llabel{p08:2}
\end{gather}
Тогда перепишем (\lref{p08:1}) при помощи (\lref{p08:2}) и перейдем к рассмотрению системы зарядов как сплошного поля с объемной плотностью тока $\rho$:
\begin{flalign}
\begin{split}
\partial_{\nu}\underset{prt}{T}^{\mu\nu}
=&
\sum_{i}
q_{i}F\indices{^\mu_\nu}u\indices{^\nu_i}\delta\left(\vec{x}-\vec{x}_i(t)\right)
=
F\indices{^\mu_\nu}
\sum_{i}
q_{i}u\indices{^\nu_i}\delta\left(\vec{x}-\vec{x}_i(t)\right)
=\\
=&
F\indices{^\mu_\nu}
\int{d(\rho{u^\nu})}
=\\
=&
F\indices{^\mu_\nu}j^\nu
\end{split}
\end{flalign}

\end{document}