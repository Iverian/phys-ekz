\documentclass[__main__.tex]{subfiles}

\begin{document}

\qtitle{Ч}{02}
Сформулируйте принцип неразличимости частиц одного сорта. Продемонстрируйте работу постулата симметризации на системе из двух тождественных (а) бозонов, (б) фермионов.\\ 

\textbf{Принцип неразличимости частиц одного сорта}
В классической механике всегда есть возможность различить (пронумеровать объекты и сохранить нумерацию в течении всего времени их существования) одинаковые объекты, в то время как квантовомеханическое описание делает невозможным различить одинаковые объекты.\\
Итак, возьмем кв. систему из 2-х микрообъектов. Пространство состояний 1-ого микрообъекта $V$ базис $\{|i>\}$, пространство состояний 2-го микрообъекта $W$ базис $\{|j>\}$, пространство состояний из 2-х микрообъектов $X = V \bigotimes W$ и базис $\{|i> |j>\}$. Далее любой вектор состояния из пространств $V,W$ можно предствить в виде $|V> = \sum_i \nu_i |i>$ или  $|W> = \sum_j \omega_j |j>$ соответственно( где $\nu_i, \omega_j$ - комплексные коэффициенты) эти состояния называют \textit{одночастичными состояниями}, а  $|X> = \sum_{ij} X_{ij} |i> |j>$ называют \textit{двухчастиичными состояниями}.\\
Оператор перестановки, в основе его определения лежит действие на базисные векторы $|i> |j>$. Это означает, что под действием оператора на 2-хчастичное состояние, в котором 1-я частица находилась в состоянии $|i>$, а 2-я в состоянии $|j>$, получается 2-хчастичное состояние, в котором 1-я частица находится в состоянии $|j>$, а 2-я в состоянии $|i>$.\\
\begin{gather}
\hat{P}|i>|j> \equiv |j>|i> \Longrightarrow <i'|<j'|\hat{P}|i>|j> = <i'|<j'||j>|i> = <i'|j><j'|i>\\
<i|<j|\hat{P}^+=<j|<i| \Longrightarrow <i'|<j'|\hat{P}^+=<j'|<i'| \Longrightarrow <i'|<j'|\hat{P}^+|i>|j> = <j'|<i'||i>|j> = <j'|i><i'|j>\\
\Longrightarrow \hat{P}^+=\hat{P}
\end{gather}
где $\hat{P}$ - эрмитов оператор и так же унитарен. Два раза преминив $\hat{P}$ - ничего не произойдет.\\
\begin{gather}
\hat{P}^2 = \hat{1} \Longrightarrow \hat{P} = \hat{P}^{-1} \Longrightarrow \hat{P}^+ = \hat{P}^{-1}
\end{gather}
Обозначим $|\widetilde{x}> = \hat{P}|x>$ для любых физ. величин $A$ состояний системы $|x>$.\\
Не существует наблюдаемых отличий между системой в состоянии $|x>$ и состемой в состоянии $|\widetilde{x}> = \hat{P}|x>$:
\begin{gather}
\llabel{ch02:1}
<x|\hat{A}|x> = <\widetilde{x}|\hat{A}|\widetilde{x}>
\end{gather}
Полученное нами выражение \lref{ch02:1} и есть \textit{математическая формулировка принципа неразличимости одинаковых частиц}.
\begin{definition}
Симметричная вектор функция описывает частицы, называемые бозонами.\\
Бозоны - частицы с целым спином $(s = 0,1,2...)$\\
Антисимметричная вектор функция описывает частицы, называемые фермионами.\\
Фермионы - частицы с полуцелым спином $(s = \frac{1}{2},\frac{3}{2},\frac{5}{2}...)$\\
\end{definition}
\textbf{Постулат симметризации}
\begin{definition}
Состояния системы, содержащей N тождественных частиц, будут все либо симметричными, либо антисимметричными относительно перестановок этих N частиц.\\
\end{definition}
Рассмотрим систему из 2-х одинаковых частиц. Уравнение Шредингера для стационарного состояния этой системы имеет вид:\\
\begin{gather}
[\hat{H}(1)+\hat{H}(2)+\hat{H}_I(1,2)]\Psi(1,2) = \epsilon \Psi(1,2)
\end{gather}
Если частицы не взаимодействуют, то есть $\hat{H}_I(1,2) = 0$, то
\begin{gather}
[\hat{H}(1)+\hat{H}(2)]\Psi(1,2) = \epsilon \Psi(1,2)
\end{gather}
разделим переменные $\phi_i(1)\phi_j(2)$, где 
\begin{gather}
\hat{H}(1)\phi_i(1)=\epsilon_i\phi_i(1) \qquad \hat{H}(2)\phi_j(2)=\epsilon_j\phi_j(2)
\end{gather}
Энергия всей системы равна сумме энергий ее составляющих:
\begin{gather}
[\hat{H}(1)\phi_i(1)]\phi_j(2)+\phi_i(1)[\hat{H}(2)\phi_j(2)]=\epsilon\phi_i(1)\phi_j(2)
\end{gather}
Приходим к $\epsilon_i+\epsilon_j=\epsilon$.\\
Если имеет дело с бозонной системой, то в качестве реений:\\
\begin{gather}
\Psi_{ij}(1,2) = \Psi_{ij}(2,1)
\end{gather}
Поэтому из решений типа $\phi_i(1)\phi_j(2)$ не имеющих отношений к реальной, требуется сконструировать:\\
\begin{gather}
\Psi_{ij}(1,2)=\frac{1}{\sqrt{2}}(\phi_i(1)\phi_j(2)+\phi_i(2)\phi_j(1))
\end{gather}
Заметим, что если оказалось $i=j$, то $\phi_j(1)\phi_j(2)$ нас устраевает.\\
Если же система фермионов, то получаются антисимметричные функции в качестве решений:
\begin{gather}
\Psi_{ij}(1,2) = -\Psi_{ij}(2,1)
\end{gather}
Тогда, чтобы получить решение, соответствующее реальности нам надо альтернировать:\\
\begin{gather}
\Psi_{ij}(1,2)=\frac{1}{\sqrt{2}}(\phi_i(1)\phi_j(2)+\phi_i(2)\phi_j(1)) = \frac{1}{\sqrt{2}} \begin{vmatrix}
\phi_i(1) & \phi_i(2)\\
\phi_j(1) & \phi_j(2)
\end{vmatrix}
\end{gather}
Значит, если положить в последнем выражении $i=j$, то определитель занулится $\Longrightarrow$ система из двух одинаковых фермионов не может находится в состоянии $\Psi$ содержащих одинаковые одночастичные состояния $\phi$ - это мы получили "Принцип запрета Паули".
\end{document}