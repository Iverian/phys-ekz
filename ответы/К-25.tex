\documentclass[__main__.tex]{subfiles}

\begin{document}

\qtitle{К}{25}
Атом водорода. Собственные значения и собственные функции операторов $\hat{L}^2$ и $\hat{L}_z$, кратность вырождения энергетического уровня, отвечающего заданному главному квантовому числу $n$.\\ 

\textbf{Атом водорода}
Из всего многообразия элементов, мы рассмотрим атом водорода, потому что это самый простой элемент, то есть $H$ имеет один протон и один электрон. Так же, существует закономерность, которая позволяет получать огромное количество спектральных линий, комбинируя значительно меньшее число велечин, называемых \textit{термами}. Для атома водорода:\\
\begin{gather}
T_n = \frac{R_H}{n^2}, \qquad n=1,2...
\end{gather}
где $R_H$ - постоянная Ритберга.\\

%%

\end{document}