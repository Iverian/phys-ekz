\documentclass[__main__.tex]{subfiles}

\begin{document}
	
	\qtitle{Ч}{20}
	Электрон находится в состоянии, описываемом кет-вектором $\left|st\right>=\frac{\left|+\right>-\left|-\right>}{\sqrt{2}}$. Найдите направление, при измерении проекции спина на которое результат с определённостью будет оказываться равным $+\frac{\hbar}{2}$.\\
	
	Решение задачи сводится к нахождению компонент $(n_x,n_y,n_z)$ единичного вектора $\vec{n}$, решая матричное уравнение:
	\begin{gather*}
		\begin{pmatrix}\hat{\sigma}\cdot\vec{n}-\hat{1}\end{pmatrix}\begin{pmatrix}\frac{1}{\sqrt{2}}\\ -\frac{1}{\sqrt{2}}\end{pmatrix} = \begin{pmatrix}0\\0\end{pmatrix}
	\end{gather*}
	или если расписать:
	\begin{gather}
		\label{ch-20-1}
		\begin{pmatrix}
			n_z-1 & n_x-in_y\\
			n_x+in_y & -n_z-1
		\end{pmatrix}\begin{pmatrix}\frac{1}{\sqrt{2}}\\ -\frac{1}{\sqrt{2}}\end{pmatrix} = \begin{pmatrix}0\\0\end{pmatrix}
	\end{gather}
	Разрешая матричное уравнение (\ref{ch-20-1}) получим в итоге
	\begin{equation*}
		\begin{cases}
			n_x = -1\\
			n_z = -in_y	
		\end{cases}
	\end{equation*}
	учитывая, что все компоненты вектора $\vec{n}$ - действительные, то $n_y = n_z = 0$\\
	Ответ: $\vec{n} = (-1,0,0)$ 
\end{document}