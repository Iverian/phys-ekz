\documentclass[__main__.tex]{subfiles}

\begin{document}

\qtitle{К}{23}
Постулаты квантовой механики: о квантовых состояниях, о физических величинах, об измерениях, динамический постулат.\\ 

\begin{itemize}
\item
\textit{постулат о квантовых состояниях:} квантовое состояние полностью задается $\psi$ -- функцией из пространства волновых функций. $\psi$ -- функции, отличающиеся только комплексным множителем, задают одно и то же состояние;
\item
\textit{постулат о физических величинах:} каждой физической величине ставится в соответствие эрмитов оператор, обладающий полной системой собственных функций;
\item
\textit{постулат об измерениях:} пусть измеряется некоторая величина $A$ и ей в соответствие поставлен оператор $\hat{A}: A \rightarrow \hat{A}$. Если оператор эрмитов (т.е. $\hat{A}^{+}=\hat{A}$, то его собственные значения вещественны и $\hat{A}\varphi_n=a_n\varphi_n$);
\item
\textit{динамический постулат:} все предсказания, которые могут быть сделаны относительно различных свойств системы в данный момент времени, следуют из значения $\psi$ -- функции в этот момент времени;
\end{itemize}

\end{document}