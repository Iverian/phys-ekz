\documentclass[__main__.tex]{subfiles}
\begin{document}
	
	\qtitle{С}{03}
	Закон Кулона, напряжённость поля, силовые линии электростатического поля, эквипотенциальные поверхности, электростатическая защита.\\ 
	
	Закон Кулона\\
	Для постоянного электрического поля уравнения Максвелла имеют вид:
	$$ div \vec{E} = 4 \pi \rho$$
	$$ rot \vec{E} = 0$$
	Абсолютная величина напряженности поля $E$ будет зависеть только от расстояния $R$ до заряда $e$. Для нахождения этой абсолютной величины применим уравнение $$ div \vec{E} = 4 \pi \rho$$ в интегральной форме $$\oint \vec{E} df = 4 \pi \int \rho dV$$.\\
	Поток электрического поля через шаровую поверхность с радиусом $R$, проведенную вокруг заряда $e$, равен $4\pi R^2 E$; этот поток должен быть равен $4\pi e$. Тогда:\\
	$$E = \frac{e}{R^2}$$
	В векторном виде:
	$$\vec{E} = \frac{e \vec{R}}{R^3}$$
	Это и есть \textbf{закон Кулона} (Поле, создаваемое точечным зарядом, обратно пропорционально квадрату расстояния до этого заряда)
	Напряженность поля.\\
	$\vec{A}$ -- векторный потенциал поля.\\
	Силу первого рода, отнесенную к заряду, равному единице, называют \textbf{напряженностью электрического поля}:\\
	$$\vec{E} = -\frac{1}{c}\frac{\partial \vec{A}}{\partial t}-grad \phi$$
	Для того чтобы описать электрическое поле, нужно задать вектор напряженности в каждой точке поля. Это можно сделать аналитически или графически. Для этого пользуются \textbf{силовыми линиями} – это линии, касательная к которым в любой точке поля совпадает с направлением вектора напряженности $\vec{E}$:\\
	\includegraphics[width=.5\linewidth]{c-03-1}\\
	Для системы зарядов, как видим, силовые линии направлены от положительного заряда к отрицательному:\\
	\includegraphics[width=.9\linewidth]{c-03-2}\\\\
	\textbf{Эквипотенциальные поверхности} - это воображаемые поверхности все точки которой имеют одинаковый потенциал.
	\begin{gather*}
		U(x,y,z) = const
	\end{gather*}
	\textbf{Электростатическая защита.}\\
	Электростатическая защита – защита приборов и оборудования, основанная на том, что напряженность электростатического поля внутри проводника равна нулю.\\
	Потенциал электростатического поля:
	$$\phi = \frac{e}{R}$$
	Работа поля по перемещению заряда из одной точки в другую, называется разностью потенциалов.\\
	
\end{document}