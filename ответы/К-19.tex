\documentclass[__main__.tex]{subfiles}

\begin{document}

\qtitle{К}{19}
Квантовомеханическое среднее и его временная эволюция. Классические уравнения Гамильтона и теорема Эренфеста.\\ 
\textbf{Классические уравнения Гамильтона}\\
В аналитической механике мы пользуемся уравнениями Гамильтона:\\
\begin{gather}
\frac{d}{dt}q_j=\frac{\partial H}{\partial p_j}; \qquad \frac{d}{dt}p_j=-\frac{\partial H}{\partial q_j}
\end{gather}\\
\textbf{Теорема Эренфеста}\\
\begin{theorem}
Квантовомеханическое среднее удовлетворяет тем же уравнениям движения, что и соответствующие классические переменные.\\
\begin{gather}
\frac{d}{dt}<q_j> = \left<\frac{\partial H}{\partial p_j}\right>\\ 
\frac{d}{dt}<p_j> = - \left<\frac{\partial H}{\partial q_j}\right>,
\end{gather}
где $q_j$ -координаты (декартовы), $p_j$ -сопряженные им импульсы и $H(q_j,p_j)$ - гамильтониан системы.
\end{theorem}
\begin{proof}
Согласно общему уравнению, определяющему изменение во времени среднего значения $A$:
\begin{gather}
i\hbar \frac{d}{dt}\left<A\right>=\left<\left[A,H\right]\right>+i\hbar\left<\frac{\partial A}{\partial t} \right> \Longrightarrow \\ i\hbar\frac{d}{dt}\left<q_j\right>=\left<\left[q_j,H\right]\right>, \qquad i\hbar\frac{d}{dt}\left<p_j\right>=\left<\left[p_j,H\right]\right>
\end{gather}
Так как для произвольных функций $q$ и $p$ можно написать:
\begin{gather}
[p_j,A] = -i\hbar \frac{\partial A}{\partial q_j}\\
[q_j,A] = i\hbar \frac{\partial A}{\partial p_j}
\end{gather}
Теперь, после вычисления коммутаторов в правых частях уравнений, получим:
\begin{gather}
\frac{d}{dt}<q_j> = \left<\frac{\partial H}{\partial p_j}\right>\\ 
\frac{d}{dt}<p_j> = - \left<\frac{\partial H}{\partial q_j}\right>,
\end{gather}
\end{proof}
\textbf{Временная эволюция квантовомеханического среднего:}\\
Будем иметь дело только с временной функцией, нормированной на единицу, т.е.
 $\|\psi(t_{0}, \vec{r})\|^{2}=1
 \Rightarrow\langle A \rangle =(\psi, \hat{A}\psi)$

Исследуем как квантомеханическое среднее эволюционирует по времени:

\begin{flalign*}
\begin{split}
\frac{d}{dt} \langle A(t) \rangle
&=
\left.
\frac{d}{dt}(\psi(t,\vec{r}),\hat{A}\psi(t,\vec{r}))=(\partial_{t}\psi,\hat{A}\psi)+(\psi,(\partial_{t}\hat{A})\psi)+(\psi,(\hat{A})\partial_{t}\psi)
\right|_{\partial_{t}\psi=\frac{1}{i\hbar}\hat{H}\psi}
=\\
&=(\psi,(\partial_{t}\hat{A})\psi)+\frac{1}{i\hbar}\hat{H}\psi((\psi , \hat{A}\hat{H}\psi)-(\hat{H}\psi, \hat{A}\psi)).
\end{split}
\end{flalign*}
При этом $(\hat{H}\psi, \hat{A}\psi)=(\psi,\hat{H} \hat{A}\psi)$, не забываем, что $\hat{H}$ -- эрмитов оператор, можем переставлять в скобочках. Также $(\psi,(\hat{A}\hat{H}-\hat{H}\hat{A})\psi)=(\psi,[\hat{A},\hat{H}]\psi).$ $\langle A H \rangle = (\psi,[\hat{A},\hat{H}]\psi).$
Тогда  

\begin{gather}
\llabel{_30:final}
\frac{d}{dt}\langle A(t) \rangle = \langle \partial_{t}\hat{A} \rangle + \frac{1}{i\hbar}\langle [\hat{A},\hat{H}] \rangle
\end{gather}
В (\lref{_30:final}) собственно показана эволюция квантомеханического среднего.
%%

\end{document}