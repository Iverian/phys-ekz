\documentclass[__main__.tex]{subfiles}

\begin{document}

\qtitle{К}{34}
Временное и стационарное уравнения Шрёдингера. Задача об электроне в трёхмерной потенциальной яме с абсолютно непроницаемыми стенками.\\ 

Запишем уравнение Шрёдингера квантовой системы:
\begin{gather}
\hat{H}\Psi=i\hbar\partial_{t}\Psi,
\llabel{k34:sh}
\end{gather}
положим, что гамильтониан ($\hat{H}$) не зависит от времени явно. Это случай консервативных систем, соответствующих классическим системам, для которых энергия есть интеграл движения. Образуем решение $\Psi$, представляющее динамическое состояние с определенной энергией $E$.
Такая волновая функция  $\Psi$ должна обладать определенной круговой частотой, соответствующей соотношению Эйнштейна $E=\hbar\omega$.
\begin{gather}
\Psi=\exp\left(-it\frac{E}{\hbar}\right)\psi(\vec{r}).
\end{gather}
Подставим такое решение в (\lref{k34:sh}) и получим:
\begin{gather}
\hat{H}\psi=E\psi,
\llabel{k34:st}
\end{gather}
уравнение (\lref{k34:st}) называется \textit{стационарным уравнением Шрёдингера}. Тогда для $\hat{H}$ в трехмерии получим:
\begin{gather}
\hat{H}
=
\frac{1}{2m}\hat{p}^2+U(\vec{r})
=
-\frac{\hbar^2}{2m}\Delta+U(\vec{r})
\llabel{k34:ham}
\end{gather}
Рассмотрим кубическую потенциальную яму $\mathbf{K}$ с шириной ребра $l$, где потенциальная энергия $U(\vec{r})$ имеет вид:
\begin{gather*}
    U(\vec{r})
    =
    \begin{cases}
        0      & \vec{r}\in\mathbf{K}     \\
        \infty & \vec{r}\not\in\mathbf{K} \\
    \end{cases},
\end{gather*}
тогда уравнение (\lref{k34:st}) согласно (\lref{k34:ham}) примет вид:
\begin{gather*}
\begin{cases}
\Delta\psi+k^2\psi=0\\
\left.\psi\right|_\mathbf{K}=0
\end{cases}
\qquad
k^2=\frac{2m}{\hbar^2}T
\end{gather*}
Решение этого уравнения примет вид с учетом краевых условий:
\begin{gather*}
\psi(\vec{r})
=
A\sin\frac{\pi{n_1}x}{l}\sin\frac{\pi{n_2}y}{l}\sin\frac{\pi{n_3}z}{l}, 
\end{gather*}
Коэффициент $A$ найдем из условия нормировки $\int\limits_{V}\psi(\vec{r})=1$:
\begin{gather*}
A=\sqrt{\frac{8}{l^3}},
\end{gather*}
подставив волновую функцию в уравнение Шрёдингера и найдем энергию, получим спектр кинетической энергии:
\begin{gather*}
T_{n_1,n_2,n_3}=\frac{\pi^2\hbar^2}{2ml^2}\left(n_1^2+n_2^2+n_3^2\right),
\end{gather*} 

\end{document}