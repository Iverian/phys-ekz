\documentclass[__main__.tex]{subfiles}

\begin{document}

\qtitle{К}{09}
Рассмотрите задачу об электроне в бесконечно глубокой потенциальной яме. Чему равна минимальная кинетическая энергия электрона? Какова вероятность обнаружить электрон в интервале $\frac{L}{6}<x<\frac{L}{3}$ (где $L$ ширина ямы) во втором возбуждённом состоянии?\\ 


Рассмотрим электрон в потенциальной яме, $0<x<l$.
Движение электрона будет задаваться волновой функцией:
\begin{gather}
 \frac{\hbar^{2}}{2m}\frac{d^{2}\psi}{dx^{2}}+(\varepsilon-V_{0})\psi=0,
\end{gather}
где $k=\frac{\sqrt{2m(\varepsilon-V_{0})}}{\hbar}$ ($V=const$)
Решение данного дифференциального уравнения:

\begin{gather}
\psi=A_{21}e^{9kx}+B_{2}e^{-9kx}.
\end{gather}

Граничные условия внутри потенциальной ямы:

$$
v(x)=
\left\{
\begin{gathered}
\psi(0)=0\\
\psi(l)=0\\
\end{gathered}
\right.
$$

И так решение диффура: 

\begin{gather}
\psi_{n}=2A sin{k_{n}x}
\end{gather}
а $k_{n}=\frac{\pi n}{l}.$

Энергия электрона:

\begin{gather}
T_{n}=\frac{\pi^{2}\hbar^{2}}{2m^{2}l^{2}}{}n^{2}.
\end{gather}

Тогда минимальная энергия электрона:

\begin{gather}
T_{min}=T_{n}=\frac{\pi^{2}\hbar^{2}}{2m^{2}l^{2}}.
\end{gather}
Но стоит вспомнить, что в квантовой механике квадрируемая функция нормирована единицей, откуда $A=\frac{1}{\sqrt{2l}}$

\begin{gather}
\psi_{n}=\sqrt{\dfrac{2}{l}} sin{\frac{\pi n x}{l}}.
\end{gather}

Вероятность обнаружения электрон в состоянии $\psi_{2}$ :

\begin{gather}
P_{2}(L/6<x<L/3)=\int_{l/6}^{l/3}dx|\psi_{2}(x)|^{2}.
\end{gather}

\end{document}