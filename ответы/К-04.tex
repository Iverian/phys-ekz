\documentclass[__main__.tex]{subfiles}

\begin{document}

\qtitle{К}{04}
Фотоэффект. Эффект Комптона. Интерференция электрона на двух щелях.\\


Фотоэффект возникает при взаимодействии вещества с поглощаемым электромагнитным излучением.\\

Различают внешний и внутренний фотоэффект.

\begin{definition}
    Внешним фотоэффектом называется явление вырывания электронов из вещества под действием падающего на него света.
\end{definition}

\begin{definition}
    Внутренним фотоэффектом называется явление увеличения концентрации носителей заряда в веществе, а следовательно, и увеличения электропроводности вещества под действием света. Частным случаем внутреннего фотоэффекта является вентильный фотоэффект — явление возникновения под действием света электродвижущей силы в контакте двух различных полупроводников или полупроводника и металла.
\end{definition}

Законы фотоэффекта:

\begin{enumerate}
    \item Число фотоэлектронов, вырываемых за 1 секунду с поверхности катода, пропорционально интенсивности света, падающего на это вещество.
    \item Кинетическая энергия фотоэлектронов не зависит от интенсивности падающего света, а зависит линейно от его частоты.
    \item Красная граница фотоэффекта зависит только от рода вещества катода.
    \item Фотоэффект практически безинерционен, так как с момента облучения металла светом до вылета электронов проходит время $\approx 10^{-9}$
\end{enumerate}

\begin{definition}
    Эффект Комптона – рассеяние электромагнитного излучения на свободном электроне, сопровождающееся уменьшением частоты излучения. В этом процессе электромагнитное излучение ведёт себя как поток отдельных частиц – корпускул (которыми в данном случае являются кванты электромагнитного поля - фотоны), что доказывает двойственную – корпускулярно-волновую природу электромагнитного излучения. С точки зрения классической электродинамики рассеяние излучения с изменением частоты невозможно.
\end{definition}

\begin{definition}
    Комптоновское рассеяние – это рассеяние на свободном электроне отдельного фотона с энергией $E = h\nu = h\frac{c}{\lambda}$ ($h$ – постоянная Планка, $ν$ – частота электромагнитной волны, $λ$ – её длина, $с$ – скорость света) и импульсом $р = \frac{E}{c}$.
\end{definition}

Рассеиваясь на покоящемся электроне, фотон передаёт ему часть своей энергии и импульса и меняет направление своего движения. Электрон в результате рассеяния начинает двигаться. Фотон после рассеяния будет иметь энергию $Е' = hν'$ (и частоту) меньшую, чем его энергия (и частота) до рассеяния. Соответственно, после рассеяния длина волны фотона $λ'$ увеличится. Из законов сохранения энергии и импульса следует, что длина волны фотона после рассеяния увеличится на величину $\Delta \lambda = \lambda' - \lambda = \frac{h}{m_e с}(1-\cos\theta)$, где $θ$ – угол рассеяния фотона, а $m_e$ – масса электрона $\frac{h}{m_e c} = 0.024 Å$ называется комптоновской длиной волны электрона.\\

Изменение длины волны при комптоновском рассеянии не зависит от $λ$ и определяется лишь углом $θ$ рассеяния $γ$-кванта. Кинетическая энергия электрона определяется соотношением
$$E_e = \frac{E_γ}{1+\frac{m_e c^2}{2E_γ \sin^2\frac{\theta}{2}}}$$
Эффективное сечение рассеяния γ-кванта на электроне не зависит от характеристик вещества поглотителя. Эффективное сечение этого же процесса,$ рассчитанное на один атом$, пропорционально атомному номеру (или числу электронов в атоме) Z.
Сечение комптоновского рассеяния убывает с ростом энергии $γ$-кванта: $\sigma_k \approx \frac{1}{E_γ}$.
\\\\\\
\textbf{Обратный Комптон-эффект}

Если электрон, на котором рассеивается фотон, является ультрарелятивистским $E_e \gg E_γ$, то при таком столкновении электрон теряет энергию, а фотон приобретает энергию. Такой процесс рассеяния используется для получения моноэнергетических пучков $γ$-квантов высокой энергии. С этой целью поток фотонов от лазера рассеивают на большие углы на пучке ускоренных электронов высокой энергии, выведенных из ускорителя.\\

Энергия рассеянного фотона $E_γ$ зависит от скорости $V$ ускоренного пучка электронов, энергии $E_{γ_0}$ и угла столкновения $θ$ фотонов лазерного излучения с пучком электронов, угла между $φ$ направлениями движения первичного и рассеянного фотона

$$
    E_γ = E_{γ_0}\frac{1-\frac{V}{c}\cos\theta}{1-\frac{V}{c}\cos(\theta-\phi)+\frac{E_{γ_0}}{E_0}(1-\cos\phi)}
$$

При «лобовом» столкновении
$$
    E_{γmax} = E_0 \frac{4E_{γ0} E_0}{4E_{γ0} E_0 + (mc^2)^2}
$$

$E_{0}$ − полная энергия электрона до взаимодействия, $ mc^2$ − энергия покоя электрона.
Если направление скоростей начальных фотонов изотропно, то средняя энергия рассеянных фотонов  $\langle E_γ \rangle$  определяется соотношением

$$
    \langle E_γ \rangle = (\frac{4E_γ}{3})*(\frac{E_e}{mc^2})
$$

При рассеянии релятивистских электронов на микроволновом реликтовом излучении образуется изотропное рентгеновское космическое излучение с энергией
$E_γ$ = 50–100 кэВ.\\\\
Законы теплового излучения.\\
\begin{enumerate}
    \item
          Закон Кирхгофа.\\
          Пусть $r_{\omega}$ --- испускающая способность. Тогда энергия, которую площадка $ds$ испускает за единицу времени на интервале частот $[\omega; \omega+d\omega]$ равна $r_\omega d\omega ds$. \\
          Поглощающая способность $a_\omega$ определяется как отношение поглощенной площадкой $ds$ на интервале $[\omega;\omega+d\omega]$ энергии к падающей на эту же площадку на том же интервале энергии ($E_{\text{пад}}$).\\
          Тогда в состоянии равновесия:
          $$
              a_\omega \cdot E_{\text{пад}} = r_\omega d\omega ds \Longrightarrow a_\omega\cdot \frac{1}{4}cu_\omega(\tau)d\omega ds = r_\omega d\omega ds,
          $$
          где $cu_\omega(\tau)d\omega$ --- плотность потока энергии в частотном интервале $[\omega; \omega+d\omega]$.\\
          Из этого и следует закон Кирхгофа:
          $$
              \frac{1}{4}cu_\omega(\tau)=\frac{r_\omega}{a_\omega}
          $$
    \item
          Закон смещения Вина. \\
          Из термодинамических соображений:
          $$
              u_\omega(\tau)=\omega^3f\left(\frac{\omega}{\tau}\right)
          $$
          $$
              f\left(\frac{\omega}{\tau}\right)\longrightarrow 0 \text{  при  } \omega \longrightarrow \infty
          $$
          Пусть $x\equiv \frac{\omega}{\tau}$. Найдем частоту $\omega_m$, доставляющую $max(u_\omega (\tau)$.
          $$
              3\omega_m^2f(x_m)+\frac{\omega^3}{\tau}f'(x_m)=0 \Longrightarrow x_m=C,
          $$
          где $C$ --- число, не зависящее от $\tau$ и $\omega$.\\
          Следовательно,
          $$
              \lambda_m\tau=\sigma=2.90\cdot 10^{-3}\text{м}\cdot K
          $$
    \item
          Закон Стефана-Больцмана.\\
          Энергетическая светимость АЧТ
          $$
              R^*=\int\limits^{\infty}\limits_0 r^*_\omega d\omega \sim \int\limits^{\infty}\limits_0 u_\omega(\tau)d\omega =\tau^4 \int\limits^{\infty}\limits_0 \frac{\omega^3}{\tau^3}f\left(\frac{\omega}{\tau}\right)d\frac{\omega}{\tau}
          $$
          $$
              R^*=\sigma \tau^4
          $$
          $$
              \sigma = 5.67\cdot 10^{-8} \frac{\text{Вт}}{\text{м}^2\cdot \text{К}^4}
          $$
\end{enumerate}

\end{document}