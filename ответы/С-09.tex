\documentclass[__main__.tex]{subfiles}

\begin{document}

\qtitle{С}{09}
Оценить среднюю объёмную плотность электрических зарядов в атмосфере, если известно, что напряженность электрического поля на поверхности Земли составляет примерно $130\text{В/м}$, а на высоте $1\text{км}$ -- примерно $40\text{В/м}$\\ 

Так как высота $h=1$км, что очень мало, для того чтобы считать напряженность поля сферы, мы будем в наших масштабах считать атмосферу Земли плоской. Теперь все очень просто, мы берем прямоугольный параллелепипед высотой $h$, тогда по теореме Гаусса(обозначим $E_0 = 130\text{В/м}$, а $E_h = 40\text{В/м}$):\\
\begin{gather}
\Phi = \oint_S EdS = [E_0-E_h]S = 4\pi q 
\end{gather}

Так же замечу, что силовые линии мы берем направлеными к Земле, в ином случае, мы получим отрицательную величину $\rho$.После чего выражая $q$ через объемную плотность параллелепипеда и объем параллелепипеда получаем:\\
\begin{gather}
E_0 - E_h = 4\pi \rho h \Longrightarrow \rho = \frac{E_0 - E_h}{4\pi h} \Longrightarrow \rho = \frac{130 - 40}{4 \cdot \pi \cdot 1}\cdot10^{-3} = 7,17\cdot10^{-3} \frac{\text{Кл}}{\text{м}^3}
\end{gather}

Следовательно, средняя объёмная плотность электрических зарядов в атмосфере равна $7,17\cdot10^{-3} \frac{\text{Кл}}{\text{м}^3}$.
\end{document}