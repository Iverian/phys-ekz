\documentclass[__main__.tex]{subfiles}

\begin{document}

\qtitle{Ч}{03}
Докажите, что собственными значениями эрмитова оператора, квадрат которого равен тождественному оператору, могут быть только $\pm 1$. Покажите, что оператор перестановки частиц удовлетворяет вышеперечисленным требованиям.\\ 

\begin{proof}
\begin{gather*}
\hat{A}|j\rangle = a_j|j\rangle\\
\hat{1}|j\rangle =\hat{A}^2|j\rangle = a_j\hat{A}|j\rangle = (a_j)^2|j\rangle \Longrightarrow (a_j)^2=1 \Longrightarrow a_j=\pm 1
\end{gather*}
\end{proof}


Рассмотрим систему из $N$ невзаимодействующих частиц, обладающих спином. Волновая функция такой системы имеет вид:
\begin{gather*}
\Psi(\vec{r_1}s_{z1}; \vec{r_2}s_{z2};...; \vec{r_N}s_{zN}) = \Psi(q_1, q_2, ..., q_N) \qquad \text{(Просто ввели небольшое переобозначение)}
\end{gather*}
Введем оператор $\hat{P}$ перестановки двух частиц местами:
\begin{gather*}
\hat{P}\Psi(q_1, q_2, ..., q_N)=\Psi(q_2, q_1, ..., q_N)
\end{gather*}
С другой стороны, по определению оператора:
\begin{gather*}
\hat{P}\Psi(q_1, q_2, ..., q_N)=\lambda \Psi(q_1, q_2, ..., q_N)
\end{gather*}
Если мы подействуем дважды оператором $\hat{P}$ на волновую функцию, то получим:
\begin{gather*}
\hat{P}^2\Psi(q_1, ..., q_N)=\Psi (q_1, ..., q_N) \Longrightarrow \hat{P}^2=\hat{1}
\end{gather*}
При этом:
\begin{gather*}
\hat{P}^2\Psi(q_1, q_2, ..., q_N)=\lambda^2 \Psi(q_1, q_2, ..., q_N)
\end{gather*}
Объединяя вышенаписанное, получаем, что $\lambda^2=1 \Longrightarrow \lambda=\pm 1$.
\end{document}