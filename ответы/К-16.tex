\documentclass[__main__.tex]{subfiles}

\begin{document}

\qtitle{К}{16}
Временная эволюция классической величины и временная эволюция квантовомеханического среднего. Интеграл движения в классической и квантовой механике.\\

\textbf{Временная эволюция классической величины: }В классической механике при отыскании первых интегралов движения важное значение имеют скобки Пуассона. Дело в том, что выражение полной производной по времени от некоторой функции канонических переменных и времени имеет вид:
\begin{gather}
\llabel{_16:1}
\frac{dA(q_j,p_j,t)}{dt} = \frac{\partial A(q_j,p_j,t)}{\partial t}+\{A,H\}\\
\{A,H\} = \frac{\partial A}{\partial q_j}\frac{\partial H}{\partial p_j} - \frac{\partial H}{\partial q_j} \frac{\partial A}{\partial p_j},
\end{gather}
где $q_j,p_j$ - канонические переменные, которыми в классической механике описывается состояние системы, $H(q_j,p_j,t)$ - функция Гамильтона, $A(q_j,p_j,t)$ - оператор физических величин, не зависящий от времени $A'(t) = A(0) = A$, $\{A,H\}$ - скобки Пуассона.\\
Уравнение (\lref{_16:1}) выражает эволюцию классической системы во времени.

\textbf{Временная эволюция квантовомеханического среднего:} Будем иметь дело только с временной функцией, нормированной на единицу, т.е.
 $\|\psi(t_{0}, \vec{r})\|^{2}=1
 \Rightarrow\langle A \rangle =(\psi, \hat{A}\psi)$

Исследуем как квантомеханическое среднее эволюционирует по времени:

\begin{flalign*}
\begin{split}
\frac{d}{dt} \langle A(t) \rangle
&=
\left.
\frac{d}{dt}(\psi(t,\vec{r}),\hat{A}\psi(t,\vec{r}))=(\partial_{t}\psi,\hat{A}\psi)+(\psi,(\partial_{t}\hat{A})\psi)+(\psi,(\hat{A})\partial_{t}\psi)
\right|_{\partial_{t}\psi=\frac{1}{i\hbar}\hat{H}\psi}
=\\
&=(\psi,(\partial_{t}\hat{A})\psi)+\frac{1}{i\hbar}\hat{H}\psi((\psi , \hat{A}\hat{H}\psi)-(\hat{H}\psi, \hat{A}\psi)).
\end{split}
\end{flalign*}
При этом $(\hat{H}\psi, \hat{A}\psi)=(\psi,\hat{H} \hat{A}\psi)$, не забываем, что $\hat{H}$ -- эрмитов оператор, можем переставлять в скобочках. Также $(\psi,(\hat{A}\hat{H}-\hat{H}\hat{A})\psi)=(\psi,[\hat{A},\hat{H}]\psi).$ $\langle A H \rangle = (\psi,[\hat{A},\hat{H}]\psi).$
Тогда  

\begin{gather}
\llabel{_30:final}
\frac{d}{dt}\langle A(t) \rangle = \langle \partial_{t}\hat{A} \rangle + \frac{1}{i\hbar}\langle [\hat{A},\hat{H}] \rangle
\end{gather}
В (\lref{_30:final}) собственно показана эволюция квантомеханического среднего.

\textbf{Интеграл движения в классической механике:}
Интеграл движения - функция обобщенных координат и импульсов системы, зависящая в общем случае явно от времени, которая при движении системы не меняется (иначе говоря, функция, постоянная на решениях уравнений Гамильтона). То есть, функция $A = A(q(t),p(t),t)$ является интегралом движения, если $\frac{dA}{dt} = 0$.Условием того, что A - интеграл движения, будет следующее выражение:
\begin{gather*}
\frac{\partial A}{\partial t} + \{A,H\} = 0
\end{gather*}

\textbf{Интеграл движения в квантовой механике:} если нет явной зависимости физической величины $A$ от времени, то:
$$
\frac{d}{dt}\langle A(t) \rangle =\frac{1}{i\hbar}\langle [\hat{A},\hat{H}] \rangle
$$
Если теперь выясняется, что $ [\hat{A},\hat{H}]=0$, то интеграл движения представляется в виде:
\begin{gather*}
\frac{d}{dt}\langle A(t) \rangle=0.
\end{gather*}
Интеграл движения в квантовой механике - это такая величина, среднее значение которой остается неизменным при любом начальном состоянии системы.

\end{document}