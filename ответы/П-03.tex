\documentclass[__main__.tex]{subfiles}

\begin{document}

\qtitle{П}{03}
Покажите, что канонический тензор энергии-импульса электромагнитного поля $T^{\mu\nu}=F^{\mu\lambda}\partial^{\nu}A_{\lambda}-\frac{1}{4}F_{\alpha\beta}F^{\alpha\beta}\eta^{\mu\nu}$ не является калибровочно инвариантным в отличие от симметризованного тензора энергии-импульса электромагнитного поля $\Theta^{\mu\nu}=T^{\mu\nu}-F^{\mu\lambda}\partial_{\lambda}A^{\nu}$.\\ 

\emph{Калибровочные преобразования} 4-вектора потенциала поля $A_\mu$ имеют вид $'A_\mu=A_\mu+\partial_{\mu}f$.

Сначала покажем \emph{калибровочную инвариантность тензора Максвелла}:
\begin{gather}
{'F}_{\mu\nu}
=
\partial_\mu{A'_\nu}-\partial_\nu{A'_\mu}
=
\partial_\mu{A_\nu}+\partial_{\mu\nu}{f}-\partial_\nu{A_\mu}-\partial_{\nu\mu}{f}
=
F_{\mu\nu}
\end{gather}

Тогда для канонического тензора энергии - импульса получим:
\begin{flalign}
{'T}^{\mu\nu}
=&
F^{\mu\lambda}\partial^{\nu}\left(A_{\lambda}+\partial_{\lambda}f\right)
-
\frac{1}{4}F_{\alpha\beta}F^{\alpha\beta}\eta^{\mu\nu}
=\\
=&
F^{\mu\lambda}\partial^{\nu}A_{\lambda}
-
\frac{1}{4}F_{\alpha\beta}F^{\alpha\beta}\eta^{\mu\nu}
+
F^{\mu\lambda}\partial\indices{^\nu_\lambda}{f}
=\\
=&
T^{\mu\nu}+F^{\mu\lambda}\partial\indices{^\nu_\lambda}{f},
\end{flalign}
т.е. канонический тензор энергии - импульса не является \emph{канонически инвариантным}.

Для симметризованного тензора энергии - импульса имеем:
\begin{flalign}
\begin{split}
{'\Theta}^{\mu\nu}
=&
{'T}^{\mu\nu}-F^{\mu\lambda}\partial_{\lambda}(A^\nu+\partial^{\nu}f)
=\\
=&
T^{\mu\nu}+F^{\mu\lambda}\partial\indices{^\nu_\lambda}{f}
-
F^{\mu\lambda}\partial_{\lambda}A^\nu-F^{\mu\lambda}\partial\indices{^\nu_\lambda}f
=\\
=&
T^{\mu\nu}-F^{\mu\lambda}\partial_{\lambda}A^\nu
=\\
=&
\Theta^{\mu\nu}
\end{split}
\end{flalign}
\end{document}