\documentclass[__main__.tex]{subfiles}

\begin{document}

\qtitle{Ч}{07}
Факторизованные и запутанные состояния. Проекторная техника. Неравенства Белла.\\ 

\textbf{Факторизованные и запутанные состояния}

Рассмотрим спиновое пространство состояний системы из двух электронов. Так как электроны являются фермионами с s = $\frac{1}{2}$, то число возможных проекций на выделенное направление равно $2s + 1 = 2$. Получаем, что размерность спинового пространства каждого электрона равна 2, т. е все пространство можно описать с помощью двух базисных векторов $|+\rangle_1$, $|-\rangle_1$\\

$|st\rangle_1 = {}^1 C_{+} |+\rangle_1 + {}^1 C_{-} |-\rangle_1$ - состояние 1-го электрона. ${}^1 C_{+}$ и ${}^1 C_{-}$ являются комплексными константами. Единицы указывают на то, что речь идет о 1-м электроне. \\

Аналогично $|stt\rangle_2 = {}^2 C_{+} |+\rangle_2 + {}^2 C_{-} |-\rangle_2$ - cостояние 2-го электрона. Дополнительная буква t подчеркивает (это идея Никифорова, не моя), что $|st\rangle_1$ и $|stt\rangle_2$ не совпадают.\\

Состояние каждого отдельного электрона можно описать с помощью двух параметров (4 - 1 - 1 = 2 (4 параметра исходно из-за того, что параметры комплексные, 2 параметра отбрасываются из-за нормировки и выбора фазы). \\

Общее состояние составной системы $|Gst\rangle$ в некоторых случаях можно представить в форме \textbf{факторизованного состояния} $|fst\rangle = |st\rangle_1 |stt\rangle_2$ (в виде тензорного произведения (знак опущен) состояний каждого элемента системы) и может быть описано с помощью 4 параметров:\\

$$|Gst\rangle = {}^1C_{+}{}^2C_{+}|++\rangle + {}^1C_{+}{}^2C_{-}|+-\rangle + {}^1C_{-}{}^2C_{+}|-+\rangle + {}^1C_{-}{}^2C_{-}|--\rangle$$.

Общее состояние системы $|Gst\rangle$ описывается как суперпозиция базисных векторов $\lbrace |++\rangle, |+-\rangle, |-+\rangle, |--\rangle \rbrace$:

$$|Gst\rangle = C_{++}|++\rangle + C_{+-}|+-\rangle + C_{-+}|-+\rangle + C_{--}|--\rangle$$.

и описывается $2 \cdot 4 - 1 - 1 = 6$ параметрами (вычитаем единицу два раза из-за нормировки, 2 - так как константы комплексные, 4 - так как констант всего 4) . Получаем, что не все состояния системы можно представить в виде произведения состояний системы. Это приводит к наличию \textbf{запутанных cостояний}.\\

\textit{Пример факторизованного состояния}

$$|Fst\rangle = \alpha |+-\rangle$$

Нормировка: $1 = \langle Fst | Fst \rangle  = | \alpha |^2 {}_1 \langle + + | + + \rangle_1 {}_2 \langle - - | - - \rangle_2 \Rightarrow \alpha = e^{i\beta}$

Можно выбрать любую фазу, например $\beta = 0$: $|Fst\rangle = |+-\rangle$

\textit{Пример запутанного состояния}

$$|Est\rangle = \alpha \left(|+-\rangle - |-+\rangle\right) \; \Rightarrow \; \langle Est | = \left(\langle + - | - \langle - + |\right) \alpha^{*}$$

Условие нормировки: $1 = \langle Est | Est \rangle  = | \alpha |^2 \left( \langle + - | + - \rangle - \langle + - | - + \rangle - \langle - + | + - \rangle + \langle - + | - + \rangle \right) \; \Rightarrow \; \alpha = \frac{e^{i\beta}}{\sqrt{2}}$.

Так как можно выбрать любую фазу (напр. $\beta = 0$), то: $|Est\rangle = \frac{|+-\rangle - |-+\rangle}{\sqrt{2}}$

Вероятность в результате измерения обнаружить систему в состоянии $|+-\rangle$ составляет $\frac{1}{2}$, с той же вероятностью можно обнаружить систему в состоянии $|-+\rangle$. При этом, в то время как состояния единой системы полностью определено, ни одна из подсистем не находится в определенном состоянии.\\

Допустим, обнаруживается, что 1-й электрон находится в состоянии $|+\rangle_1$, тогда понятно, что 2-й находится в состоянии $|-\rangle_2$. Когда проводится измерение, запутанное состояние:

$$|Est\rangle = \frac{|+-\rangle - |-+\rangle}{\sqrt{2}}$$

коллапсирует в состояние $|Pst\rangle = |+-\rangle$. То есть при измерении, проводимом над 1-м электроном, определяется состояние 2-го электрона, причем \textit{мгновенно}, это называется \textbf{квантовой нелокальностью}.

\textbf{Проекторная техника}

\begin{definition}
Оператор $\hat{P}$ называется проектором, если он эрмитов ($\hat{P}^+ = \hat{P}$) и идемпотентен ($\hat{P}^2 = \hat{P}$).
\end{definition}

\begin{definition}
Оператор $\hat{P}_w$ называется проектором на подпространство $W$, если $\hat{P}_w |v\rangle = |w\rangle$.
\end{definition}

Пусть $|u\rangle = |w\rangle + |\varepsilon\rangle$, где $|u\rangle \in V$, $|w\rangle \in W$, $|\varepsilon\rangle \in \Sigma$, тогда:

$$\langle u | \hat{P}_w | v \rangle = \langle u | w \rangle = \langle w | w \rangle$$,
$$\langle v | \hat{P}_w | u \rangle^{*} = \langle v | w \rangle^{*} = \langle w | w \rangle$$

Так как $\langle u | \hat{P}_w | v \rangle = \langle v | \hat{P}_w | u \rangle^{*}$, то $\hat{P}^{+}_w = \hat{P}_w$

$$\hat{P}^2_w | v \rangle = \hat{P}_w \left(\hat{P}_w | v \rangle \right) = \hat{P}_w | w \rangle = |w\rangle = \hat{P}_w | v \rangle \; \Rightarrow \; \hat{P}^2_w = \hat{P}_w$$

Следовательно, оператор $\hat{P}_w$ является эрмитовым и идемпотентным, а значит удовлетворяет первому определению. 

\textbf{Неравенство Белла}

\end{document}
