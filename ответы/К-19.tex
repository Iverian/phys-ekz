\documentclass[__main__.tex]{subfiles}

\begin{document}

\qtitle{К}{19}
Квантовомеханическое среднее и его временная эволюция. Классические уравнения Гамильтона и теорема Эренфеста.\\ 

По собственным функциям наблюдаемого оператора Вы можете разложить любую функцию из Гильбертова пространства, описывающую текущее состояние квантовой системы:

\begin{gather*}
	\psi (x) = \sum_nc_n\varphi_n(x)
\end{gather*}

Рассмотрим следующие выражения:
\begin{enumerate}
	\item $(\psi,\hat A \psi) = \left(\sum_mc_m\varphi_m,\sum_nc_na_n\varphi_n\right) = \sum_{mn}c_m^*c_na_n(\varphi_m,\varphi_n) = \sum_{mn}c_m^*c_na_n\delta_{mn} = \sum_m|c_m|^2a_m $
	\item $(\psi,\psi) = \left(\sum_mc_m\varphi_m,\sum_nc_n\varphi_n\right) = \sum_{mn}c_m^*c_n(\varphi_m,\varphi_n) = \sum_{mn}c_m^*c_n\delta_{mn} = \sum_m|c_m|^2 $
\end{enumerate}

Квантовомеханическое среднее значение наблюдаемой A находится по формуле:
\begin{gather*}
	<A> = \sum_jP_{a_j}a_j = \frac{\sum_j|c_j|^2a_j}{\sum_p|c_p|^2} = \frac{(\psi,\hat{A}\psi)}{(\psi,\psi)}
\end{gather*}
Исследуем, как кв.мех. среднее эволюционирует во времени. Возьмем такую вектор функцию, которая нормирована на единицу,т.е
\begin{gather*}
	\vert\vert \psi(t_0,\vec{r})\vert\vert ^2 = 1 \Rightarrow\;  <A> = \frac{(\psi,\hat{A}\psi)}{(\psi,\psi)} = \frac{(\psi,\hat{A}\psi)}{\vert\vert\psi\vert\vert^2} = (\psi,\hat{A}\psi)
\end{gather*}
Теперь посмотрим как она эволюционирует во времени:
\begin{gather*}
	\frac{d}{dt}<A(t)> = \frac{d}{dt}\left(\psi(t,\vec{r}), \hat{A} \psi(t,\vec{r})\right) = \left(\frac{d}{dt}\psi,\hat{A}\psi\right)+\left(\psi,\frac{d}{dt}(\hat{A})\psi\right)+\left(\psi,\hat{A}\frac{d}{dt}\psi\right) = 
\end{gather*}

Вспомним, что  $\frac{d}{dt}\psi = \frac{1}{ih}\widehat{H}\psi$
\begin{gather*}
	= \left(\psi,\frac{d}{dt}(\hat{A})\psi\right)+\frac{1}{ih}\left[(\psi,\widehat{A}\widehat{H}\psi) - (\widehat{H}\psi,\widehat{A}\psi)\right] = 
\end{gather*}
Так как $\widehat{H}$ - эрмитов, то его можно переносить из ячейки в ячейку
\begin{gather*}
	= \left(\psi,\frac{d}{dt}(\hat{A})\psi\right)+\frac{1}{ih}\left[(\psi,\widehat{A}\widehat{H}\psi) - (\psi,\widehat{H}\widehat{A}\psi)\right] = \left(\psi,\frac{d}{dt}(\hat{A})\psi\right)+\frac{1}{ih}\left[(\psi,(\hat{A}\hat{H}-\hat{H}\hat{A})\psi)\right] =\\ = \left(\psi,\frac{d}{dt}(\hat{A})\psi\right)+\frac{1}{ih}(\psi,[\hat{A},\hat{H}]\psi)
\end{gather*}
Таким образом получаем:
\begin{gather*}
	\frac{d}{dt}<A(t)> = <\frac{d}{dt}\hat{A}>+\frac{1}{ih}<[\hat{A},\hat{H}]>
\end{gather*}
В случае если нет явной зависимости от времени, тогда
\begin{gather*}
	\frac{d}{dt}<A>=\frac{1}{ih}<[\hat{A},\hat{H}]>
\end{gather*}
Вдобавок заметим, если $[\hat{A},\hat{H}] = 0$ , то $\frac{d}{dt}<A> = 0 \Rightarrow \;\;<A>$ - интеграл движения

\begin{definition}
		\llabel{21-def-fullenergy}
		Величина
		\begin{gather*}
		E = \sum_{\alpha=1}^{s}\frac{\partial{L}}{\partial{\dot{q}_\alpha}}\dot{q}_\alpha - L
		\end{gather*}
		называется обобщённой энергией
	\end{definition}
	
	Основной величиной, определяющей механические свойства систем в формализме
	Лагранжа, является функция Лагранжа $L(q,\dot{q},t)$. В рамках самой классической механики
	эта функция не имеет непосредственного физического смысла. Для решения ряда задач
	классической механики, а также при формулировке перехода к квантовой теории удобно
	работать с величинами, более тесно связанными с механическими свойствами систем. Оказывается, что уравнения движения механики можно представить в виде, в котором роль
	основной величины, определяющей механические свойства системы, играет обобщенная
	энергия системы, а в качестве независимых переменных используются обобщенные координаты и обобщенные импульсы системы. Математически такой переход осуществляется
	с помощью так называемого преобразования Лежандра
	\begin{gather}
	\llabel{21-fulldiffLagr}
	dL(q,\dot{q},t) = \sum_{\alpha=1}^{s}\frac{\partial{L}}{\partial{q_\alpha}}	
	\end{gather}
	Величина $\frac{\partial{L}}{\partial{\dot{q}_\alpha}}$ есть, по определению, обобщенный импульс $p_{q_\alpha}$, соответствующий обобщенной координате $q_α$. Для краткости, обозначение pqα будет сокращаться ниже до $p_α$. С этим обозначением, а также с помощью уравнений Лагранжа равенство (5.1) можно
	переписать так:
	\begin{gather}
	\llabel{21-fulldifflagr}
	dL(q,\dot{q},t) = \sum_{\alpha=1}^{s}\dot{p}_\alpha dq_{\alpha} + \sum_{\alpha=1}^{s}p_\alpha d\dot{q}_\alpha + \frac{\partial{L}}{\partial{t}}.
	\end{gather}
	Правая часть уравнения \lref{21-fulldifflagr} содержит дифференциалы независимых переменных $q, \dot{q} и t$. Для того чтобы перейти от этого набора к новому набору независимых переменных $q, p, t$, напишем тождественно
	
	\begin{gather*}
	p_\alpha d\dot{q}_\alpha = d(p_\alpha \dot{q}_\alpha) - \dot{q}_\alpha dp_\alpha,\space \alpha = 1,...,s
	\end{gather*}
	и представим уравнение $\lref{21-fulldifflagr}$ в виде:
	\begin{gather}
	\llabel{21-fulldifflagr1}
	d\big(\sum_{\alpha=1}^{s}p_\alpha \dot{q}_\alpha - L(q, \dot{q}, t)\big) = -\sum_{\alpha=1}^{s}\dot{p}_\alpha dq_\alpha + \sum_{\alpha=1}^{s}\dot{q}_\alpha dp_\alpha - \frac{\partial{L}}{\partial{t}}dt.
	\end{gather}
	Тот факт, что правая часть этого тождества содержит дифференциалы переменных $q, p, t$ означает, что величина, стоящая в его левой части под знаком полного дифференциала, также может быть выражена как функция этого набора переменных. В соответствии с определением $\lref{21-def-fullenergy}$, эта величина численно совпадает с обобщенной энергией системы. Выраженная через обобщенные координаты и импульсы (и время), она называется функцией Гамильтона системы и обозначается через $H(q, p, t)$. Таким образом, по определению, при построении функции Гамильтона переменные $q, p$ рассматриваются как независимые переменные, аналогично тому, как в функции Лагранжа независимыми являются переменные $q, \dot{q}$. Для того чтобы получить эту функцию, следует разрешить определение $p=\frac{\partial{L}}{\partial{\dot{q}}}$ относительно $\dot{q}$ и подставить результат в функцию $E(q,\dot{q},t)$:
	\begin{gather*}
	H(q,p,\dot{q}) = \big(\sum_{\alpha=1}^{s}p_\alpha\dot{q}_\alpha - L(q,\dot{q},t)\big)\big|_{\dot{q}=\dot{q}(p,q)}
	\end{gather*}
	Расписав явно полный дифференциал функции $H(q, p, t)$ в левой части $\lref{21-fulldifflagr1}$, получим
	\begin{gather}
	\llabel{21-fulldiffH}
	\sum_{\alpha=1}^{s}\frac{\partial{H}}{\partial{q_\alpha}}dq_\alpha + \sum_{\alpha=1}^{s}\frac{\partial{H}}{\partial{p_\alpha}}dp_\alpha + \frac{\partial{H}}{\partial{t}}dt = -\sum_{\alpha=1}^{s}\dot{p}_\alpha dq_\alpha + \sum_{\alpha=1}^{s}\dot{q}_\alpha dp_\alpha - \frac{\partial{L}}{\partial{t}}dt.
	\end{gather}
	Приравнивая коэффициенты при дифференциалах независимых переменных в этом тож-
	дестве, находим следующие уравнения
	\begin{gather}
	\llabel{21-hamiltequat1}
	\dot{p}_\alpha = -\frac{\partial{H}}{\partial{q_\alpha}}, \space \alpha = 1,...,s,
	\end{gather}
	\begin{gather}
	\llabel{21-hamilequat2}
	\dot{q}_\alpha = \frac{\partial{H}}{\partial{p_\alpha}}, \space \alpha = 1,...,s,
	\end{gather}
	\begin{gather}
	\llabel{21-hamilequat3}
	\frac{\partial{H}}{\partial{t}} = -\frac{\partial{L}}{\partial{t}}.
	\end{gather}
	Уравнения $\lref{21-fulldifflagr1}. \lref{21-hamilequat2}$ представляют собой систему $2s$ дифференциальных уравнений первого порядка для $2s$ функций $q_α(t)$, $p_α(t), α = 1, ..., s$. Эти уравнения называются уравнениями
	Гамильтона или каноническими уравнениями.\\
	
	Теорема Эренфенста\\
	
	\begin{definition}
		\llabel{21-def-wave}
		Волновой пакет (цуг волн)— определённая совокупность волн, обладающих разными частотами, которые описывают обладающую волновыми свойствами формацию, в общем случае ограниченную во времени и пространстве. Так, в квантовой механике описание частицы в виде волновых пакетов (совокупности солитонов) способствовало принятию статистической интерпретации квадрата модуля волновой функции
	\end{definition}
	
	Если потенциальная энергия пренебрежимо мало изменяется в области нахождения пакета, то движения волнового пакета будет аналогично движению соответсвующей классической частицы. И действительно, если под векторами "координаты" и "импульса" пакета понимать средние значения этих величин, то можно показать, что  классическое и квантовое движения всегда согласуются друг с другом. Компонентой "скорости" пакеа будет производная по времени от среднего значения соответствующей компоненты радиус-вектора; поскольку $<x>$ зависит только от времени, $а x$ в подынтегральном выражении 
	\begin{gather*}
	<\vec{r}>	= \int{rP(\vec{r},t)}d\tau = \int{\bar{\psi}(\vec{r},t)d\tau}
	\end{gather*}
	представляет собой переменную область интегрирования, то соотвествующая компонента "скорости" равна
	\begin{gather*}
	\frac{d}{dt}<x> = \frac{d}{dt}\int{\bar{\psi}x\psi}d\tau = \int{\bar{\psi}x\frac{\partial{\psi}}{\partial{t}}d\tau} + \int{\frac{\partial{\bar{\psi}}}{\partial{t}}x\psi d\tau}
	\end{gather*}
	Это выражения можно упростить, подставляя выражения производных по времени от $\psi$ и $\bar{\psi}$ из волнового уравнения Шрёдингера
	\begin{gather*}
	ih\frac{\partial{\psi}}{\partial{t}} = -\frac{\hbar^2}{2m}\nabla^2\psi + V(\vec{r},t)\psi
	\end{gather*}
	Члены, содержащие $V$, при этом взаимно уничтожаются:
	\begin{gather*}
	\frac{d}{dt}<x> = -\frac{i}{\hbar}\big[\int{\bar{\psi}x(-\frac{\hbar^2}{2m}\nabla^2\psi + V\psi)d\tau} - \int{(-\frac{\hbar^2}{2m}\nabla^2\bar{\psi} + V\psi)x\psi d\tau}\big] = \frac{ih}{2m}\int{[\bar{\psi}x(\nabla^2\psi)-(\nabla^2\bar{\psi})x\psi]d\tau} 
	\end{gather*}
	Второе слагаемое можно проинтегрировать по частям:
	\begin{gather*}
	\int{(\nabla^2\bar{\psi})x\psi d\tau} = -\int{(grad \bar{\psi})grad(x\psi)d\tau} + \int_{A}^{}{(x\psi grad \bar{\psi})_ndA}
	\end{gather*}
	Поскольку на больших расстояниях функция $\psi$, характеризующая волновой пакет, обращается в нуль, то равен нулю и интеграл от составляющей вектора $x\psi grad \bar{\psi}$ по нормали к элементу бесконечно удаленной граничной поверхности $A$. Вторично интегрируя по частям(и вновь замечая, что поверхностный интеграл равен нулю), получаем
	\begin{gather*}
	\int{(\nabla^2\bar{\psi})x\psi d\tau} = \int{\bar{\psi}\nabla^2(x\psi)d\tau}
	\end{gather*} 
	Таким образом,
	\begin{gather}
	\llabel{21-eren}
	\frac{d}{dt}<x> = \frac{ih}{2m}\int{\psi[x\nabla^2\psi - \nabla^2(x\psi)]d\tau} = -\frac{iH}{m}\int{\bar{\psi}\frac{\partial{\psi}}{\partial{x}}d\tau} = \frac{1}{m}<p_x>
	\end{gather}
	Поскольку в силу
	\begin{gather*}
	<\vec{r}>	= \int{rP(\vec{r},t)}d\tau = \int{\bar{\psi}(\vec{r},t)d\tau}
	\end{gather*}
	величина $<x>$ всегда вещественна, из соотношения $\lref{21-eren}$ вытекает, между прочим, что вещественно и значение $<p_x>$. Это можно показать также с помощью  формулы $<\vec{p}> = \int{\bar{\psi}(-ih)grad\psi d\tau}$, если произвести в ней интегрирование по частям и принять во внимание, что функция $\psi$ описывает волновой пакет.\\
	Подобным же образом можно вычислить производную по времени от компоненты "импульса" частицы. Снова пользуясь волновым уравнением и дважды интегрируя по частям, получим
	\begin{gather}
\llabel{21-eren2}
\frac{d}{dt}<p_x> = -ih\frac{d}{dt}\int{\bar{\psi}\frac{\partial{\psi}}{\partial{x}}d\tau} = -ih\big(\int{\bar{\psi}\frac{\partial}{\partial{x}}\frac{\partial{\psi}}{\partial{t}}d\tau} + \int{\frac{\partial{\bar{\psi}}}{\partial{t}}\frac{\partial{\psi}}{\partial{x}}d\tau}\big) =\\ -\int{\bar{\psi}\frac{\partial}{\partial{x}}\big(-\frac{\hbar^2}{2m}\nabla^2\psi + V\psi\big)d\tau} + \int{\big(-\frac{\hbar^2}{2m}\nabla^2\bar{\psi} + V\bar{\psi}\big)\frac{\partial{\psi}}{\partial{x}}d\tau} = -\int{\bar{\psi}\big[\frac{\partial}{\partial{x}}(V\psi)-V\frac{\partial{\psi}}{\partial{x}}\big]d\tau} =\\ -\int{\bar{\psi}\frac{\partial{V}}{\partial{x}\psi}d\tau} = <-\frac{\partial{V}}{\partial{x}}>
	\end{gather}
	Уравнения $\lref{21-eren} и \lref{21-eren2}$(вместе с уравнениями для других компонент) аналогичны классическим уравнениям движения
	\begin{gather*}
	\frac{d\tau}{dt}= \frac{\vec{p}}{m},\space \frac{dp}{dt} = -grad V.
	\end{gather*}	
%%

\end{document}