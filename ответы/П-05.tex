\documentclass[__main__.tex]{subfiles}

\begin{document}

\qtitle{П}{05}
Воспользуйтесь полевыми уравнениями Эйлера-Лагранжа чтобы получить уравнения Максвелла <<с источниками>>.\\ 

	Выражение для действия (функция $\mathcal{L}$ в данном случае - плотность лагранжиана ): 
\begin{gather}
\label{P5-1}
A=\int \mathcal{L}dt=\int \mathcal{L}( A^{\alpha},\partial_\beta  A^{\alpha})d^4x
\end{gather}

\begin{gather}
\label{P5-2}
0=\delta A=
\int d^4x\left(
\frac{\partial \mathcal{L}}{\partial A^\alpha}\delta A^\alpha+\frac{\partial  \mathcal{L}}{\partial(\partial_\beta A^\alpha)}\delta(\partial_\beta A^\alpha)\right)=
\int d^4x(\frac{\partial \mathcal{L}}{\partial  A^\alpha}\delta A^\alpha-\partial\beta\frac{\partial  \mathcal{L}}{\partial(\partial_\beta A^\alpha)}\delta(\partial_\beta A^\alpha)) 
\end{gather}
Принимая во внимание произвольность, переходим к полевому уравнению Эйлера-Лагранжа: \\
\begin{gather}
\label{P5-3}
\frac{\partial \mathcal{L}}{\partial  A^\alpha}-\partial_\beta\frac{\partial  \mathcal{L}}{\partial(\partial_\beta A^\alpha)}=0
\end{gather}
Расмотрим подробнее плотность лагранжиана: $ \mathcal{L} = \mathcal{L}_{emf}+ \mathcal{L}_{int}.$\\
\begin{gather}
\label{P5-4}
\mathcal{L}_{emf}=-\frac{1}{4}F\cdot\cdot F=-\frac{1}{4}F_{\alpha\beta}F^{\alpha\beta}=\frac{1}{4}F_{\alpha\beta}F^{\beta\alpha}\\
\label{P5-5}
\mathcal{L}_{int}=- A\jmath=-\jmath_\alpha A^\alpha
\end{gather}
$\jmath$-плотность тока. Из пунктов \ref{P5-4} и \ref{P5-5} получаем, что:
\begin{gather}
\label{P5-6}
\mathcal{L}=\frac{1}{4}F_{\alpha\beta}F^{\beta\alpha}-\jmath_\alpha A^\alpha
\end{gather}
Подсчитаем $\frac{\partial \mathcal{L}}{\partial  A^\alpha}$ и $\partial\beta\frac{\partial  \mathcal{L}}{\partial(\partial_\beta\beta A^\alpha)}$ и подставим их в уравнение Эйлера-Лагранжа.
\begin{gather}
\label{P5-7}
\frac{\partial \mathcal{L}}{\partial A^\alpha}=
\frac{\partial \mathcal{L}_{int}}{\partial  A^\alpha}=
-j_\nu\delta\indices{^\nu_\alpha} = -j_\alpha\\
\label{P5-8}
\frac{\partial  \mathcal{L}}{\partial(\partial_\beta A^\alpha)}=\frac{\partial  \mathcal{L}_{emf}}{\partial(\partial_\beta A^\alpha)}=
\frac{\partial\frac{1}{4}F_{\mu\nu}F^{\mu\nu}}{\partial(\partial_\beta A^\alpha)} =
\frac{\partial\frac{1}{4}F_{\mu\nu}}{\partial(\partial_\beta A^\alpha)}F^{\mu\nu}+F_{\mu\nu}\frac{\partial\frac{1}{4}F^{\mu\nu}}{\partial(\partial_\beta A^\alpha)}
\end{gather}
Заметим, что 
$F_{\mu\nu}\frac{\partial\frac{1}{4}F^{\mu\nu}}{\partial(\partial_\beta A^\alpha)}=\frac{\partial\frac{1}{4}F_{\mu\nu}}{\partial(\partial_\beta A^\alpha)}F^{\mu\nu}$, тогда $\frac{\partial\frac{1}{4}F_{\mu\nu}F^{\nu\mu}}{\partial(\partial_\beta A^\alpha)}=2F_{\mu\nu}\frac{\partial\frac{1}{4}F^{\mu\nu}}{\partial(\partial_\beta A^\alpha)}$
. Вернемся к \ref{P5-8}:\\
\begin{gather}
\label{P5-9}
\frac{1}{2}F_{\mu\nu}\frac{\partial^\mu A^\nu-\partial^\nu A^\mu}{{\partial(\partial_\beta A^\alpha)}}=
\frac{1}{2}F\indices{^{\mu}_\nu}\frac{\partial_\mu A^\nu}{{\partial(\partial_\beta A^\alpha)}}-\frac{1}{2}F\indices{^{\nu}_\mu}\frac{\partial_\nu A^\mu}{{\partial(\partial_\beta A^\alpha)}}=
\frac{1}{2}\tensor{F}{^{\mu}_\nu}\tensor{\delta}{^{\beta}_\mu}\tensor{\delta}{^{\nu}_\alpha}-\frac{1}{2}\tensor{F}{^{\nu}_\mu}\tensor{\delta}{^{\beta}_\nu}\tensor{\delta}{^{\mu}_\alpha}
\end{gather}
Итак, получаем:
$\frac{\partial \mathcal{L}}{\partial(\partial_\beta\Alpha^\alpha)}=
\frac{1}{2}(F^\beta_{\ \ \alpha}-F_\alpha^{\ \beta})$
\\
Подставляем \ref{P5-7} предыдущее равенство в \ref{PS-3} и получаем уравнения " с источниками":\\
\begin{gather}
\label{P5-10}
-j-\frac{1}{2}\partial_\beta(\frac{1}{2}F^\beta_\alpha-F_\alpha^\beta)=0\\
\label{P5-11}
\partial^\beta F_{\beta\alpha}=-j_\alpha
\end{gather}
Рассмотрим подробнее систему \ref{P5-11}:\\
\begin{gather}
\label{P5-12}
\begin{cases}
\partial^0F_{00}+\partial^1F_{10}+\partial^2F_{20}+\partial^3F_{30}=-\jmath_0\\
\partial^0F_{01}+\partial^1F_{11}+\partial^2F_{21}+\partial^3F_{31}=-\jmath_1\\
\partial^0F_{02}+\partial^1F_{12}+\partial^2F_{22}+\partial^3F_{32}=-\jmath_2\\
\partial^0F_{03}+\partial^1F_{13}+\partial^2F_{23}+\partial^3F_{33}=-\jmath_3
\end{cases}
\begin{cases}
\partial_xE_x+\partial_yE_y+\partial_zE_z=\rho\\
-\partial_t(-E_x)-\partial_yB_z+\partial_zB_y=-\jmath_x\\
-\partial_t(-E_y)+\partial_xB_z+\partial_zB_x=-\jmath_y\\
-\partial_t(-E_z)-\partial_xB_y+\partial_yB_x=-\jmath_z
\end{cases}
\end{gather}
Окончательный вид уравнений <<с источниками>>:\\
\begin{gather}
\label{P5-13}
\begin{cases}
\nabla\bar{E}=\rho\\
\nabla\times\bar{B}=\partial_t\bar{E}+\bar{j}
\end{cases}
\end{gather}
где $\rho$ - плотность стороннего электрического заряда, а $\bar{j}$ - плотность электрического тока.

\end{document}