\documentclass[__main__.tex]{subfiles}

\begin{document}

\qtitle{К}{12}
Исходя из обобщённого соотношения неопределённостей Хайзенберга, получите соотношение неопределённостей <<время-энергия>>, приведите примеры использования, дайте интерпретацию результата.\\ 

\textit{Обобщенное соотношение неопределенностей Хайзенберга:}
$$\Delta A \Delta B \ge \left| \Big \langle \frac { \lbrack \widehat{A}, \widehat{B} \rbrack}{2} \Big \rangle \right| $$
В обобщенное соотношение вместо $\widehat{B}$ подставим гамильтониан:
$$\Delta A \Delta E \ge \left| \Big \langle \frac { \lbrack \widehat{A}, \widehat{H} \rbrack}{2} \Big \rangle \right|$$

Если $\widehat{A}$ не зависит от времени, то $\lbrack \widehat{A}, \widehat{H} \rbrack$ полностью определяет скорость эволюции во времени
квантовомеханического среднего $A$:\\

$$\frac{d}{dt}\langle A \rangle = \frac{1}{2 \hbar} \langle \lbrack \widehat{A}, \widehat{H} \rbrack \rangle \Rightarrow \left| \langle \widehat{A}, \widehat{H}\rangle \right| = \hbar \left| \frac{d}{dt}\langle A \rangle \right|$$

Тогда: $\frac{\Delta A}{\left| \frac{d}{dt} \langle A \rangle \right| } \Delta E \ge \frac{\hbar}{2}$

где $\frac{\Delta A}{\left| \frac{d}{dt} \langle A \rangle \right| } = \Sigma_A $ - характеристическое время эволюции статистического распределения А. Это время, необходимое для того, чтобы центр $\langle A \rangle$ распределения сместился на ширину распределения $\Delta A$. \\

\textbf{Примеры использования:}

В ряде случаев в квантовой механике оказывается невозможным одновременно охарактеризовать частицу ее положением в пространстве (координатами) и скоростью (или импульсом). Так например электрон не может иметь одновременно определенных точных значений координаты $x$ и вектора импульса $p_{x}$. Неопределенности значений  $x$ и $p_{x}$ имеют вид:
\begin{gather}
	\llabel{_28:neop}
	\Delta x\Delta p_{x}\geq h
\end{gather}
Из \lref{_28:neop} следует, что чем меньше неопределенность одной величины ($x$ или  $p_x$), тем больше неопределенность другой.
Соотношение, аналогичное \lref{_28:neop}, имеет место для y и  $p_{y}$  , для z и  $p_{z}$  , а также для других пар величин (в классической механике такие пары называются канонически сопряженными). Обозначив канонически сопряженные величины буквами A и B, можно записать:

\begin{gather}
	\llabel{_28:heiz}
	\Delta A\Delta B\geq h
\end{gather}

Утверждение \lref{_28:heiz} о том, что произведение неопределенностей значений двух сопряженных переменных не может быть по порядку меньше постоянной Планка h, \textbf{называется соотношением неопределенностей Гейзенберга(Хайзенберга)}.

\textbf{Пример использования 1:}

\textit{Энергия и время} являются канонически сопряженными величинами, поэтому для них верно:

\begin{gather}
	\llabel{mynigga}
	\Delta E\Delta t\geq h
\end{gather}

Это соотношение означает, что определение энергии с точностью   должно занять интервал времени, равный, по меньшей мере,

$$
\Delta t \sim \frac{h}{\Delta E}
$$

\textbf{Интерпретация:} энергия $E$ есть динамическая переменная системы, а время $t$ есть параметр. Соотношение \lref{mynigga} связывает неопределённость $\Delta E$ значения, принимаемого этой динамической переменной с интервалом времени $\Delta t$, характеристическим для временной эволюции системы.

\textbf{Пример использования 2 :}

Соотношение неопределенностей указывает, в какой мере возможно пользоваться понятиями классической механики применительно к микрочастицам, в частности с какой степенью точности можно говорить о траекториях микрочастиц. Движение по траектории характеризуется вполне определенными значениями координат и скорости в каждый момент времени. Подставив в \lref{_28:neop} вместо $p_{x}$  произведение m$v_x$  , получим соотношение:

\begin{gather}
	\Delta x\Delta u_{x}\geq \frac{h}{m}
\end{gather}

Из этого соотношения следует, что чем больше масса частицы, тем меньше неопределенности ее координаты и скорости, следовательно тем с большей точностью можно применять к этой частице понятие траектории. 
%%

\end{document}