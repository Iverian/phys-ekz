\documentclass[__main__.tex]{subfiles}

\begin{document}

\qtitle{П}{16}
Воспользуйтесь связью полей $\vec{E}$ и $\vec{B}$ с 4-векторным потенциалом, чтобы убедиться, что тензорная форма уравнений Максвелла <<с источниками>> (в вакууме) $\partial^{\alpha}F_{\alpha\gamma}=-j_{\gamma}$ эквивалентна уравнениям $\nabla\cdot\vec{E}=\rho$ и $\nabla\times\vec{B}=\partial_{t}\vec{E}+\vec{j}$. Запишите два последних уравнения в интегральной форме. Как следует изменить уравнения, чтобы они стали справедливы для материальных сред?\\

Выпишем тензор Максвелла:
\begin{gather*}
    F_{\alpha\beta}=
    \begin{pmatrix}
        0    & E_x  & E_y  & E_z  \\
        -E_x & 0    & B_z  & -B_y \\
        -E_y & -B_z & 0    & B_x  \\
        -E_z & B_y  & -B_x & 0
    \end{pmatrix}
\end{gather*}
А теперь просто распишем все подробно ( далее запись $F_{12,3}$ означает диффиренцирование компоненты 12 по 3 переменной ( 1-я компонента -- время )):
\begin{gather*}
    F_{11,1}+F_{21,2}+F_{31,3}+F_{41,4} = -j_1 \Rightarrow \partial_x E_x+\partial_y E_y+\partial_z E_z = \rho,\\
    F_{12,1}+F_{22,2}+F_{32,3}+F_{42,4} = -j_1 \Rightarrow -\partial_t(-E_x)-\partial_y B_z+\partial_z B_y = -j_x,\\
    F_{13,1}+F_{23,2}+F_{33,3}+F_{43,4} = -j_1 \Rightarrow -\partial_t(-E_y)+\partial_x B_z-\partial_z B_x = -j_y,\\
    F_{14,1}+F_{24,2}+F_{34,3}+F_{44,4} = -j_1 \Rightarrow -\partial_t(-E_z)-\partial_x B_y+\partial_y B_x = -j_z.
\end{gather*}
Первое ур-е дает нам $\nabla\cdot\vec E = \rho$, а последующие три - $\nabla\times\vec B = \partial_{t}\vec E + \vec j$. Запишем в интегральном виде:
\begin{gather*}
    \oint\limits_S E\cdot dS = Q,\\ где\;Q - электрический\; заряд, \;заключённый \;в\; объёме \;v,\; ограниченном\; поверхностью \;s\\
    \oint\limits_l B\cdot dl = I + \frac{d}{dt}\int\limits_S E\cdot dS,\\ где\; I -  электрический\; ток,\; проходящий\; через\; поверхность\; s.
\end{gather*}
Звучат они так:
\begin{itemize}
    \item
          Поток электрической индукции через замкнутую поверхность $S$ пропорционален величине свободного заряда, находящегося в объёме $V$, который окружает поверхность $S$.
    \item
          Полный электрический ток свободных зарядов и изменение потока электрической индукции через незамкнутую поверхность $S$ пропорциональны циркуляции магнитного поля на замкнутом контуре $l$, который является границей поверхности $S$.
\end{itemize}
Для того, чтобы ур-я стали справедливы для материальных сред, нужно вместо $\vec{E}$ поставить $\vec{D}=\epsilon_0\epsilon\vec E = \epsilon_0(1+\kappa)\vec E,$ где $\epsilon$ - диэлектрическая проницаемость диэлектрика, а $\vec B$ на $\vec H = \frac{1}{\mu_0\mu}\vec B$, где $\mu$ - магнитная проницаемость.

\end{document}