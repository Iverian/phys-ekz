\documentclass[__main__.tex]{subfiles}

\begin{document}

\qtitle{К}{05}
Ультрафиолетовая катастрофа. Формула Рэлея-Джинса. Формула Планка.\\ 

\begin{definition}
	Формула Рэлея-Джинса - это определение равновесной плотности излучения $u_{\omega}(T)$, исходя из теоремы классической статистики о равнораспределении энергии по степеням свободы. Равновесное излучение в плотности представляет собой систему стоячих волн.	
\end{definition}
\begin{flalign}
	k_BT\frac{dN_{\omega, \omega+d\omega}}{V} = k_BT\frac{\omega^2d\omega}{c^3\pi^2}\\
	\Longrightarrow u_\omega(T) = \frac{\omega^2k_BT}{c^4 \pi^2}
\end{flalign}
Итак, вспомним, что плотность внутренней энергии:
$$ u = \frac{\tau}{2}nkT,$$
где $\tau$ - степени свободы. Рассмотрим полость - прямоугольный параллелепипед (форму берем для простоты, а от материала из которого собран параллепипед -ничего не зависит). Стороны параллепипеда - $a,b,c$. Должны образововаться стоячие волны. Возникают только такие $\lambda_m$, что:\\
\begin{flalign}
	\frac{\lambda}{2}n_1=a,
	\frac{\lambda}{2}n_2=b,
	\frac{\lambda}{2}n_3=c.
\end{flalign}
\begin{flalign}
	K_{x,n_1} = \frac{\pi n_1}{a},
	K_{x,n_2} = \frac{\pi n_2}{b},
	K_{x,n_3} = \frac{\pi n_3}{c},
\end{flalign}
Возможны только:\\
\begin{flalign}
	\vec{K}_{n_1,n_2,n_3} = \left\{\frac{\pi}{a}n_1;\frac{\pi}{b}n_2;\frac{\pi}{c}n_3\right\}
\end{flalign}
И соответствующие части $\omega = cK$. Различные стоячие волны, соответствующие различным $K$ наз. \textit{модами}. В интервале от $K$ до $|\vec{K}|+d|\vec{K}|$. Берем в расчете на ед. объема.\\
\begin{flalign}
	\frac{dN_{K,K+dK}}{V} = \frac{K^2dK}{\pi^2}
\end{flalign}
Это уже с учетом, что возникают две независимые поляризации.\\
В интервале $\omega, \omega+d\omega$:\\
\begin{flalign}
	\frac{dN_{\omega, \omega+d\omega}}{V} = \frac{\omega^2d\omega}{c^3\pi^2}
\end{flalign}
Каждая стоячая волна (мода) несет некоторую энергию. Анализ с позиции классической физики показывает, что эти моды и нужно считать степенями свободы и положить, что в тепловом равновесии $k_B$ на каждую моду приходится $k_BT$ (Для электромагнитных волн тоже самое). Тогда в интервале частот $\omega, \omega+d\omega$ будет заключена плотность энергии:\\
\begin{flalign}
	k_BT\frac{dN_{\omega, \omega+d\omega}}{V} = k_BT\frac{\omega^2d\omega}{c^3\pi^2}\\
	\Longrightarrow u_\omega(T) = \frac{\omega^2k_BT}{c^3 \pi^2}
\end{flalign}
У нас есть соотношение Вина $u_\omega(T) = \omega^3f(\frac{\omega}{T})$. И учитывая это соотношение выведенную формулу можно записать:\\
\begin{flalign}
	\llabel{_25:2}
	u_\omega(T) = \frac{\omega^2k_BT}{c^3 \pi^2} \Longrightarrow u_\omega(T) =\omega^3 \frac{k_BT}{c^3 \pi^2} \frac{T}{\omega}
\end{flalign}
Зная связь испускательной способности абсолютно чёрного тела с равновесной плотностью энергии теплового излучения $r_\omega ^* = \frac{c}{4}u_\omega(T)$, получаем:\\
\begin{flalign}
	r_\omega ^*(T) = \frac{\omega^2k_BT}{4 c^2 \pi^2} 
\end{flalign}
\textbf{Ультрафиолетовая катастрофа}\\
Интегрирование (\lref{_25:2}) по $\omega$ в пределах от 0 до $\infty$ для равновесной плотности энергии $u (T)$  дает бесконечно большое значение. Этот результат, получивший название ультрафиолетовой катастрофы, входит в противоречие с экспериментом: равновесие между излучением и излучающим телом должно устанавливаться при конечных значениях $u ( T )$ . Несогласие с экспериментом вызвано некими закономерностями, которые несовместимы с классической физикой.\\
\begin{flalign}
	 u(T) = \int_{0}^{\infty}u_\omega(T)d\omega = \infty
\end{flalign}
\textbf{Формула Планка}\\
\begin{definition}
	Формула Планка — выражение для спектральной плотности мощности излучения (спектральной плотности энергетической светимости) абсолютно чёрного тела, которое было получено Максом Планком для плотности энергии излучения $u(\omega,T)$ 
\end{definition}
\begin{flalign}
	u(\omega,T)=\frac{\omega^2}{\pi^2 \cdot  c^3}\cdot \frac{\hbar\omega}{e^\frac{\hbar\omega}{kT}-1}
\end{flalign}
\end{document}