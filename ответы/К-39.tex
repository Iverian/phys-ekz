\documentclass[__main__.tex]{subfiles}
\begin{document}

\qtitle{К}{39}
Рассмотрите стационарное уравнение Шрёдингера для электрона в сферически симметричном поле. Реализуйте процедуру разделения переменных. Охарактеризуйте свойства радиальной составляющей волновой функции $R_{nl}(r)$.\\

В сферических координатах стационарное уравнение Шредингера для частицы в центральном потенциале $U(r)$ имеет вид
\begin{gather}
    \llabel{k39-1}
    -\frac{\hbar^2}{2m}\left[\frac{1}{r^2}\frac{\partial}{\partial r}\left(r^2\frac{\partial \psi}{\partial r}\right)+\frac{1}{r^2sin\theta}\frac{\partial}{\partial\theta}\left(sin\theta\frac{\partial \psi}{\partial \theta}\right)+\frac{1}{r^2sin^2\theta}\frac{\partial^2\psi}{\partial \varphi^2}\right]+U(r)\psi = \varepsilon\psi
\end{gather}
Решение уравнения (\lref{k39-1}) записывается в виде произведения радиальной и угловой функций
\begin{gather*}
    \psi(r,\theta,\varphi)=R_{nl}(r)Y_{lm}(\theta,\varphi)
\end{gather*}
где радиальная функция $R_{nl}(r)$ и угловая функция $Y_{lm}(\theta,\varphi)$, называемая сферической, удовлетворяют уравнениям
\begin{gather*}
    \hat{L}^2Y_{lm}(\theta,\varphi) = \hbar l(l+1)Y_{lm}(\theta,\varphi)
\end{gather*}
или
\begin{gather}
    \llabel{k39-2}
    -\frac{\hbar^2}{2m}\frac{d^2}{dr^2}\left[rR_{nl}(r)\right]+\left[U(r)+\frac{\hbar^2 l(l+1)}{2mr^2}\right]\left[r R_{nl}(r)\right] = \varepsilon\left[rR_{nl}(r)\right]
\end{gather}
Уравнение (\lref{k39-2}) определяет собственные значения энергии $E$ и радиальные собственные функции $R_{lm}(r)$, от которых зависит энергия системы.\\
Решения уравнения (\lref{k39-1}) существуют лишь при определенных значениях квантовых чисел
\begin{itemize}
    \item $n$ (радиальное квантовое число)
    \item $l$ (орбитальное квантовое число)
    \item $m$ (магнитное квантовое число)
\end{itemize}
Возможные энергетические состояния системы (уровни энергии) определяются числами $n$ и $l$ и в случае сферически симметричных состояний не зависят от квантового числа $m$. Число $n$ может быть только целым: $n=1,2,\cdots,\infty$. Число $l$ может принимать значения $0,1,2,3,\cdots,\infty$

\end{document}