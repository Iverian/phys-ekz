\documentclass[__main__.tex]{subfiles}

\begin{document}

\qtitle{Э}{06}
Переменный синусоидальный ток частоты $\omega$ течёт по обмотке прямого соленоида, радиус сечения которого $R$. Найдите отношение амплитудных значений электрической и магнитной энергий внутри соленоида.\\ 

Магнитная энергия вычисляется по формуле:
\begin{gather*}
W_M = \int \frac{B^2}{2\mu_0}dV, \qquad B = \mu_0nI
\end{gather*}
Вычисляем интеграл и получаем:
\begin{gather*}
W_M = \frac{(\mu_0nI)^2}{2\mu_0}\cdot (\pi R^2h)=\dfrac{1}{2}\mu_0 n^2I^2\pi R^2 h 
\end{gather*}
Запишем формулу для вычисления электрической энергии:
\begin{gather*}
W_{\text{ЭЛ}}= \int\frac{\varepsilon_0E^2}{2}dV
\end{gather*}
Рассмотрим циркуляцию вектора $E$ по кругу радиуса $r<R$, параллельному боковой стороне соленоида:
\begin{gather*}
\int Edl = \frac{d}{dt}\int BdS \Longrightarrow 2\pi r E = \frac{d}{dt}\left(\mu_0nI\pi r^2\right)
\end{gather*} 
Пусть ток $I$ меняется по закону:
\begin{gather*}
I=I_0\sin{(\omega t)}
\end{gather*}
Тогда:
\begin{gather*}
E = \frac{\mu_0n\pi r^2 I_0 \omega \cos{(\omega t)}}{2\pi r} = \frac{1}{2}\mu_0nr\omega I_0 \cos{(\omega t)}
\end{gather*}
*примечание: в решебнике в этом месте вылезает минус -- не ебу, откуда. в принципе он на решение не влияет, но если тут ошибка, то тот кто будет править - присмотритесь пожалуйста:3\\

Далее нам не помешает понять, а что же такое $dV$. Если мы вспомним, как выглядит соленоид, то поймем, что это будет такое цилиндрическое кольцо толщиной $dr$. Тогда:
\begin{gather*}
dV=\pi r^2h - \pi (r+dr)^2h=2\pi h r dr - \pi h (dr)^2 \approx 2\pi h r dr
\end{gather*}
Подставляем все полученное в интеграл выше и считаем:
\begin{gather*}
W_{\text{ЭЛ}}= \int\limits_0^R \frac{\varepsilon_0}{2}\left( \frac{1}{2}\mu_0nr\omega I_0 \cos{(\omega t)}\right)^2\cdot 2\pi h r dr = \frac{1}{16}R^4\varepsilon_0\mu_0^2n^2I_0^2\cos^2(\omega t)\omega^2\pi h
\end{gather*}
Да, выглядит хреново, но сейчас все поправим, ведь нам нужно лишь отношение максимальных значений этих энергий:
\begin{gather*}
\frac{W_{\text{ЭЛ}^{max}}}{W_M^{max}} = \frac{\frac{1}{16}R^4\varepsilon_0\mu_0^2n^2I_0^2\omega^2\pi h}{\dfrac{1}{2}\mu_0 n^2I_0^2\pi R^2 h } = \frac{1}{8}R^2\omega^2\varepsilon_0\mu_0
\end{gather*}

\end{document}