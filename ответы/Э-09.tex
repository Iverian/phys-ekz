\documentclass[__main__.tex]{subfiles}

\begin{document}

\qtitle{Э}{09}
Электромагнитная индукция, закон Фарадея, правило Ленца, явление самоиндукции, индуктивность контура.\\ 

\begin{definition}
Электромагнитная индукция --- явление возникновения электрического тока в замкнутом проводящем контуре при изменении во времени магнитного потока, пронизывающего контур.
\end{definition}
\begin{definition}
Магнитным потоком $\Phi$ через площадь $S$ контура называется величина 
\begin{gather*}
\Phi = BS\cos\alpha,
\end{gather*}
где $B$ -- модуль вектора магнитной индукции, $\alpha$ -- угол между вектором магнитной индукции и нормалью к плоскости контура.
\end{definition}

\textbf{Закон Фарадея}\\
При изменении магнитного потока в проводящем контуре возникает ЭДС индукции $\mathcal E_{\text{ИНД}}$, равная скорости изменения магнитного потока через поверхность, ограниченную контуром, взятой со знаком минус:
\begin{gather*}
\mathcal E_{\text{ИНД}}=-\frac{\Delta \Phi}{\Delta t}
\end{gather*}
\textbf{Правило Ленца}\\
Индукционный ток, возбуждаемый в замкнутом контуре при изменении магнитного потока, всегда направлен так, что создаваемое им магнитное поле препятствует изменению магнитного потока, вызывающего индукционный ток.\\

\begin{definition}
Явление самоиндукции - это возникновение в проводящем контуре ЭДС, создаваемой вследствие изменения силы тока в самом контуре.
\end{definition}
\begin{definition}
Индуктивность (или коэффициент самоиндукции) — коэффициент пропорциональности между электрическим током, текущим в каком-либо замкнутом контуре, и полным магнитным потоком, называемым также потокосцеплением, создаваемым этим током через поверхность, краем которой является этот контур.

Индуктивность является электрической инерцией, подобной механической инерции тел. А вот мерой этой электрической инерции как свойства проводника может служить служить ЭДС самоиндукции. Характеризуется свойством проводника противодействовать появлению, прекращению и всякому изменению электрического тока в нём.
\begin{gather*}
\Phi=LI,
\end{gather*}
где $L$ -- индуктивность контура, $I$ -- сила тока в контуре.
\end{definition}
\end{document}