\documentclass[__main__.tex]{subfiles}

\begin{document}

\qtitle{К}{40}
Покажите, что два наблюдаемых оператора коммутируют тогда и только тогда, когда имеют общую систему собственных функций. Приведите примеры совместных и несовместных наблюдаемых величин.\\

\begin{theorem}
    Два наблюдаемых оператора коммутируют тогда и только тогда когда имеют общую систему собственных функций.
\end{theorem}
Физический смысл этого утверждения: динамические переменные представляемые этими двумя наблюдаемыми операторами могут быть одновременно точно измерены(такие переменные называются \textbf{совместными})\\

В качетсве примера величин, для которых эти условия выполняются могут служить $x$ и $p_y$:
\begin{gather*}
    [\hat{x},\hat{p_y}] = \left[\hat{x},\frac{\hbar}{i}\frac{\partial}{\partial y}\right] = 0
\end{gather*}
Таким образом $x$ и $p_y$ - могут быть одновременно измерены.\\
Величины $x$ и $p_x$ не выполняют условия теоремы:
\begin{gather*}
    [\hat{x},\hat{p_x}] = \left[\hat{x},\frac{\hbar}{i}\frac{\partial}{\partial x}\right] = i\hbar
\end{gather*}
следовательно они не могут быть одновременно измерены
\begin{proof}
    $\Rightarrow$) Если 2 оператора коммутируют, то они имеют общую систему собственных функций.\\\\
    $[\hat{A},\hat{B}]=0$ и пусть ${\varphi_n}$ - полная система функций оператора $\hat{A}$, т.е $\hat{A}\varphi_n = a_n\varphi_n \; \Rightarrow \; \hat{B}\hat{A}\varphi_n = \hat{B}a_n\varphi_n \; \Rightarrow \; \hat{A}(\hat{B}\varphi_n) = a_n(\hat{B}\varphi_n) \; \Rightarrow \; \hat{B}\varphi_n$ - собственные функции оператора $\hat{A}$ отвечающие собственным значениям $a_n$.  \\\\

    Если собственные значения оператора $\hat{A}$ невырожденные, то $(\hat{B}\varphi_n)$ может отличаться от собственных функций $\varphi_n$ оператора $\hat{A}$ только числовым множителем. Это утверждение может быть записанно как $\hat{B}\varphi_n = b_n\varphi_n$ .Таким образом получаем, что $\varphi_n$ - собственные функции $\hat{B}$ тоже.\\\\
    $\Leftarrow$) Если 2 физические величины A и B одновременно могут иметь определенные значения $a_n$ и $b_n$, то $[\hat{A},\hat{B}] = 0$.\\

    Это означает, что:
    \begin{gather*}
        \hat{A}\varphi_n = a_n\varphi_n \;\; \Rightarrow \; \hat{B}\hat{A}\varphi_n=\hat{B}a_n\varphi_n=a_n\hat{B}\varphi_n=a_nb_n\varphi_n\\
        \hat{B}\varphi_n = b_n\varphi_n \;\; \Rightarrow \;
        \hat{A}\hat{B}\varphi_n=\hat{A}b_n\varphi_n = b_n\hat{A}\varphi_n = b_na_n\varphi_n
    \end{gather*}
\end{proof}

\end{document}