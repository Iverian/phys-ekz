\documentclass[__main__.tex]{subfiles}

\begin{document}

\qtitle{П}{01}
4-векторный потенциал и тензор Максвелла. Связь полей $\vec{E}$ и $\vec{B}$ с 4-векторным потенциалом $A^\mu$. Действие для электромагнитного поля. Преобразования полей при переходе из одной инерциальной системы отсчёта в другую. Релятивистские инварианты электромагнитного поля.\\

\begin{definition}
    Свойства поля характеризуются 4-вектором $A_i$, так называемым 4-потенциалом, компоненты которого являются функциями координат и времени. Эти величины входят в действие ввиде члена:
    $$-\frac{e}{c}\int_{a}^{b}A_idx^i$$
    где $A_i$ берутся в точках мировой линии частицы.
\end{definition}
\begin{definition}
    Тензором Максвелла называется:
    $$F_{\alpha\beta}=
        \begin{pmatrix}
            0   & -E_x & -E_y & -E_z \\
            E_x & 0    & B_z  & -B_y \\
            E_y & -B_z & 0    & B_x  \\
            E_z & B_y  & -B_y & 0
        \end{pmatrix}$$
\end{definition}
где $\bar E$ - вектор напряжения электрического поля, а $\bar B$ - магнитного.
Каждая компонента считается как:
$$
    F_{\alpha\beta}=\frac{\partial A_\beta}{\partial x^{\alpha}}
    -\frac{\partial A_\alpha}{\partial x^{\beta}},
$$
где $A$ - действие.

\textbf{Связь полей $\vec E$ и $\vec B$ с 4-векторным потенциалом $A^\mu$}
\begin{flalign*}
    \left. \begin{matrix}
        -E_x = F_{01} = \partial_0\mathcal A_1 - \partial_1 \mathcal A_0 \\
        -E_y = F_{02} = \partial_0\mathcal A_2 - \partial_2 \mathcal A_0 \\
        -E_z = F_{03} = \partial_0\mathcal A_3 - \partial_3 \mathcal A_0 \\
    \end{matrix}\right\}& \qquad \vec E = -\nabla\phi - \partial_t \vec \mathcal A\\
    \left. \begin{matrix}
        B_x = F_{23} = \partial_2\mathcal A_3 - \partial_3 \mathcal A_2  \\
        -B_y = F_{13} = \partial_1\mathcal A_3 - \partial_3 \mathcal A_1 \\
        B_z = F_{12} = \partial_1\mathcal A_2 - \partial_2 \mathcal A_1  \\
    \end{matrix}\right\}& \qquad \vec B = \nabla\times \vec \mathcal A
\end{flalign*}
\begin{gather*}
    p_\alpha = (- \mathcal E, \vec p)
    \frac{dp_\alpha}{dr} = qF_{\alpha\beta}\frac{dx^\beta}{dr};\\
    dr = \sqrt{1-v^2}dt = \frac{dt}{\gamma} \rightarrow \gamma\frac{dp_\alpha}{dt} = qF_{\alpha\beta}\gamma\frac{dx^\beta}{dt} \rightarrow \frac{dp_\alpha}{dt} = qF_{\alpha\beta}\frac{dx^\beta}{dt};\\
    \frac{d}{dt}\left(\begin{matrix}-\mathcal E\\ p_x\\p_y\\p_z\end{matrix}\right) =
    q\left(\begin{matrix}
            0   & -E_x & -E_y & -E_z \\
            E_x & 0    & B_z  & -B_y \\
            E_y & -B_z & 0    & B_x  \\
            E_z & B_y  & -B_x & 0
        \end{matrix}\right)\left(\begin{matrix}1\\v_x\\v_y\\v_z\end{matrix}\right)
\end{gather*}
или:
\begin{gather*}
    \frac{d\mathcal E}{dt} = q\vec E\cdot \vec v\\
    \frac{d\vec p}{dt} = q\vec E + q\vec v \times \vec B (\text{Сила Лоренца})
\end{gather*}
\begin{definition}
    действие для заряда в электромагнитном поле имеет вид:
    $$S=\int_{a}^{b}(-mcds-\frac{e}{c}A_idx^i)$$
\end{definition}
\textbf{Преобразование полей при переходе из одной инерциальной системы отсчёта в другую:} так как проекции векторов $\vec{E}$ и $\vec{B}$ связаны с компонентами тензора $F_{\alpha \beta}$, то формулы их преобразования при переходе из одной ИСО в другую могут быть найдены из общего правила преобразования тензорных величин:

$$
'F^{\alpha \beta} = {\Lambda^{\alpha}}_\gamma {\Lambda^{\beta}}_\delta  F^{\gamma \delta}
$$
где ${\Lambda^{\alpha}}_\beta$ - элементы матрицы преобразования Лоренца.

\textbf{Релятивистские инварианты электромагнитного поля.}
\begin{flalign}
\begin{split}
    inv
    =&
    F^2 \cdot \cdot \eta
    =
    F_{\mu\nu} e^\nu \otimes e^\mu \cdot F^{\rho\sigma} e_\rho \otimes e_\sigma \cdot \cdot \eta_{\alpha \beta} e^\alpha \otimes e^\beta
    =\\
    =&
    F_{\mu\nu} F^{\rho\sigma} \delta_{\rho}^{\nu} \delta_{\sigma}^{\alpha} \eta^{\mu\beta} \eta_{\alpha\beta}
    =
    F_{\mu\nu} F^{\nu\mu}
    =
    -2 F_{oj}F^{oj} - F_{jk}F^{jk}
    =
    2\vec{E}\cdot\vec{E} - 2\vec{B}\cdot\vec{B}
    =\\
    =&
    2(\vec{E}^2 - \vec{B}^2)
\end{split}
\end{flalign}
$$
    inv = \operatorname{det}F_{\mu\nu} = \begin{vmatrix} 0 & -E_x & -E_y & -E_z \\ E_x & 0 & B_z & -B_y \\ E_y & -B_z & 0 & B_x \\ E_z & B_y & -B_x & 0 \end{vmatrix} = {(\vec{E} \cdot \vec{B})}^2
$$

\end{document}