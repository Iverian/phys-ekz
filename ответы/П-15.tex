\documentclass[__main__.tex]{subfiles}

\begin{document}

\qtitle{П}{15}
Воспользуйтесь связью полей $\vec{E}$ и $\vec{B}$ с 4-векторным потенциалом, чтобы убедиться, что тензорная форма уравнений Максвелла <<без источников>> $\partial_{\alpha}F_{\beta\gamma}+\partial_{\gamma}F_{\alpha\beta}+\partial_{\beta}F_{\gamma\alpha}=0$ эквивалентна уравнениям $\nabla\times\vec{E}=-\partial_{t}\vec{B}$ и $\nabla\cdot\vec{B}=0$. Запишите два последних уравнения в интегральной форме.\\ 

Выпишем тензор Максвелла:
\begin{gather*}
F_{\alpha\beta}=
\begin{pmatrix}
	0 & E_x & E_y & E_z\\
	-E_x & 0 & B_z & -B_y\\
	-E_y & -B_z & 0 & B_x\\
	-E_z & B_y & -B_x & 0
\end{pmatrix}
\end{gather*}

Введём обозначение:
\begin{gather*}
	\Phi_{\alpha\beta\gamma} = \partial_{\alpha}F_{\beta\gamma}+\partial_{\gamma}F_{\alpha\beta}+\partial_{\beta}F_{\gamma\alpha}
\end{gather*}
Из курса физики вам должно быть известно, что этот тензор равен нулю и полностью антисимметричен, потому нетривиальность возникает только тогда, когда все компоненты различны, потому выписать нужно будет только 4 соотношения, а именно( далее запись $F_{12,3}$ означает диффиренцирование компоненты 12 по 3 переменной ( 1-я компонента -- время )):
\begin{gather*}
	\Phi_{123} = 0 \Rightarrow	F_{12,3}+F_{31,2}+F_{23,1} = 0,\\
	\Phi_{124} = 0 \Rightarrow F_{12,4}+F_{41,2}+F_{24,1} = 0,\\
	\Phi_{134} = 0 \Rightarrow F_{13,4}+F_{41,3}+F_{34,1} = 0,\\
	\Phi_{234} = 0 \Rightarrow F_{23,4}+F_{42,3}+F_{34,2} = 0 \Longrightarrow \\
	-\partial_y E_x + \partial_x E_y + \partial_t B_z = 0,\\
	-\partial_z E_x + \partial_x E_z - \partial_t B_y = 0,\\
	-\partial_z E_y + \partial_y E_y + \partial_t B_x = 0,\\
	\partial_z B_z + \partial_y B_y + \partial_x B_x = 0.
\end{gather*}
Первые три равенства дают нам $\nabla\times\vec{E} = -\partial_t\vec{B}$, а последнее очевидно $\nabla\cdot\vec{B} = 0.$ В интегральной форме они запишутся так:
\begin{gather*}
\int\limits_V \nabla\cdot B\;dV = \int\limits_S B\cdot dS = 0,\\
\oint\limits_l E\cdot dl = -\frac{d}{dt}\int\limits_S B\cdot dS\;.
\end{gather*}
А звучат они так:
\begin{itemize}
	\item Поток магнитной индукции через замкнутую поверхность равен нулю (магнитные заряды не существуют).
	\item Изменение потока магнитной индукции, проходящего через незамкнутую поверхность s, взятое с обратным знаком, пропорционально циркуляции электрического поля на замкнутом контуре l, который является границей поверхности s.
\end{itemize}

\end{document}