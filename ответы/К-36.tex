\documentclass[__main__.tex]{subfiles}
\begin{document}
	
	\qtitle{К}{36}
	Укажите условие, при соблюдении которого возможно произвести разделение временной и пространственных переменных в нерелятивистском временном уравнении Шрёдингера. Реализуйте процедуру разделения переменных и получите временную зависимость для стационарных состояний.\\ 
	
	\textbf{Временное уравнение Шредингера}:
	\begin{gather*}
	\hat{H}\psi(t,\vec{r}) = i\hbar\frac{\partial}{\partial t}\psi(t,\vec{r})
	\end{gather*}
	где $\hat{H} = \frac{\hat{p}^2}{2m}+U(t,\vec{r})$ - Гамильтониан или полная энергия системы.
	Уравнение Шредингера позволяет найти $\psi$ функцию в любой момент времени, если известно ее значение в начальный момент времени, т.е уравнение Шредингера выражает принцип причинности в квантовой механике.\\\\
	Пусть $U(\vec{r})$, т.е потенциальная энергия не зависит явно от времени тогда существует решение вида $\psi(t,\vec{r})=X(t)\psi(\vec{r})$ - процедура разделения переменных.
	\begin{gather*}
	\hat{H}\psi(t,\vec{r})=-\frac{\hbar^2}{2m}\Delta\psi(t,\vec{r})+U(t,\vec{r})\psi(t,\vec{r})
	\end{gather*}
	Используя процедуру разделения переменных получаем:
	\begin{gather*}
	\left[-\frac{\hbar^2}{2m}\Delta\psi(\vec{r})+U(\vec{r})\psi(\vec{r})\right]X(t)=i\hbar \psi(\vec{r})\frac{dX(t)}{dt}
	\end{gather*}
	Поделив обе части равенства на $\psi(\vec{r})X(t)$ получим:
	\begin{gather*}
	-\frac{\hbar^2}{2m}\frac{\Delta\psi(\vec{r})}{\psi(\vec{r})}+U(\vec{r})=i\hbar\frac{\frac{dX(t)}{dt}}{X(t)} = \varepsilon = const
	\end{gather*}
	Как видим левая часть равенства зависит от $\vec{r}$ и равна $\hat{H}$, а правая зависит от $t$\\\\
	Отсюда получаем \textbf{стационарное уравнение Шредингера}
	\begin{gather*}
	\hat{H}\psi(\vec{r}) = \varepsilon\psi(\vec{r})
	\end{gather*}
	
\end{document}