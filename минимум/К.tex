\documentclass[__minimum__.tex]{subfiles}

\begin{document}

\begin{comment}
+ 1) Закон Киргофа
+ 2) Закон смещения Вина
+ 3) Закон Стефана Больцмана
+ 4) Формула Рэлея-Джинса
+ 5) Формула Планка
+ 6) Соотношение неопределенностей Гейзенберга
+ 7) Постулаты квантовой механики
+ 8) Коммутатор
+ 9) Эрмитов оператор
+ 10) Постулат о вещественных велечинах
+ 11) Постулат об измерениях
+ 12) Наблюдаемый оператор
+ 13) Равенство Парсевалля
+ 14) Квантомеханические средние
+ 15) Временное и стационарное уравнения Шредингера
+ 16) Вектор плотности тока вероятности
+ 17) Теорема Эренфеста
+ 18) Дуальное пространство
+ 19) Коммутационные соотношения
+ 20) Кванто-механический гармонический осцилятор
+ 21) Постулаты Бора
+ 22) Полный и собственный момент импульса
23) Матрицы Паули
\end{comment}

\ssection{К Квантовая механика}

\begin{definition}
  Абсолютно черное тело (АЧТ) -- это тело, способное поглощать при любой температуре все падающее на него излучение всех частот.
\end{definition}

\paragraph{Законы теплового излучения.}
\begin{enumerate}
  \item
        \emph{Закон Кирхгофа.} Пусть $r_{\omega}$ --- испускающая способность. Тогда энергия, которую площадка $ds$ испускает за единицу времени на интервале частот $[\omega; \omega+d\omega]$ равна $r_\omega d\omega ds$. Поглощающая способность $a_\omega$ определяется как отношение поглощенной площадкой $ds$ на интервале $[\omega;\omega+d\omega]$ энергии к падающей на эту же площадку на том же интервале энергии ($E_{\text{пад}}$). Тогда в состоянии равновесия:
        $$
          a_\omega \cdot E_{\text{пад}} = r_\omega d\omega ds \Longrightarrow a_\omega\cdot \frac{1}{4}cu_\omega(\tau)d\omega ds = r_\omega d\omega ds,
        $$
        где $cu_\omega(\tau)d\omega$ --- плотность потока энергии в частотном интервале $[\omega; \omega+d\omega]$. Из этого и следует закон Кирхгофа:
        $$
          \frac{1}{4}cu_\omega(\tau)=\frac{r_\omega}{a_\omega}
        $$
  \item
        \emph{Закон смещения Вина.} Из термодинамических соображений:
        $$
          u_\omega(\tau)=\omega^3f\left(\frac{\omega}{\tau}\right),
          \qquad
          \omega\rightarrow\infty
          \colon
          f\left(\frac{\omega}{\tau}\right)\rightarrow{0},
        $$
        Пусть $x\equiv \frac{\omega}{\tau}$. Найдем частоту $\omega_m$, доставляющую $\max(u_\omega)$.
        $$
          3\omega_m^2f(x_m)+\frac{\omega^3}{\tau}f'(x_m)=0 \Longrightarrow x_m=C,
        $$
        где $C$ --- число, не зависящее от $\tau$ и $\omega$, следовательно:
        $$
          \lambda_m\tau=\sigma=2.90\cdot 10^{-3}\text{м}\cdot K.
        $$
  \item
        \emph{Закон Стефана - Больцмана.} Энергетическая светимость АЧТ:
        $$
          R^*=\int\limits^{\infty}\limits_0 r^*_\omega d\omega \sim \int\limits^{\infty}\limits_0 u_\omega(\tau)d\omega =\tau^4 \int\limits^{\infty}\limits_0 \frac{\omega^3}{\tau^3}f\left(\frac{\omega}{\tau}\right)d\frac{\omega}{\tau}
        $$
        $$
          R^*=\sigma \tau^4,
          \qquad
          \sigma = 5.67\cdot 10^{-8} \frac{\text{Вт}}{\text{м}^2\cdot\text{К}^4},
        $$
\end{enumerate}

\paragraph{Выражения для спектральной плотности $u_\omega(T)$}
\begin{enumerate}
  \item
        \emph{Формула Рэлея - Джинса} (получена из классической теоремы о равнораспределении энергии по степеням свободы)
        \begin{flalign}
          u_\omega(T)
          =
          \frac{\omega^2k_BT}{c^3 \pi^2}
        \end{flalign}
  \item
        \emph{Формула Планка} (получена из предположения квантуемости энергии: энергия испускается порциями $\hbar\omega$)
        \begin{flalign}
          u(\omega,T)
          =
          \frac{\omega^2}{\pi^2 \cdot  c^3}\cdot
          \frac{\hbar\omega}{e^\frac{\hbar\omega}{kT}-1}
        \end{flalign}
\end{enumerate}

\begin{definition}
  \llabel{k:def:erm}
  Оператор $\hat{A}$ называется эрмитовым (самосопряженным), если для любых волновых функций $\xi$ и $\eta$ выполняется $\left(\hat{A}\xi,\eta\right)=\left(\xi,\hat{A}\eta\right)$
\end{definition}

\paragraph{Постулаты квантовой механики:}
\begin{itemize}
  \item
        \textit{постулат о квантовых состояниях:} квантовое состояние полностью задается $\psi$ -- функцией из пространства волновых функций. $\psi$ -- функции, отличающиеся только комплексным множителем, задают одно и то же состояние;
  \item
        \textit{постулат о физических величинах:} каждой физической величине ставится в соответствие эрмитов оператор, обладающий полной системой собственных функций;
  \item
        \textit{постулат об измерениях:} пусть измеряется некоторая величина $A$ и ей в соответствие поставлен оператор $\hat{A}: A \rightarrow \hat{A}$. Если оператор эрмитов (т.е. $\hat{A}^{+}=\hat{A}$, то его собственные значения вещественны и $\hat{A}\varphi_n=a_n\varphi_n$);
  \item
        \textit{динамический постулат:} все предсказания, которые могут быть сделаны относительно различных свойств системы в данный момент времени, следуют из значения $\psi$ -- функции в этот момент времени;
\end{itemize}

\begin{definition}
  Коммутатором операторов $\hat{A}$ и $\hat{B}$ называют оператор $\left[\hat{A},\hat{B}\right]=\hat{A}\hat{B}-\hat{B}\hat{A}$.
\end{definition}

\paragraph{Обобщенное соотношение неопределенностей Гейзенберга} для двух наблюдаемых $A$ и $B$ соотношение неопределенностей имеет вид:
\begin{flalign}
  \begin{split}
    &
    \Delta{A}\Delta{B}
    \ge
    \left|\left<\frac{\left[\hat{A},\hat{B}\right]}{2}\right>\right|,
    \\
    &
    \Delta{A}=\sqrt{D[A]}
    =
    \sqrt{\left<(A-\left<A\right>)^2\right>},
    \\
  \end{split}
\end{flalign}

\paragraph{Примеры соотношений неопределенности:}
\begin{enumerate}
  \item
        \emph{соотношение неопределенностей Гейзенберга}
        \begin{gather}
          \Delta x\Delta p_{x}\geq\frac{\hbar}{2},
        \end{gather}
\end{enumerate}

\paragraph{Равенство Парсевалля:} для гильбертова пространства $H$ со скалярным произведением $\left<\cdot,\cdot\right>$ верно:
\begin{gather}
  \forall{x\in H}\colon\Vert{x}\Vert^2=\sum_{i=1}^{\infty}\left|\left<x,e_k\right>\right|^2,
\end{gather}
где $\Vert\cdot\Vert=\sqrt{\left<\cdot,\cdot\right>}$ -- норма, индуцированная скалярным произведением, $\left\{e_k\right\}$ -- ортонормированный базис $H$. К примеру, для пространства $C[0,l]$ с скалярным произведением:
\begin{gather}
  \forall{f,g\in{C[0,l]}}\left<f,g\right>=\int\limits_{0}^{l}f(x)g(x)dx,
\end{gather}
имеем ортонормированный базис $\left\{\frac{2}{l}\sin\frac{\pi{kx}}{l}\right\}$, тогда:
\begin{gather}
  \Vert{f}\Vert^2=\int\limits_{0}^{l}f^2(x)dx
  =
  \sum_{k=1}^{\infty}
  \left(\frac{2}{l}\int\limits_{0}^{l}f(x)\sin\frac{\pi{kx}}{l}dx\right)^2
\end{gather}

\paragraph{Уравнение Шредингера:} основное уравнение квантовой механики, описывающее изменение во времени состояния квантовомеханической системы, задаваемого волновой функцией $\Psi$:
\begin{gather}
  \hat{H}\Psi=i\hbar\partial_{t}\Psi,
  \llabel{k:shred}
\end{gather}
где $\hat{H}$ -- эрмитов \emph{оператор полной энергии системы (гамильтониан)}. Для одного электрона этот оператор примет вид:
\begin{gather}
  \hat{H}=\frac{\hat{p}}{2m}+\hat{U}(t,\vec{r}),
\end{gather}
где $\hat{U}$ -- оператор потенциальной энергии.

В случае, когда состояние квантовомеханической системы характеризуется определенной энергией $\varepsilon$ волновую функцию можно представить в виде:
\begin{gather}
  \Psi(t,\vec{r})=\exp\left(-\frac{i\varepsilon{t}}{\hbar}\right)\psi(\vec{r}),
\end{gather}
тогда уравнение Шредингера относительно $\psi$ примет вид:
\begin{gather}
  \hat{H}\psi=\varepsilon\psi,
  \llabel{k:stat-sh}
\end{gather}
уравнение (\lref{k:stat-sh}) называется \emph{стационарным уравнением Шредингера}.

\paragraph{Квантовомеханическое среднее:} в квантовой системе в состоянии $\Psi$ \emph{квантовомеханическим средним} наблюдаемой $A$ называется величина:
\begin{gather}
  \left<A\right>=\left<\Psi,\hat{A}\Psi\right>,
\end{gather}
\emph{временная эволюция квантовомеханического среднего} выражается в виде:
\begin{flalign}
  \begin{split}
    \frac{d}{dt}\left<A\right>
    =
    \frac{d}{dt}\left<\Psi,\hat{A}\Psi\right>
    =&
    \left.
    \left<\partial_{t}\Psi,\hat{A}\Psi\right>
    +
    \left<\Psi,\left(\partial_t\hat{A}\right)\Psi\right>
    +
    \left<\Psi,\hat{A}\partial_t{\Psi}\right>
    \right|_{\partial_{t}\Psi=\frac{1}{i\hbar}\hat{H}\Psi}
    =\\
    =&
    \left.
    \left<\frac{1}{i\hbar}\hat{H}\Psi,\hat{A}\Psi \right>
    +
    \left<\partial_{t}A\right>
    +
    \left<\Psi,\frac{1}{i\hbar}\hat{A}\hat{H}\Psi\right>
    \right|_{\left<\hat{H}\xi,\eta\right>=\left<\xi,\hat{H}\eta\right>}
    =\\
    =&
    \left.
    \left<\partial_{t}A\right>
    +
    \frac{1}{i\hbar}
    \left(
    \left<\Psi,\hat{A}\hat{H}\Psi\right>
    -
    \left<\Psi,\hat{H}\hat{A}\Psi\right>
    \right)
    \right|_{\left[\hat{A},\hat{B}\right]=\hat{A}\hat{B}-\hat{B}\hat{A}}
    =\\
    =&
    \left<\partial_{t}A\right>
    +\frac{1}{i\hbar}\left<\left[A,H\right]\right>
  \end{split}
\end{flalign}
, согласно Опр.\lref{k:def:erm} и (\lref{k:shred}).
\begin{definition}
  Наблюдаемая $A$ называется интегралом движения, если
  \begin{gather}
    \frac{d}{dt}\left<A\right>=0.
  \end{gather}
\end{definition}

\paragraph{Вектор плотности тока вероятности:} рассмотрим квантовомеханическую систему с гамильтонианом
\begin{gather}
  \hat{H}=-\frac{\hbar^2}{2m}\Delta+U(t,\vec{r}),
\end{gather}
тогда обозначим $\rho=\Psi^{*}\Psi$ -- \emph{плотность вероятности}, следовательно
\begin{flalign}
  \begin{split}
    \partial_t{\rho}
    =&
    \left.
    \Psi\partial_{t}\Psi^{*}+\Psi^{*}\partial_{t}\Psi
    \right|_{\partial_{t}\Psi=\frac{1}{i\hbar}\hat{H}\Psi}
    =\\
    =&
    \left.
    -\frac{1}{i\hbar}\Psi\hat{H}\Psi^{*}+\frac{1}{i\hbar}\Psi^{*}\hat{H}\Psi
    \right|_{\hat{H}=-\frac{\hbar^2}{2m}\Delta+U(t,\vec{r})}
    =\\
    =&
    \frac{i\hbar}{2m}
    \left(\Psi^{*}\Delta{\Psi}-\Psi\Delta{\Psi^{*}}\right)
    =\\
    =&
    -\nabla\operatorname{Re}\left(\Psi^{*}\frac{\hbar}{im}\nabla\Psi\right)
    =\\
    =&
    -\nabla\vec{j}\left(t, \vec{r}\right),
  \end{split}
\end{flalign}
где $\vec{j}=\operatorname{Re}\left(\Psi^{*}\frac{\hbar}{im}\nabla\Psi\right)$ -- \emph{вектор плотности тока вероятности}, из вышесказанного получим \emph{уравнение непрерывности}:
\begin{gather}
  \partial_{t}{\rho}+\nabla\vec{j}=0,
\end{gather}

\paragraph{Теорема Эренфеста:} квантовомеханические средние удовлетворяют тем же уравнения движения, что и классические величины:
\begin{gather}
  \frac{d}{dt}\left<q_{i}\right>=\left<\frac{dH}{d\Pi_i}\right>,
\end{gather}
где $q_{i}$ -- обобщенные координаты, $\Pi_{i}$ -- обобщенный импульс, соответствующий обобщенной координате $q_{i}$.

\begin{definition}
  Пространство $\bar{\mathbb{V}}$ называется \emph{комплекно - сопряженным} векторному пространству $\mathbb{V}$ над полем $\mathrm{C}$, если они совпадают поэлементно, имеют одинаковые операции сложения, отличается лишь операция умножения на число $\circ$:
  \begin{gather}
    \forall\alpha\in\mathbb{C}\forall\vec{x}\in\bar{\mathbb{V}}\colon\alpha\circ\vec{x}=\bar{\alpha}\cdot\vec{x},
  \end{gather}
  где $\cdot$ -- операция умножения на число в $\mathbb{V}$.
\end{definition}

\paragraph{Сопряженное (дуальное) пространство:} вспомним, что в случае вещественных евклидовых пространств сопряженное пространство $\mathrm{V}^{*}$ совпадает с $\mathrm{V}$. В случае комплексного гильбертова пространства $H$ сопряженное ему $H^{*}$ совпадает с \emph{комплексно - сопряженным пространством $\bar{H}$}.

В квантовой механике дуальное пространство используется для построения \emph{бра - кет} формализма: положим $\Psi$ -- волновая функция из гильбертова пространства $H$, такая функция называется \emph{кет - вектором} и обозначается $\left|\Psi\right>$, соотвествующая ей $\Psi^{*}$ из сопряженного пространства $H^{*}$ называется \emph{бра - вектором} и обозначается $\left<\Psi\right|$. Тогда скалярное произведение получит очень красивое обозначение: $\left.\left<\Psi_1\right|\Psi_2\right>$, а если вписать сюда эрмитов оператор $\hat{A}\colon(x,\hat{A}y)=(\hat{A}x,y)$, то естественным выглядит его ключевое свойство $\left<x\right|\hat{A}\left|y\right>$

\paragraph{Коммутационные соотношения:} как известно согласно \emph{постулату о измерениях} существуют наблюдаемые, которые нельзя измерить точно в один и тот же момент времени, классический пример $\Delta{x}\Delta{p_x}\ge\frac{\hbar}{2}$, из \emph{обобщенного соотношения неопределенностей} получим, что величины можно измерить сколь угодно точно в один и тот же момент времени, когда они коммутируют, приведем несколько известных коммутационных соотношений:
\begin{gather}
  \left[\hat{x},\hat{p}_x\right]=-i\hbar,
  \qquad
  \left[\hat{y},\hat{p}_x\right]=0,
  \\
  \left[\hat{L}_z,\hat{L}_x\right]=i\hbar\hat{L}_y,
  \qquad
  \left[\hat{L}_z,\hat{L}_y\right]=-i\hbar\hat{L}_x,
\end{gather}
где $\hat{L}$ -- оператор момента импульса ($\hat{L}_z=\hat{x}\hat{p}_y-\hat{y}\hat{p}_x$)

\paragraph{Квантовомеханический осциллятор}
Рассмотрим классический гармонический осциллятор: его полная механическая энергия примет вид
\begin{gather*}
  E = \frac{kx^2}{2}+\frac{p^2}{2m},
\end{gather*}
для перехода к квантовому аналогу заменим жесткость пружины $k$ (какие еще пружины в микромире?) на ее выражение через собственную частоту $\omega$: $k=m\omega^2$, теперь по аналогии с классическим выражением запишем квантовомеханический гамильтониан (оператор полной энергии системы):
\begin{gather*}
  \hat{H} = \frac{m\omega^2\hat{x}^2}{2}+\frac{\hat{p}^2}{2m}
\end{gather*}
Теперь для нахождения энергетического спектра нужно найти собственные значения $\hat{H}$:
\begin{gather*}
  \hat{H}|v\rangle = E|v\rangle
\end{gather*}
Для решения этой задачи воспользуемся \textit{оператором числа элементарных возбуждений} $\hat{n}=\hat{b}^{+}\hat{b}$, где $\hat{b}$ и $\hat{b}^{+}$ -- операторы такие, что $[\hat{b},\hat{b}^{+}]=1$. Введем для этой задачи оператор $\hat{b}$:
\begin{gather}
  \hat{b} = \sqrt{\frac{m\omega}{2\hbar}}\hat{x} + i\sqrt{\frac{1}{2m\omega\hbar}}\hat{p}
  \llabel{_50:b}
\end{gather}
Т.к. $\hat{x}$ и $\hat{p}$ эрмитово самосопряженные, то
\begin{gather*}
  \hat{b}^{+} = \sqrt{\frac{m\omega}{2\hbar}}\hat{x} - i\sqrt{\frac{1}{2m\omega\hbar}}\hat{p}
\end{gather*}
Проверим $[\hat{b},\hat{b}^{+}]=1$:
\begin{gather*}
  [\hat{b},\hat{b}^{+}]
  =
  \left.
  -\frac{i}{2\hbar}[\hat{x},\hat{p}]+\frac{i}{2\hbar}[\hat{p},\hat{x}]
  \right|_{[\hat{x},\hat{p}]=-[\hat{p},\hat{x}]}
  =
  \left.
  -\frac{i}{\hbar}[\hat{x},\hat{p}]
  \right|_{[\hat{x},\hat{p}]=i\hbar}
  =
  1
\end{gather*}
Оператор $\hat{n}$ примет вид:
\begin{gather*}
  \hat{n}=\hat{b}^{+}\hat{b}=\frac{m\omega\hat{x}^2}{2\hbar}+\frac{\hat{p}^2}{2m\omega\hbar}-\frac{1}{2}
\end{gather*}
Тогда получим:
\begin{gather*}
  \hbar\omega\left(\hat{n}+\frac{1}{2}\right)
  =
  \frac{m\omega^2\hat{x}^2}{2}+\frac{\hat{p}^2}{2m}
  =
  \hat{H}
\end{gather*}
Обозначим за $|n\rangle$ собственные функции оператора $\hat{n}$, тогда:
\begin{gather*}
  \hat{H}|n\rangle
  =
  \hbar\omega\left(\hat{n}+\frac{1}{2}\right)|n\rangle
  =
  \hbar\omega\left(n+\frac{1}{2}\right)|n\rangle
  =
  E_n|n\rangle
\end{gather*}
Получим, что собственные функции $\hat{H}$ совпадают с СФ оператора $\hat{n}$. Несложно показать, что \emph{СЗ оператора $\hat{n}$ -- натуральные числа}, а СФ $\hat{n}$ имеют вид:
\begin{gather}
  \left|n\right>
  =
  \frac{2^{-n/2}}{x_0\sqrt{n!}}\left(x-x_0^2\partial_{x}\right)\left|0\right>,
  \qquad
  \left|0\right>
  =
  \frac{1}{\sqrt{x_0\sqrt{\pi}}}\exp\left(-\frac{x^2}{2x_0^2}\right),
\end{gather}
где $x_0=\frac{\hbar}{m\omega}$.

\paragraph{Постулаты Бора:}
\begin{enumerate}
  \item
        Атом может длительно пребывать только в стационарных состояниях, характеризующихся определенной энергией $E_1,E_2,...$. В таких состояниях атомы не излучают электромагнитных волн.
  \item
        Излучение света происходит при переходе атома из состояния с большей энергией в состояние с меньшей энергией. Энергия излученного фотона равна разности энергий стационарных состояний.
        \begin{gather*}
          \hbar\omega_{n_1 \rightarrow n_2} = E_{n_2} - E_{n_1}
        \end{gather*}
  \item
        \textit{Правило квантования круговых орбит электрона:} момент импульса электрона, вращающегося на стационарной орбите атома водорода может принимать только дискретные значения
        \begin{gather*}
          \forall n\in\mathbb{N}\colon L = n\hbar
        \end{gather*}
\end{enumerate}

\paragraph{Полный и собственный момент импульса:} определим оператор \emph{полного момента импульса $\hat{J}$} из коммутационных соотношений:
\begin{gather}
  \left[\hat{J}_{x},\hat{J}_{y}\right]=-i\hbar\hat{J}_{z},
  \\
  \left[\hat{J}_{y},\hat{J}_{z}\right]=-i\hbar\hat{J}_{x},
  \\
  \left[\hat{J}_{z},\hat{J}_{x}\right]=-i\hbar\hat{J}_{y},
\end{gather}
по классической логике новый оператор $\hat{J}$ должен совпадать с ранее введенным оператором момента импульса $\hat{L}$, хрен вам, оператор $\hat{J}$ выражается как:
\begin{gather}
  \hat{J}=\hat{L}+\hat{S},
\end{gather}
где $\hat{S}$ -- оператор \emph{собственного момента импульса (спина)}. Этот оператор вводится из-за того, что из-за херни с неопределенностями классического аналога момента импульса -- оператора $\hat{L}$ недостаточно для описания вращения квантовых систем.

Полученный оператор $\hat{S}$ не связан с движением частицы в пространстве, а связан только с ее внутренним строением.

\begin{theorem}
  Собственные значения квадрата эрмитова оператора неотрицательны.
\end{theorem}
\begin{proof}
  Пусть $|v>$ - собственный вектор оператора $\hat{A}^2$, отвечающий собственному значению $\alpha_v$: $\hat{A}^2|v> = \alpha_v|v>$\\
  Учитывая, что $\hat{A}^{+} = \hat{A}$ имеем:
  \begin{gather*}
    0 \leq \left<v|\hat{A}^2|v\right>
    =
    \left<v|\hat{A}^{+}\hat{A}|v\right>
    =
    \left<v|\alpha_v|v\right>
    =
    \alpha_v\left<v|v\right>
    \geq
    0
    \quad
    \Rightarrow
    \alpha_v \geq 0
  \end{gather*}
\end{proof}

На СЗ $\hat{J}$ накладываются ограничения: т.к. $\left[\hat{J}^2,\hat{J}_z\right]=0$, то их СФ $\left|jm\right>$ совпадают, тогда если:
\begin{flalign}
  \begin{split}
    &
    \hat{J}^2\left|jm\right>=j(j+1)\left|jm\right>,
    \\
    &
    \hat{J}_z\left|jm\right>=m\left|jm\right>,
  \end{split}
\end{flalign}
тогда
\begin{gather}
  -j\le m \le j,
\end{gather}

\end{document}