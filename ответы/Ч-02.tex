\documentclass[__main__.tex]{subfiles}

\begin{document}

\qtitle{Ч}{02}
Сформулируйте принцип неразличимости частиц одного сорта. Продемонстрируйте работу постулата симметризации на системе из двух тождественных (а) бозонов, (б) фермионов.\\ 

\textbf{Принцип неразличимости частиц одного сорта}
В классической механике всегда есть возможность различить (пронумеровать объекты и сохранить нумерацию в течении всего времени их существования) одинаковые объекты, в то время как квантовомеханическое описание делает невозможным различить одинаковые объекты.\\
Итак, возьмем кв. систему из 2-х микрообъектов. Пространство состояний 1-ого микрообъекта $V$ базис $\{|i>\}$, пространство состояний 2-го микрообъекта $W$ базис $\{|j>\}$, пространство состояний из 2-х микрообъектов $X = V \bigotimes W$ и базис $\{|i> |j>\}$. Далее любой вектор состояния из пространств $V,W$ можно предствить в виде $|V> = \sum_i \nu_i |i>$ или  $|W> = \sum_j \omega_j |j>$ соответственно( где $\nu_i, \omega_j$ - комплексные коэффициенты) эти состояния называют \textit{одночастичными состояниями}, а  $|X> = \sum_{ij} X_{ij} |i> |j>$ называют \textit{двухчастиичными состояниями}.\\
Оператор перестановки, в основе его определения лежит действие на базисные векторы $|i> |j>$. Это означает, что под действием оператора на 2-хчастичное состояние, в котором 1-я частица находилась в состоянии $|i>$, а 2-я в состоянии $|j>$, получается 2-хчастичное состояние, в котором 1-я частица находится в состоянии $|j>$, а 2-я в состоянии $|i>$.\\
\begin{gather}
\hat{P}|i>|j> \equiv |j>|i> \Longrightarrow <i'|<j'|\hat{P}|i>|j> = <i'|<j'||j>|i> = <i'|j><j'|i>\\
<i|<j|\hat{P}^+=<j|<i| \Longrightarrow <i'|<j'|\hat{P}^+=<j'|<i'| \Longrightarrow <i'|<j'|\hat{P}^+|i>|j> = <j'|<i'||i>|j> = <j'|i><i'|j>\\
\Longrightarrow \hat{P}^+=\hat{P}
\end{gather}
где $\hat{P}$ - эрмитов оператор и так же унитарен. Два раза преминив $\hat{P}$ - ничего не произойдет.\\
\begin{gather}
\hat{P}^2 = \hat{1} \Longrightarrow \hat{P} = \hat{P}^{-1} \Longrightarrow \hat{P}^+ = \hat{P}^{-1}
\end{gather}
Обозначим $|\widetilde{x}> = \hat{P}|x>$ для любых физ. величин $A$ состояний системы $|x>$.\\
Не существует наблюдаемых отличий между системой в состоянии $|x>$ и состемой в состоянии $|\widetilde{x}> = \hat{P}|x>$:
\begin{gather}
\llabel{ch02:1}
<x|\hat{A}|x> = <\widetilde{x}|\hat{A}|\widetilde{x}>
\end{gather}
Полученное нами выражение \lref{ch02:1} и есть \textit{математическая формулировка принципа неразличимости одинаковых частиц}.
\begin{definition}
Симметричная вектор функция описывает частицы, называемые бозонами.\\
Антисимметричная вектор функция описывает частицы, называемые фермионами.\\
\end{definition}
\textbf{Постулат симметризации}
\begin{definition}
Состояния системы, содержащей N тождественных частиц, будут все либо симметричными, либо антисимметричными относительно перестановок этих N частиц.\\
\end{definition}
$S_{ij} = \frac{1}{2}(1+\hat{P}_{ij})$ - оператор симметризации.\\
$A_{ij} = \frac{1}{2}(1-\hat{P}_{ij})$ - оператор антисимметризации.\\
Для двух бозонов:\\
\begin{gather}
\sqrt{\frac{N!}{n_1!n_2!...n_x!...}}S|q_1^{n_1},q_2^{n_2}...q_x^{n_x}...>
\end{gather}
где $|q_1^{n_1},q_2^{n_2}>$ - базисный вектор, в котором первые $n_1$ частиц находятся в состоянии $|q_1>$, вторые $n_2$ частиц в состоянии $|q_2>$.\\
Перестановка двух частиц, находящихся в одном и том же состоянии, не изменяет вектор $|q_1^{n_1},q_2^{n_2}...q_x^{n_x}...>$, следовательно не влияет на оператор симметрезации $S$.\\
\end{document}