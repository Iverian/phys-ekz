\documentclass[__main__.tex]{subfiles}

\begin{document}

\qtitle{П}{02}
Полевая версия теоремы Нётер. Сохраняющийся нётеровский ток. Канонический и симметризованный тензоры энергии импульса электро-магнитного поля. Физический смысл компонент симметризованного тензора энергии-импульса электромагнитного поля. Вектор Пойнтинга и теорема Пойнтинга.\\ 

Выражение для действия (функция $\mathcal{L}$ в данном случае - плотность Лагранжиана ):
\begin{gather}
A=\int Ldt=\int \mathcal{L}(A^{\alpha},\partial_\beta  A^{\alpha})dx^4,
\end{gather}
\begin{flalign}
\begin{split}
0=\delta A
=&
\int\left(\frac{\partial\mathcal{L}}{\partial A^\alpha}\delta A^\alpha+\frac{\partial\mathcal{L}}{\partial(\partial_\beta A^\alpha)}\delta(\partial_\beta A^\alpha)\right)dx^4
=\\
=&
\int\left(\frac{\partial \mathcal{L}}{\partial A^\alpha}\delta A^\alpha-\partial_\beta\frac{\partial\mathcal{L}}{\partial(\partial_\beta A^\alpha)}\delta(\partial_\beta A^\alpha)\right)dx^4,
\end{split}
\end{flalign}
Принимая во внимание произвольность, переходим к полевому уравнению Эйлера-Лагранжа: \\
\begin{gather}
\frac{\partial \mathcal{L}}{\partial  A^\alpha}-\partial_\beta\frac{\partial  \mathcal{L}}{\partial(\partial_\beta A^\alpha)}=0 \llabel{eq:lagranj}
\end{gather}
Рассмотрим подробнее плотность Лагранжиана:
$\mathcal{L} = \mathcal{L}_{emf}+ \mathcal{L}_{int}.$\\
\begin{flalign}
&
\mathcal{L}_{emf}=-\frac{1}{4}F\cdot\cdot F=-\frac{1}{4}F_{\alpha\beta}F^{\alpha\beta}=\frac{1}{4}F_{\alpha\beta}F^{\beta\alpha} \llabel{eq:lemf} \\
&
\mathcal{L}_{int}=- A\jmath=-\jmath_\alpha A^\alpha \llabel{eq:lint}
\end{flalign}
$\jmath$-плотность тока. Из пунктов (\lref{eq:lemf}) и (\lref{eq:lint}) получаем, что:
\begin{gather}
\mathcal{L}=\frac{1}{4}F_{\alpha\beta}F^{\beta\alpha}-\jmath_\alpha A^\alpha
\end{gather}
Подсчитаем $\frac{\partial \mathcal{L}}{\partial  A^\alpha}$ и $\partial_\beta\frac{\partial\mathcal{L}}{\partial(\partial_\beta\beta A^\alpha)}$ и подставим их в уравнение Эйлера-Лагранжа.
\begin{gather}
\frac{\partial \mathcal{L}}{\partial A^\alpha}=
\frac{\partial \mathcal{L}_{int}}{\partial  A^\alpha}=
-j_\nu\delta\indices{^\nu_\alpha} = -j_\alpha \llabel{eq:dlint} \\
%
\frac{\partial  \mathcal{L}}{\partial(\partial_\beta A^\alpha)}=\frac{\partial  \mathcal{L}_{emf}}{\partial(\partial_\beta A^\alpha)}=
\frac{\partial\frac{1}{4}F_{\mu\nu}F^{\mu\nu}}{\partial(\partial_\beta A^\alpha)} =
\frac{\partial\frac{1}{4}F_{\mu\nu}}{\partial(\partial_\beta A^\alpha)}F^{\mu\nu}+F_{\mu\nu}\frac{\partial\frac{1}{4}F^{\mu\nu}}{\partial(\partial_\beta A^\alpha)} \llabel{eq:dlda}
\end{gather}
Заметим, что
$F_{\mu\nu}\frac{\partial\frac{1}{4}F^{\mu\nu}}{\partial(\partial_\beta A^\alpha)}=\frac{\partial\frac{1}{4}F_{\mu\nu}}{\partial(\partial_\beta A^\alpha)}F^{\mu\nu}$, тогда $\frac{\partial\frac{1}{4}F_{\mu\nu}F^{\nu\mu}}{\partial(\partial_\beta A^\alpha)}=2F_{\mu\nu}\frac{\partial\frac{1}{4}F^{\mu\nu}}{\partial(\partial_\beta A^\alpha)}$
. Вернемся к (\lref{eq:dlda}):
\begin{gather}
\frac{1}{2}F_{\mu\nu}\frac{\partial^\mu A^\nu-\partial^\nu A^\mu}{{\partial(\partial_\beta A^\alpha)}}=
\frac{1}{2}F\indices{^{\mu}_\nu}\frac{\partial_\mu A^\nu}{{\partial(\partial_\beta A^\alpha)}}-\frac{1}{2}F\indices{^{\nu}_\mu}\frac{\partial_\nu A^\mu}{{\partial(\partial_\beta A^\alpha)}}=
\frac{1}{2}\tensor{F}{^{\mu}_\nu}\tensor{\delta}{^{\beta}_\mu}\tensor{\delta}{^{\nu}_\alpha}-\frac{1}{2}\tensor{F}{^{\nu}_\mu}\tensor{\delta}{^{\beta}_\nu}\tensor{\delta}{^{\mu}_\alpha}
\end{gather}
Итак, получаем:
\begin{gather}
\frac{\partial \mathcal{L}}{\partial(\partial_\beta\Alpha^\alpha)}
=
\frac{1}{2}(F^\beta_{\ \ \alpha}-F_\alpha^{\ \beta}),
\end{gather}
Подставляем (\lref{eq:dlint}) предыдущее равенство в (\lref{eq:lagranj}) и получаем уравнения <<с источниками>>:
\begin{gather}
-j-\frac{1}{2}\partial_\beta(\frac{1}{2}F^\beta_\alpha-F_\alpha^\beta)=0\\
\partial^\beta F_{\beta\alpha}=-j_\alpha \llabel{eq:dbfba}
\end{gather}
Рассмотрим подробнее систему (\lref{eq:dbfba}):
\begin{gather}
\begin{cases}
\partial^0F_{00}+\partial^1F_{10}+\partial^2F_{20}+\partial^3F_{30}=-\jmath_0 \\
\partial^0F_{01}+\partial^1F_{11}+\partial^2F_{21}+\partial^3F_{31}=-\jmath_1 \\
\partial^0F_{02}+\partial^1F_{12}+\partial^2F_{22}+\partial^3F_{32}=-\jmath_2 \\
\partial^0F_{03}+\partial^1F_{13}+\partial^2F_{23}+\partial^3F_{33}=-\jmath_3
\end{cases}
\begin{cases}
\partial_xE_x+\partial_yE_y+\partial_zE_z=\rho          \\
-\partial_t(-E_x)-\partial_yB_z+\partial_zB_y=-\jmath_x \\
-\partial_t(-E_y)+\partial_xB_z+\partial_zB_x=-\jmath_y \\
-\partial_t(-E_z)-\partial_xB_y+\partial_yB_x=-\jmath_z
\end{cases}
\end{gather}
Окончательный вид уравнений <<с источниками>>:\\
\begin{gather}
\begin{cases}
\nabla\bar{E}=\rho \\
\nabla\times\bar{B}=\partial_t\bar{E}+\bar{j}
\end{cases},
\end{gather}
где $\rho$ - плотность стороннего электрического заряда, а $\bar{j}$ - плотность электрического тока. Остальные пункты приведем без доказательства, для краткости:

Первая пара уравнений Лагранжа имеет вид:
\begin{gather}
\begin{cases}
\nabla\times\bar{E}=-\partial_t\bar{B} \\
\nabla\bar{B}=0
\end{cases}
\end{gather}

\begin{theorem}[Нётер]
	Инвариантность действия относительно некоторой непрерывной группы симметрии приводит к соответствующему закону сохранения.
\end{theorem}

Выражение для нетеровского тока, относительно симметрии трансляции:
\begin{gather}
\mathcal{J}^\mu=\left(\mathcal{L}(x)\delta_\nu^\mu-\frac{\partial \mathcal{L}}{\partial (\partial_\mu\mathcal{A}(x))}\partial_\nu\mathcal{A}(x)\right)\delta x^\nu+\frac{\partial\mathcal{L}}{\partial(\partial_\mu\mathcal{A}(x))}\delta\mathcal{A}(x),
\end{gather}

Канонический тензор энергии-импульса имеет вид:
\begin{gather*}
\underset{emf}{T}^{\mu\nu} = F^{\mu\lambda}\partial^\nu A_\lambda - \frac{1}{4}F_{\alpha\beta}F^{\alpha\beta}\eta^{\mu\nu}
\end{gather*}
\begin{gather*}
\underset{emf}{T^{\mu\nu}}
-
F^{\mu\lambda}\partial_\lambda A^\nu
=
F^{\mu\lambda}(\partial^\nu A_\lambda
-
\partial_\lambda A^\nu)
-
\frac{1}{4}F_{\alpha\beta}F^{\alpha\beta}\eta^{\mu\nu}
=
F^{\mu\lambda}F\indices{^\nu_\lambda}
-
\frac{1}{4}F_{\alpha\beta}F^{\alpha\beta}\eta^{\mu\nu}
\equiv
\underset{emf}{\Theta}^{\mu\nu}
\end{gather*}
$\underset{emf}{\Theta}^{\mu\nu}$ - симметризованый тензор энергии-импульса электромагнитного поля. В матричном виде имеет вид:
\begin{gather*}
\underset{emf}{\Theta^{\mu\nu}} =
\left[
\begin{array}{cccc}
W   & S_x         & S_y         & S_z         \\
S_x & \sigma_{xx} & \sigma_{xy} & \sigma_{xz} \\
S_y & \sigma_{yx} & \sigma_{yy} & \sigma_{yz} \\
S_z & \sigma_{zx} & \sigma_{zy} & \sigma_{zz} \\
\end{array}
\right],
\end{gather*}
где $\vec{S}=\{S_x,S_y,S_z\}=\vec{E}\times\vec{B}$ -- вектор Пойнтинга, $W$ -- плотность энергии, $\sigma_{ij}$ -- тензор напряжений.

\begin{definition}
	Вектор Пойтинга: $\vec{S} = [\vec{E} \vec{B}]$ 
\end{definition}

\begin{theorem}
	Теорема Пойтинга:\\
	Если коротко, то теорема описывает закон сохранения энергии электромагнитного поля
	\begin{gather*}
	\frac{\delta\varepsilon}{\delta t}+\nabla\vec{S} = -\vec{\gamma}\cdot\vec{E}
	\end{gather*}
	где $\varepsilon = \frac{E^2+B^2}{2}$
\end{theorem}
Канонический тензор энергии-импульса имеет вид:
\begin{gather*}
\underset{emf}{T}^{\mu\nu} = F^{\mu\lambda}\partial^\nu A_\lambda - \frac{1}{4}F_{\alpha\beta}F^{\alpha\beta}\eta^{\mu\nu}
\end{gather*}
\begin{gather*}
\underset{emf}{T^{\mu\nu}} - F^{\mu\lambda}\partial_\lambda A^\nu = F^{\mu\lambda}(\partial^\nu A_x - \partial_\lambda A^\nu) - \frac{1}{4}F_{\alpha\beta}F^{\alpha\beta}\eta^{\mu\nu} = F^{\mu\lambda}F^\nu_\lambda - \frac{1}{4}F_{\alpha\beta}F^{\alpha\beta}\eta^{\mu\nu}\equiv \underset{emf}{\Theta}^{\mu\nu}
\end{gather*}
$\underset{emf}{\Theta}^{\mu\nu}$ - симметризованый тензор энергии-импульса электромагнитного поля.\\

Выразим компоненты тензора $T^{\mu\nu}$ через напряжённости электрического и магнитного полей. C помощью значений 
$F_{\mu\nu}= 
\begin{bmatrix}
0 & E_x & E_y & E_z\\
-E_x & 0 & -H_z & H_y\\
-E_y & H_z & 0 & -H_x\\
-E_z & -H_y & H_x & 0
\end{bmatrix},
\quad F^{\mu\nu} = 
\begin{bmatrix}
0 & -E_x & -E_y & -E_z\\
E_x & 0 & -H_z & H_y\\
E_y & H_z & 0 & -H_x\\
E_z & -H_y & H_x & 0
\end{bmatrix}
$
легко убедиться в том, что $T^{00}$ совпадает, как и следовало, с плотностью энергии $W = \frac{E^2 + H^2}{8\pi}$, а компоненты $cT^{0\alpha}$ - с компонентами вектора Пойнтинга $S = \frac{c}{4\pi}[EH]$. Пространственные же компоненты $T^{\alpha\beta}$ образуют трёхмерный тензор с составляющими
\begin{gather*}
\sigma_{xx} = \frac{1}{8\pi}(E^2_y + E^2_z - E^2_x + H^2_y + H^2_z - H^2_x),\\
\sigma_{xy} = -\frac{1}{4\pi}(E_xE_y + H_xH_y)
\end{gather*}
и т. д., или
\begin{gather*}
\sigma_{\alpha\beta} = \frac{1}{4\pi}\{-E_\alpha E_\beta - H_\alpha H_\alpha + \frac{1}{2}\delta_{\alpha\beta}(E^2 + H^2)\}.
\end{gather*}
Этот трёхмерный тензор называют максвелловским тензором напряжений.

\end{document}